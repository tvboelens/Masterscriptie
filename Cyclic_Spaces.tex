\chapter{Cyclic spaces.}
\section{The cyclic category.}
In this chapter we recall the definition and basic properties of cyclic spaces, which form a
generalization of simplicial spaces. Our main reference is \cite[Chapters~6-7]{loday}.
Recall (see for example \cite[Chapter VI.7]{maclane}) 
that the simplex category $\Delta$ has objects the finite linearly ordered sets 
$[n]=\{0\le 1\le\ldots\le n\}$
and morphisms order preserving maps. There are coface maps
$$\delta_i:[n-1]\to [n], 0\le i\le n,$$
which for each $i$ is the unique injective map that misses $i$. There are also codegeneracy maps
$$\sigma_j:[n+1]\to [n], 0\le j\le n$$
which for each $j$ is the unique surjective map that maps $j$ and $j+1$ 
to the same element. The coface and codegeneracy maps satisfy the 
following cosimplicial identities
\begin{align}\label{eq:cosimplicialidentities}
\begin{split}
\delta_i\delta_{j-1} &= \delta_j\delta_i \text{ if } i < j,\\
\sigma_i\sigma_{j+1} &= \sigma_j \sigma_i \text{ if } i\le j,\\
\sigma_j \delta_i &= 
\begin{cases} 
\delta_i \sigma_{j-1}  &\text{ if } i<j\\
\mathrm{id}  &\text{ if } i=j, i = j+1\\
\delta_{i-1} \sigma_{j}  &\text{ if } i>j+1.
\end{cases}
\end{split}
\end{align}
It can be shown that the coface and codegeneracy maps together with the above relations
generate $\Delta$ \cite[Proposition~2, \pno~178]{maclane}.

\begin{mydef}
The cyclic category $\Lambda$ has as objects the finite sets $[n]$,
with morphisms generated by 
\begin{align*}
\delta_i:&[n-1]\to [n], 0\le i\le n,\\
\sigma_j:&[n+1]\to [n], 0\le j\le n,\\
\tau_n:&[n]\to [n],
\end{align*}
subject to the relations \eqref{eq:cosimplicialidentities} and the
following relations:
\begin{align}\label{eq:cocyclicidentities}
\begin{split}
\tau_n\delta_i &= \delta_{i-1} \tau_{n-1} \text{ if } 1\le i\le n, 
\text{ and }\tau_n\delta_0 = \delta_n,\\
\tau_n\sigma_i &= \sigma_{i-1}\tau_{n+1} \text{ if } 1\le i \le n,
\text{ and }\tau_n\sigma_0 = \sigma_n \tau^2_{n+1},\\
\tau_n^{n+1} &= \mathrm{id}.
\end{split}
\end{align}
\end{mydef}

We now relate the simplex category to the cyclic category. The following 
theorem is \cite[Theorem~6.1.3, \pno~203]{loday}.
\begin{thm}\label{thm:simplexcyclic}
There is a faithful embedding $\iota:\Delta\to \Lambda$. Furthermore,
\begin{enumerate}[(i)]
\item $\mathrm{Aut}_{\Lambda}([n])$ is cyclic of order $n+1$ with
generator $\tau_n$,
\item Any element of $\mathrm{Hom}_\Lambda([n],[m])$ can be uniquely written
as $\varphi\circ g$ with $\varphi \in \mathrm{Hom}_\Delta([n],[m])$
and $g\in \mathrm{Aut}_{\Lambda}([n])$
\end{enumerate}
\end{thm}

Recall that a simplicial object in a category $\mathcal{C}$ is a functor
$X_\bullet:\Delta^{op}\to \mathcal{C}$. 
Equivalently, a simplicial object is given by a family $\{X_n:n\in \NN_0\}$
of objects in $\mathcal{C}$, together with morphisms
\begin{align}
d_i: &X_n\to X_{n-1}, 0\le i\le n,\\
s_i: &X_n \to X_{n+1}, 0\le i \le n,
\end{align} 
satisfying the \textit{simplicial identities}
\begin{align}\label{eq:simplicialidentities}
\begin{split}
d_{j-1} d_i &= d_i d_j \text{ if } i < j,\\
s_{j+1}s_i &=  s_i s_j \text{ if } i\le j,\\
d_i s_j &= 
\begin{cases} 
s_{j-1} d_i   &\text{ if } i<j\\
\mathrm{id}  &\text{ if } i=j, i = j+1\\
s_{j} d_{i-1}  &\text{ if } i>j+1,
\end{cases}
\end{split}
\end{align}
which are
dual to \eqref{eq:cosimplicialidentities}.
Note the analogy to the following definition.
\begin{mydef}
A cyclic object in a category $\mathcal{C}$ is a functor
$X_\bullet:\Lambda^{op} \to \mathcal{C}$. If $\mathcal{C}$ is the category
of sets or of topological spaces, then we refer to $X_\bullet$ as a cyclic set
or cyclic space respectively.
A morphism of cyclic objects is a natural
transformation of functors.
\end{mydef}

Analagously to simplicial objects, a cyclic object is equivalently given
by a family of objects $\{X_n:n\in \NN_0\}$ in $\mathcal{C}$,
together with morphisms
\begin{align}
d_i: &X_n\to X_{n-1}, 0\le i\le n,\\
s_i: &X_n \to X_{n+1}, 0\le i \le n,\\
t_n: &X_n\to X_n,
\end{align} 
such that \eqref{eq:simplicialidentities} and the following
identities hold:

\begin{align}\label{eq:cyclicidentities}
\begin{split}
d_i t_n &=  t_{n-1} d_{i-1} \text{ if } 1\le i\le n, 
\text{ and }d_0 t_n = d_n,\\
s_i t_n &= t_{n+1} s_{i-1} \text{ if } 1\le i \le n,
\text{ and }s_0 t_n= t^2_{n+1} s_n,\\
t_n^{n+1} &= \mathrm{id}
\end{split}
\end{align}
These are of course dual to \eqref{eq:cocyclicidentities}.
We refer to \eqref{eq:simplicialidentities} together with
\eqref{eq:cyclicidentities} as the \textit{cyclic identities}.

From the definition we see that the cyclic objects in $\mathcal{C}$ form a category.
There is a fully faithful embedding from the category of
sets into the category of spaces
by endowing any set with the discrete topology. This
yields a fully faithful embedding of the category
of cyclic sets into the category of cyclic spaces.

We also note that theorem \ref{thm:simplexcyclic} gives us 
a faithful embedding $j:\Delta^{op}\to \Lambda^{op}$,
so that if $X_\bullet$ is a cyclic object in $\mathcal{C}$,
we can associate to it the simplicial object $j^\ast X_\bullet$,
which we call the underlying simplicial object of $X_\bullet$.
\begin{mydef}
The geometric realization of a cyclic space $X_\bullet$
is the realization of its underlying simplicial space. It is denoted by $|X_\bullet|$.
\end{mydef}
%we obtain
%a fully faithful embedding of the category of sets into
%the category of spaces. Thus, we can regard any cyclic
%set as a cyclic space. In particular, any property of cyclic
%spaces is inherited by cyclic sets.
%
%We also note that the embedding $$
The most important property of cyclic spaces is given by the following theorem,
which is proven in \cite[Theorem~7.1.4, \pno~229]{loday}.
\begin{thm}\label{thm:cyclicrealization}
Let $X_\bullet$ be a cyclic space.
\begin{enumerate}[(i)]
\item The geometric realization of $X_\bullet$ is equipped with a canonical
$\TT$-action.
\item With this action, geometric realization becomes a functor
from cyclic spaces to $\TT$-spaces.
\end{enumerate}
\end{thm}
%
%If $X$ and $Y$ are $\TT$-spaces, we give $X\times Y$ the diagonal action.
%This is then the  product in the categorical sense. 
It is well known
(see for example \cite[Theorem~11.5, \pno~103]{mayloopspaces})
that for any two simplicial spaces $X_\bullet$ and  $Y_\bullet$ the map
\begin{equation}\label{eq:realizationproduct}
|X_\bullet\times Y_\bullet | \xrightarrow{|\mathrm{pr}_{X_\bullet}|\times
|\mathrm{pr}_{Y_\bullet}|} |X_\bullet|\times |Y_\bullet |
\end{equation}
is a homeomorphism. If $X_\bullet$ and $Y_\bullet$ are cyclic spaces,
the above theorem implies this map is $\TT$-equivariant, since $\mathrm{pr}_{X_\bullet}$ and 
$\mathrm{pr}_{Y_\bullet}$ are morphisms of cyclic spaces. Hence,
we have the following proposition.

\begin{prop}\label{prop:cyclicproduct}
Let $X_\bullet$ and $Y_\bullet$ be cyclic spaces. There is a natural 
$\TT$-equivariant homeomorphism
$$
\left| X_\bullet \times Y_\bullet \right|\xrightarrow{\cong}
\left| X_\bullet\right| \times \left| Y_\bullet \right|.$$
In other words, geometric realization is a strict monoidal functor.
\end{prop}

%\begin{proof}
%By direct inspection the following diagram commutes
%$$
%\begin{tikzcd}[column sep = huge, row sep = large]
%\lvert C_\bullet \rvert \times \lvert X_\bullet \times Y_\bullet \rvert
%\arrow{d}{\Delta \times \varphi_{X_\bullet,Y_\bullet}}
%&\lvert F(X_\bullet \times Y_\bullet) \rvert
%\arrow{l}[swap]{ p_{X_\bullet \times Y_\bullet}}
%\arrow{d}{| F( \mathrm{pr}_{X_\bullet}) |\times | F(\mathrm{pr}_{Y_\bullet}) |}
%\arrow{r}{|\mathrm{ev}_{X_\bullet \times Y_\bullet}|}
%&\lvert X_\bullet \times Y_\bullet \rvert
%\arrow{d}{\varphi_{X_\bullet, Y_\bullet}}\\
%\lvert C_\bullet \rvert \times \lvert C_\bullet \rvert \times \lvert X_\bullet \rvert \times \lvert Y_\bullet \rvert
%& \lvert F(X_\bullet) \rvert \times \lvert F(Y_\bullet) \rvert
%\arrow{l}[swap]{p_{X_\bullet} \times p_{Y_\bullet}}
%\arrow{r}{|\mathrm{ev}_{X_\bullet}| \times |\mathrm{ev}_{Y_\bullet}|}
%& \lvert X_\bullet \rvert \times \lvert Y_\bullet \rvert
%\end{tikzcd}.
%$$
%The claim follows immediately from the definition of the action.
%\end{proof}

\section{The nerve of a category.}
In this section we fix a cartesian monoidal category
$(\mathcal{V},\times,\ast)$ and assume that $\mathcal{V}$ has all
small coproducts. We recall that $\ast $ denotes a final object of $\mathcal{V}$.
Let $\mathcal{C}$ be a small $\mathcal{V}$-category. Recall that the nerve of $\mathcal{C}$
is the simplicial $\mathcal{V}$-object with 
$k$-simplices
\begin{equation}\label{eq:nervesimplices}
N(\mathcal{C})_k = 
\coprod_{c_0,\ldots, c_n\in \mathrm{ obj } \mathcal{C}}
\mathcal{C}(c_0,c_1)\times \cdots \times \mathcal{C}(c_{k-1}, c_k).
\end{equation}
for $k\ge 1$ and $0$-simplices
\begin{equation}\label{eq:nervezerosimplices}
N(\mathcal{C})_0 = 
\coprod_{c\in \mathrm{ obj } \mathcal{C}}
\ast.
\end{equation}
The $i$-th degeneracy map is induced by the identity element
\begin{equation*}\begin{tikzcd}
\mathcal{C}(c_0,c_1)\times \cdots \times \mathcal{C}(c_{k-1}, c_k)\\
 \mathcal{C}(c_0,c_1)\times \cdots \times \mathcal{C}(c_{i-1}, c_i) \times \ast
\times \mathcal{C}(c_{i}, c_{i+1})\times
\cdots \times \mathcal{C}(c_{k-1}, c_k)
\arrow[swap]{u}{\cong}\arrow{d}\\
\mathcal{C}(c_0,c_1)\times \cdots \times \mathcal{C}(c_{i-1}, c_i) \times \mathcal{C}(c_{i}, c_i)
\times \mathcal{C}(c_{i}, c_{i+1})\times
\cdots \times \mathcal{C}(c_{k-1}, c_k).
\end{tikzcd}
\end{equation*}
For any object $c$ of $\mathcal{C}$ we denote by 
\[
\varphi_c:\ast\to \coprod_{c\in \mathrm{ obj } \mathcal{C}}\ast
\]
the canonical map into the coproduct. Then
$d_i:N(\mathcal{C})_1\to N(\mathcal{C})_0$
is the map induced by
\[
\mathcal{C}(c_0,c_i)\to \ast\xrightarrow{\varphi_{c_i}} \coprod_{c\in \mathrm{ obj } \mathcal{C}}\ast.
\]
%For $0<i<k$ the $i$-th face map is induced by the composition map
%\[
%\mathcal{C}(c_0,c_1)\wedge \cdots \wedge \mathcal{C}(c_{k-1}, c_k)
%\to \mathcal{C}(c_0,c_1)\wedge \cdots \mathcal{C}(c_{i-1}, c_i)
%\wedge \mathcal{C}(c_i, c_{i+2})\wedge \cdots
% \times \mathcal{C}(c_{k-1}, c_k)
%\]
In higher degrees,  the $i$-th face map is for $0<i<k$ induced by the composition morphism
\begin{equation*}
\mathcal{C}(c_0,c_1)\times \cdots \times \mathcal{C}(c_{k-1}, c_k)\to 
\mathcal{C}(c_0,c_1)\times \cdots\times
\mathcal{C}(c_{i}, c_{i+2})\times 
 \cdots \times \mathcal{C}(c_{k-1}, c_k)
\end{equation*}
and the $0$-th and $k$-th face map are given by the projections
\begin{align*}
\mathcal{C}(c_0,c_1)\times \cdots \times \mathcal{C}(c_{k-1}, c_k)
&\to \mathcal{C}(c_1,c_2)\times \cdots \times \mathcal{C}(c_{k-1}, c_k)\\
\mathcal{C}(c_0,c_1)\times \cdots \times \mathcal{C}(c_{k-1}, c_k)
&\to \mathcal{C}(c_0,c_1)\times \cdots \times \mathcal{C}(c_{k-2}, c_{k-1})
\end{align*}
One immediately sees that this construction is functorial, that is a $\mathcal{V}$-functor 
$F: \mathcal{C}\to \mathcal{D}$ induces a map of simplicial $\mathcal{V}$-objects 
$N(F) : N(\mathcal{C})_\bullet \to N(\mathcal{D})_\bullet$. Furthermore,
permuting the factors in the product gives a natural isomorphism
of simplicial objects
\begin{equation}\label{eq:nerveproduct}
N(\mathcal{C})_\bullet \times N(\mathcal{D})_\bullet
\xrightarrow{\cong} N(\mathcal{C} \times \mathcal{D})_\bullet.
\end{equation}
%
%
%by $N(F)(f_1,\ldots, f_k) = (F(f_1),\ldots, F(f_k))$. This turns
%the nerve into a functor from the category of small categories
%to the category of simplicial sets. It follows that the classifying
%space is a functor from the category of small categories
%to the category of spaces. Finally, there is an isomorphism
%\begin{equation}\label{eq:nerveproduct}
%N(\mathcal{C})_\bullet \times N(\mathcal{D})_\bullet
%\xrightarrow{\cong} N(\mathcal{C} \times \mathcal{D})_\bullet,
%\end{equation}
%which sends $((f_1,\ldots, f_k),(g_1,\ldots, g_k))$
%to $((f_1,g_1),\ldots, (f_k,g_k))$.

A monoid object $M$ in $\mathcal{V}$ determines a $\mathcal{V}$-category
with one object and morphism object $M$ and we denote by $N(M)_\bullet$
the nerve of this category.
If $M$ is 
a commutative, the multiplication morphism
\[
\mu:M\times M\rightarrow M
\]
is a morphism of monoids, so we can compose \eqref{eq:nerveproduct} with
$N(\mu)$:
\begin{equation}\label{eq:nervemultiplication}
N(M)_\bullet\times N(M)_\bullet\cong N(M\times M)_\bullet\xrightarrow{N(\mu)} N(M)_\bullet.
\end{equation}
It follows that $N(M)_\bullet$ is a simplicial object in $\mathcal{V}\mathbf{-Mod}$. 

In the special cases $\mathcal{V} = \mathbf{Set}$ and $\mathcal{V} = \mathbf{Top}$
we can apply geometric realization to $N(\mathcal{C})_\bullet$. The
resulting space is called the \textit{classifying space} of $\mathcal{C}$
and is denoted by $B\mathcal{C}$.
In the case of a commutative (topological) monoid $M$,
we see from the remarks above that
$BM$ is a topological monoid with the multiplication given by
$$BM\times BM\xrightarrow{\cong} |N(M)_\bullet \times N(M)_\bullet|
\to BM,$$
where the first map is the inverse of \eqref{eq:realizationproduct} and the second map
is induced by \eqref{eq:nervemultiplication}. We will frequently apply this
in the special case of a (topological) group $G$.
\begin{rem}
One might try to generalize the above construction to a general
symmetric monoidal category. In this case however, a problem arises,
namely that a morphism $X\wedge Y\to X\wedge \ast$ does not in
general exist, or may not be canonical. This happens for example
in the case $\mathcal{V} = \mathbf{Top}_\ast$.
\end{rem}


We conclude this section with the description of a simplicial model
for the universal bundle $EG\to BG$ of a discrete group $G$.
We will need this to compute the cyclic nerve of the integers.
We let $EG_\bullet$ be the simplicial set with $k$-simplices
\[
EG_k = G^{k+1}
\]
and face and degenaracy maps
\begin{align}
\begin{split}
d_i(g_0,\ldots, g_n) &= 
\begin{cases}
(g_1,\ldots, g_n) &\text{ if } i = 0\\
(g_0, \ldots, g_{i-1}, g_ig_{i+1}, g_{i+2},\ldots, g_n) &\text{ if } 0<i\le n
\end{cases}\\
s_i(g_0,\ldots, g_n) &= (g_0,\ldots, g_{i-1}, 1, g_i,\ldots g_n).
\end{split}
\end{align}
The group $G$ acts simplicially on $EG_\bullet$ by
\begin{equation}
(g_0,\ldots, g_n)\cdot g = (g_0,\ldots, g_ng)
\end{equation}
and the projection
\[
p:EG_\bullet\to N(G)_\bullet, (g_0,\ldots, g_n)\mapsto (g_0,\ldots, g_{n-1})
\]
is a map of simplicial sets which factors through an isomorphism
\begin{equation}\label{eq:simplicialuniversalbundle}
EG_\bullet/G\cong N(G)_\bullet.
\end{equation}
We denote by $EG$ the geometric
realization of $EG_\bullet$. The orbits of a simplicial $G$-action
can be written as a coequalizer of simplicial sets
and since coequalizers commute with realization
we obtain from \eqref{eq:simplicialuniversalbundle}
the isomorphism $EG/G\cong BG$. One can show that the $G$-action
on $EG$ is properly discontinuous, so $|p|:EG\to BG$ is a $G$-principal covering
by \cite[Proposition~14.1.12, \pno~333]{diecktop}. 
Finally,  \cite[Lemma~4.6, \pno~290]{goerssjardine} says that
$EG_\bullet$, hence also $EG$ is contractible, showing that $|p|:EG\to BG$ is a model for
the universal bundle. Finally, we remark that $N(G)_\bullet$ and $EG_\bullet$
are simplicial commutative groups if $G$ is commutative, with multiplication at each 
simplicial level given coordinatewise.
\section{The cyclic nerve of a category.}
In the following we  need
a cyclic variant of the nerve, which plays an important role in
comparing $T(A)$ with $T(A[C_\infty])$. We generalize our
previous situation slightly and assume
$(\mathcal{V},\wedge,\ast)$ is a symmetric monoidal category
with small coproducts. We denote coproducts by $\bigvee$
instead of $\coprod$. For a small $\mathcal{V}$-category $\mathcal{C}$
we let $CN(\mathcal{C})_\bullet$ be the cyclic object in $\mathcal{V}$ with
$k$-simplices 
\begin{equation}\label{eq:cyclicnervesimplices}
CN(\mathcal{C})_k = \bigvee_{c_0,\ldots, c_n\in \mathrm{ obj } \mathcal{C}}
\mathcal{C}(c_0,c_1)\wedge \cdots \wedge \mathcal{C}(c_{k-1}, c_k)\wedge
\mathcal{C}(c_k, c_0).
\end{equation}
For $0\le i<k$ the $i$-th face map is again induced by the composition morphism
\begin{equation*}
\mathcal{C}(c_0,c_1)\wedge \cdots \wedge \mathcal{C}(c_{k-1}, c_k)\wedge \mathcal{C}(c_k, c_0) \to 
\mathcal{C}(c_0,c_1)\wedge \cdots\wedge
\mathcal{C}(c_{i}, c_{i+2})\wedge 
 \cdots \wedge \mathcal{C}(c_{k-1}, c_k)\wedge \mathcal{C}(c_k, c_0),
\end{equation*}
and the $k$-th face map is similarly induced by the composition morphism
\begin{equation*}
\mathcal{C}(c_0,c_1)\wedge \cdots \wedge \mathcal{C}(c_{k-1}, c_k)\wedge \mathcal{C}(c_k, c_0) \to 
\mathcal{C}(c_k,c_1)\wedge \mathcal{C}(c_1,c_2)\wedge\cdots
\wedge \mathcal{C}(c_{k-1}, c_k).
\end{equation*}
As before, the $i$-th degeneracy map is induced by the unit maps
\begin{equation*}\begin{tikzcd}
\mathcal{C}(c_0,c_1)\wedge \cdots \wedge \mathcal{C}(c_{k-1}, c_k)\wedge \mathcal{C}(c_k, c_0)\\
 \mathcal{C}(c_0,c_1)\wedge \cdots \wedge \mathcal{C}(c_{i-1}, c_i) \wedge \ast
\wedge \mathcal{C}(c_{i}, c_{i+1})\wedge
\cdots \wedge \mathcal{C}(c_{k-1}, c_k)\wedge \mathcal{C}(c_k, c_0)
\arrow[swap]{u}{\cong}\arrow{d}\\
\mathcal{C}(c_0,c_1)\wedge \cdots \wedge \mathcal{C}(c_{i-1}, c_i) \wedge \mathcal{C}(c_{i}, c_i)
\wedge \mathcal{C}(c_{i}, c_{i+1})\wedge
\cdots \wedge \mathcal{C}(c_{k-1}, c_k) \wedge \mathcal{C}(c_k, c_0)
\end{tikzcd}
\end{equation*}
and finally the cyclic structure is induced by the isomorphism
\[
\mathcal{C}(c_0,c_1)\wedge \cdots \wedge \mathcal{C}(c_{k-1}, c_k)\wedge \mathcal{C}(c_k, c_0)
\xrightarrow{\cong} \mathcal{C}(c_k, c_0) \wedge \mathcal{C}(c_0,c_1)\wedge \cdots \wedge \mathcal{C}(c_{k-1}, c_k).
\]
%The face and degeneracy maps are defined as
%\begin{align*}
%d_i(f_0,\ldots, f_k) &= 
%\begin{cases}
%(f_0,\ldots, f_{i-1}, f_{i+1}\circ f_i, f_{i+2},\ldots , f_k) 
%&\text{ if } i<k\\
%(f_0\circ f_k, f_1,\ldots, f_{k-1})
%&\text{ if } i=k
%\end{cases}\\
%s_i(f_0,\ldots, f_n) &= (f_0,\ldots, f_{i}, \mathrm{id}, f_{i+1},\ldots, f_n)
%\end{align*}
%and the cyclic structure is given by
%$$t_k(f_0,\ldots , f_k) = (f_k, f_0,\ldots, f_{k-1}).$$
This cyclic object is called the cyclic nerve of $\mathcal{C}$. 
Similar to the nerve, the cyclic nerve is a functor from the category
of small $\mathcal{V}$-categories to the category of cyclic objects in $\mathcal{V}$. Also,
analogously to \eqref{eq:nerveproduct} there is an isomorphism
\begin{equation}\label{eq:cnerveproduct}
CN(\mathcal{C})_\bullet \wedge CN(\mathcal{D})_\bullet
\xrightarrow{\cong} CN(\mathcal{C} \wedge \mathcal{D})_\bullet.
\end{equation}
As before, if $M$ is a monoid object in $\mathcal{V}$, we can regard
it as a $\mathcal{V}$-category with one object and Hom-object
$M$ so that we can define the cyclic nerve of $M$
as the cyclic nerve of this category.

We specialize again to the cases $\mathcal{V} = \mathbf{Set}$, $\mathcal{V} = \mathbf{Top}$
and $\mathcal{V} = \mathbf{Top}_\ast$.
Just as before, if $M$ is a commutative  monoid in $\mathcal{V}$, then the multiplication
$$\mu:M\wedge M\to M,$$
is a morphism of monoids and we compose the induced map on the cyclic nerve with 
\eqref{eq:cnerveproduct} to obtain a map of cyclic spaces:
\begin{equation}\label{eq:cnervemultiplication}
CN(M)_\bullet\wedge CN(M)_\bullet\cong CN(M\wedge M )_\bullet
\xrightarrow{CN(\mu)}
CN(M)_\bullet.
\end{equation}
Thus, $CN(M)_\bullet$ is a cyclic object in the category of (pointed) topological monoids
and we endow its geometric realization with the multiplication given by
$$|CN(M)_\bullet| \wedge |CN(M)_\bullet| \xrightarrow{\cong}
|CN(M)_\bullet \wedge CN(M)_\bullet|\to |CN(M)_\bullet|,$$
where the first map is the inverse of \eqref{eq:realizationproduct} and the second map
is induced by \eqref{eq:cnervemultiplication}. This makes $|CN(M)_\bullet|$
into a (pointed) topological monoid with $\TT$-action. Furthermore, by
proposition \ref{prop:cyclicproduct} and theorem \ref{thm:cyclicrealization}
the multiplication is $\TT$-equivariant. As before, the same remarks
apply to the special case of a topological group $G$.

\section{The relation between the nerve and the cyclic nerve of a group.}
In this section we assume all spaces
are unpointed. We fix some notation. Let $z\in \TT$. Then we denote by $m_z:\TT\to \TT$
multiplication by $z$.
\begin{mydef}
For any space $X$, the free loop space of $X$ is $\mathcal{L}X:=\mathrm{map}(\TT, X)$.
It is a $\TT$-space with the action given by
$$\TT\times \mathcal{L}X\to \mathcal{L}X, (z, f)\mapsto f\circ m_z.$$
\end{mydef}
From the  definition we see that assigning to a space its free loop space
yields a functor. Furthermore, 
%there is a natural isomorphism
%$$\mathcal{L}X\times \mathcal{L}Y\xrightarrow{\cong} \mathcal{L}(X\times Y),
%(f,g)\mapsto f\times g.$$
%Thus, 
for a topological group $G$ we can equip
$\mathcal{L}G$ with a group structure by multiplying pointwise in the target.

For any category $\mathcal{C}$, we construct a $\TT$-equivariant map
\[
\gamma: \left| CN(\mathcal{C})_\bullet\right|\rightarrow \mathcal{L}B\mathcal{C},
\]
which is natural in $\mathcal{C}$.
Consider the projection map
\begin{equation}\label{eq:nerveprojection}
\mathrm{proj}: CN(\mathcal{C})_\bullet\rightarrow N(\mathcal{C})_\bullet, (f_0,\ldots ,f_n)
\mapsto (f_1,\ldots f_n).
\end{equation}
%This is a map of simplicial sets. If $G$ is commutative, this map is multiplicative.
%We consider the free loop space $\mathcal{L}BG:=\mathrm{map}(S^1, BG)$.
%\todo{define free loop space for any space and note it is a functor}
%This space has a $\TT$-action which is induced by the canonical
%$\TT$-action on $S^1$. 
Then we let 
$$\gamma: \left| CN(\mathcal{C})_\bullet\right|\rightarrow \mathcal{L}B\mathcal{C}$$
be the adjoint of the composition
$$ \TT \times \left| CN(\mathcal{C})_\bullet\right| 
\xrightarrow{\mathrm{action}}
\left| CN(\mathcal{C})_\bullet\right|\xrightarrow{|\mathrm{proj}|}
B\mathcal{C}.$$ 
For all $w,z\in S^1, x\in \left| CN(\mathcal{C})_\bullet\right|$ we calculate
\begin{align*}
(z\cdot\gamma(x))(w)&=\gamma(x)(wz)\\
&=|\mathrm{proj}|((wz)\cdot x)\\
&=|\mathrm{proj}|(w\cdot(z\cdot x))\\
&=\gamma(z\cdot x)(w),
\end{align*}
proving that $\gamma$ is $\TT$-equivariant. It follows directly from the definition
that $\gamma$ is natural in $\mathcal{C}$.

If $\mathcal{C} = G$ is a commutative group, considered as a category with one object,
then \eqref{eq:nerveprojection} is a multiplicative map.
The following calculation shows that  in this case also $\gamma$ is
 a multiplicative map:
\begin{align*}
(\gamma(x) \gamma(y))(z) &= (\gamma(x)(z)) ( \gamma(y)(z))\\
&= (|\mathrm{proj}|(z\cdot x))  (|\mathrm{proj}|(z\cdot y))\\
&= |\mathrm{proj}|((z\cdot x)(z\cdot y))\\
&= |\mathrm{proj}|(z\cdot(xy))\\
&= \gamma(xy)(z).
\end{align*}
Here we used that the multiplication of
$\lvert CN(G)_\bullet \rvert$
is $\TT$-equivariant.

\begin{thm}
For any group $G$, the map
$$\gamma: \left| CN_\bullet(G) \right| \rightarrow \mathcal{L}BG$$
is a nonequivariant homotopy equivalence.
For any finite subgroup $C_a\subset \TT$ of order $a$, the
induced map on
$C_a$-fixed points is also a homotopy equivalence. If $G$ is commutative, $\gamma$
is a multiplicative map.
\end{thm}

\begin{proof}
That $\gamma$ is a homotopy equivalence is proven in \cite[Theorem~7.3.11, \pno~246]{loday}.
The proof of the statement for the fixed points is given in \cite[Proposition~2.6]{bhmcyclotomic}.
\end{proof}

\section{The cyclic nerve of the integers.}
We specialize to the case $G=\ZZ$ and consider the map
$E\ZZ\to \RR$ induced by
\[
\ZZ^{n+1}\times \Delta^{n}, ((k_0,\ldots, k_n),(t_0,\ldots t_n))\mapsto 
\sum_{j=0}^n\left(\sum_{i=0}^{j} t_i\right)k_j.
\]
This is a well-defined $\ZZ$-equivariant group homomorphism,
so it induces
a homotopy equivalence
$\theta: B\ZZ\rightarrow \TT$, given by 
\begin{equation}\label{eq:circleclassifying}
[((k_1,\ldots, k_n),(t_0,\ldots t_n))]\mapsto \mathrm{exp}
\left(\sum_{j=1}^n\left(\sum_{i=0}^{j-1} t_i\right)k_j\right),
\end{equation}
which in turn yields a multiplicative $\TT$-homotopy equivalence
$\mathcal{L}B\ZZ\rightarrow \mathcal{L}\TT.$ 


In the following we want to study the equivariant
homotopy type of $\mathcal{L}\TT$. The main goal is
to prove
theorem \ref{thm:cycretract}.
For all $i\in \ZZ$ we let $\TT_i$ be the space 
$\TT$ but with the action
\begin{equation}\label{eq:circlemodifiedaction}
\TT\times \TT_i\to \TT_i, (z,w)\mapsto z^iw.
\end{equation}
In particular, observe that $\TT_0$ is $\TT$ with the trivial
action and by our conventions on mapping spaces (see section \ref{sec:unstablehomotopytheory})
one sees $\mathcal{L}\TT = \map(\TT_{-1},\TT_0)$.
We define
$$\varphi_i:\TT_i\to \mathcal{L}\TT$$
as the adjoint of the map
\[
\TT_i\times \TT_{-1}\to \TT_0, (z,w)\mapsto zw^i.
\]
This is equivariant, hence $\varphi_i$ is also equivariant.
Another description for $\varphi_i$ is that it sends $z$ to the loop running $i$
times around the circle and starting at $z$. 
The $\varphi_i$  assemble to a $\TT$-map 
$\varphi:\coprod_{i\in \ZZ} \TT_i\to \mathcal{L}\TT.$ Recall that the multiplication
on $\mathcal{L}\TT$ is given by pointwise multiplication in the target.
If we define multiplication on the left hand side by
$$(z,i)\cdot (w,j) = (zw, i+j),$$
then $\varphi$ becomes a multiplicative map.
We define a retraction $r:\mathcal{L}\TT\to\coprod_{i\in \ZZ} \TT_i$
as follows. Denote by $\mathcal{L}_i\TT$ the subspace of maps of degree $i$.
Then $r$ is on $\mathcal{L}_i\TT$ given as the composition
$$\mathcal{L}_i\TT \xrightarrow{\mathrm{ev}_1} \TT_i 
\to \coprod_{j\in \ZZ} \TT_j,$$
where the second map is the inclusion into the coproduct.
Then $\varphi$ is a retract of $r$ as a map of sets and it remains to
be shown that $r$ is continuous.
\begin{lem}
\begin{enumerate}[(i)]
\item The subspaces $\mathcal{L}_i\TT$ are open and form the
path components of $\mathcal{L}\TT$.
\item The degree map $\mathrm{deg}:\mathcal{L}\TT \to \ZZ$ is continuous.
\item The map $r$ is continuous.
\end{enumerate}
\end{lem}
\begin{proof}
That the path components of $\mathcal{L}\TT$ are the subspaces
$\mathcal{L}_i\TT$ follows directly from the fact that two elements in $\mathcal{L}\TT$
are homotopic iff they have the same degree. To show that $\mathcal{L}_i\TT$
is open, denote by $\mathrm{const}_1$ the map that has constant value $1$
and put 
\[
U = \{f\in \mathcal{L}\TT: f(S^1)\subset \TT\setminus\{-1\}\},
\]
which is open by the definition of the compact-open topology. Clearly $
\mathrm{const}_1\in U\subset \mathcal{L}_0\TT$. Now, if $f\in \mathcal{L}_i\TT$,
we get $f\in f\cdot U\subset \mathcal{L}_i\TT$, since the degree is a group homomorphism,
yielding the first claim.



%and note that it suffices to show there is an open set $U\subset \mathcal{L}_0\TT$
%containing $\mathrm{const}_1$, since $\mathcal{L}\TT$ is a topological group.
%We put $$U = \{f\in \mathcal{L}\TT: f(S^1)\subset \TT\setminus\{-1\}\},$$ which
%is open by the definition of the compact-open topology. Clearly $
%\mathrm{const}_1\in U\subset \mathcal{L}_0\TT$. This shows the first claim.

The continuity of the degree map is a direct consequence of the first statement.
To see the continuity of $r$, note that its restriction to $\mathcal{L}_i\TT$
is continuous and that the first statement implies $\mathcal{L} \TT= \coprod_{i\in \ZZ}
\mathcal{L}_i\TT$.
\end{proof}
%If $i\in \ZZ\setminus \{0\}$, we
%define a map
%$$\phi_{i}:S^1/C_{|i|}\rightarrow \mathrm{map}(S^1, S^1)$$
%by letting
%$$\phi_i(zC_{|i|})(w) = (zw)^i.$$
%This is a homeomorphism onto its image with inverse
%$$\phi_{i}(S^1/C_{|i|})\xrightarrow{\mathrm{ev}_1} \TT\xrightarrow{\rho_i} \TT/C_{|i|} .$$
%Moreover, we define
%$\phi_0:S^1\rightarrow \mathrm{map}(S^1, S^1), z\mapsto c_z,$
%where $c_z$ denotes the constant map with value $z$. This is also a homeomorphism
%onto its image. We write $S^1/C_0$ instead of $S^1$ for notational convenience
%and obtain a map
%\begin{equation}
%\phi:\coprod_{i\in \ZZ} S^1/C_{|i|}\rightarrow \mathrm{map}(S^1, S^1).
%%\end{equation}
%We  want to equip $\coprod_{i\in \ZZ} \TT/C_{|i|}$
%with a group structure such that $\varphi$ becomes
%multiplicative. First note that $\varphi$ maps 
%$\coprod_{i\in \ZZ} \TT/C_{|i|}$ homeomorphically onto
%a subgroup of $\mathcal{L}$.
%Since $r\circ \varphi = \mathrm{id}$, any multiplicative structure
%must satisfy
%\begin{equation}\label{eq:multiplication}
%x\cdot y = r(\varphi(x\cdot y)) = r(\varphi(x)\varphi(y))
%\end{equation}
%for all $x,y\in \coprod_{i\in \ZZ} \TT/C_{|i|}$,
%thus we define $x\cdot y = r(\varphi(x)\varphi(y))$.
%This amounts to the following identity:
%$$zC_{|i|}\cdot wC_{|j|} = 
%\begin{cases}
%(z^i w^j)^{\frac{1}{i+j}}C_{|i+j|} &\text{ if } i\neq 0 \neq j, i+j\neq 0,\\
%z^i w^j C_0&\text{ if } i\neq 0 \neq j, i+j= 0,\\
%z wC_{|i|} &\text{ if } i\neq 0  =  j, \\
%z wC_{|j|} &\text{ if } i = 0  \neq  j, \\
%z w C_0 &\text{ if } i = 0  =  j. 
%\end{cases}$$
%%and we observe that the diagram
%%$$
%%\begin{tikzcd}
%%\coprod_{i\in \ZZ} S^1/C_{|i|}\times \coprod_{i\in \ZZ} S^1/C_{|i|}
%%\arrow{r}{\phi\times \phi}
%%\arrow{d}{\mu}
%%& \mathrm{map}(S^1, S^1)\times \mathrm{map}(S^1, S^1)
%%\arrow{d}\\
%%\coprod_{i\in \ZZ} S^1/C_{|i|}
%%\arrow{r}{\phi}
%%&\mathrm{map}(S^1, S^1),
%%\end{tikzcd}
%%$$
%%where the right horizontal arrow is the pointwise multiplication as described above, commutes.
%\begin{lem}
%Let $\mathrm{map}(\RR,\RR)_j=\{f\in \mathrm{map}(\RR,\RR): f(x+1) = f(x) + j\}$.
%For $f\in \mathrm{map}(\RR,\RR)_j$, let $\eta(f):\TT\to \TT$ denote the unique
%map such that
%$$
%\begin{tikzcd}
%\RR \arrow{r}{f}\arrow{d}{p}
%&\RR\arrow{d}{p}\\
%\TT\arrow{r}{\eta(f)}
%&\TT
%\end{tikzcd}
%$$
%commutes. Then $\eta: \mathrm{map}(\RR,\RR)_j\to \mathcal{L}\TT_j$
%is a covering map with typical fiber $\ZZ$.
%\end{lem}
%\begin{proof}
%If $K\subset \TT$ is compact and $U\subset \TT$ is open, let 
%$$M(K,U) = \{f\in \mathcal{L}\TT_j:f(K)\subset U \}\subset \mathcal{L}\TT_j.$$
%These sets form a subbase for the topology on $\mathcal{L}\TT_j$.
%Similarly, if $K\subset \RR$ is compact and $U\subset \RR$ is open, 
%we let 
%$$M(K,U) = \{f\in  \mathrm{map}(\RR,\RR)_j :f(K)\subset U \}\subset  \mathrm{map}(\RR,\RR)_j.$$
%Again, these sets form a subbasis for the topology on $ \mathrm{map}(\RR,\RR)_j$.
%The continuity of $\eta$ follows from the equality 
%$$\eta^{-1}(M(K,U)) = M(p^{-1}(K)\cap [0,1], p^{-1}(U)),$$ which is elementary to check.
%
%For given $f\in \mathrm{map}(\RR,\RR)_j$, we put $U = \TT\setminus\{1\}$ and choose
%a closed interval $$[a,b]\subset p^{-1}(f^{-1}(U))\cap [0,1].$$ Then for $K= p([a,b])$
%we have $f\in M(K,U)$ and from
%$$\eta^{-1}(M(K,U)) = M(p^{-1}(K)\cap [0,1], p^{-1}(U)) = M([a,b], \RR\setminus \ZZ)
%= \coprod_{k\in \ZZ} M([a,b], (k, k+1))$$
%we obtain that $\eta$ is trivial over $M(K,U)$.
%\end{proof}

%We fix some notation. If $X$ and $Y$ are spaces, $K\subset X$ is a compact subset
%and $V\subset Y$ an open subset, we define $M(K,V)\subset \mathrm{map}(X,Y)$
%by
%$$M(K,V) = \{f:X\rightarrow Y: f(K)\subset V\}.$$
%These sets form a subbasis for the compact open topology.
%We denote by $\map_0(X,Y)$ the mapping space with the compact open topology.
%In general $\map_0(X,Y)$ is not CGWH, even if $X$ and $Y$ are. So we
%only obtain the mapping space after retopologizing, i.e.
%$\map(X,Y) = k\map_0(X,Y)$. Furthermore, we use $\prod_{I,0}, \lim_{I,0}$
%to denote products and limits in the category of topological spaces.
We now collect two basic facts about mapping spaces, which we need for
our calculation of the cyclic nerve of the integers.
\begin{lem}\label{lem:mapinclusion}
Let $f:Y\to Z$ be an open embedding. For any space $X$ the induced map
$$f_\ast:\mathrm{map}(X,Y)\to \map(X,Z)$$
is an inclusion.
\end{lem}

\begin{proof}
Let $T$ be a space and $\psi:T\rightarrow \map(X,Y)$ a set map.
We show that $\psi$ is continuous iff $f_\ast\circ\psi$ is continuous.
Let $$\tilde \psi:T\times X\rightarrow Y$$
be the adjoint of $\psi$. Since $f$ is an open embedding, $\tilde \psi$
is continuous iff $f\circ \tilde\psi$ is continuous. The claim now follows
from the observation that $f\circ \tilde\psi$ is the adjoint of
$f_\ast\circ\psi$.
\end{proof}

\begin{lem}\label{lem:maprepresentable}
For any diagram $\{X_i:i\in I\}$ in $\mathbf{Top}$
the structure maps $\psi_j: X_j\to \colim_I X_i$  induce a natural homeomorphism
\[ \map(\colim_I X_i, Y)\cong \lim_{I^{\mathrm{op}}} \map(X_i, Y).
\]
%Let 
%$\{X_i: i\in I\}$
%be a diagram of 
%%CGWH 
%spaces. Let $X$ be the colimit 
%%in the category of CGWH spaces 
%with the structure maps
%$\psi_i:X_i \to X$ and suppose the following condition is satisfied:
%for every compact subset $K\subset X$ there are compact subsets
%$$K_{j}\subset X_{i_j}, j=1,\ldots n,$$ such that
%$K = \cup_{j=1}^n \psi_{i_j}(K_{j})$.
%Then the structure maps induce a natural map 
%$$\psi: \mathrm{map}(X, Y)\to \lim_I
%\mathrm{map}(X_i, Y)$$
%and this map is a homeomorphism.
\end{lem}

\begin{proof}
There are natural isomorphisms of sets
\begin{align*}
\mathrm{Hom}_{\mathbf{Top}}(Z,\map(\colim_I X_i, Y)) 
&\cong \mathrm{Hom}_{\mathbf{Top}}(Z\times (\colim_I X_i), Y) \\
&\cong \mathrm{Hom}_{\mathbf{Top}}(\colim_I (Z\times X_i), Y)\\
&\cong \lim_{I^{\mathrm{op}}}\mathrm{Hom}_{\mathbf{Top}}(Z\times X_i, Y)\\
&\cong \lim_{I^{\mathrm{op}}}\mathrm{Hom}_{\mathbf{Top}}(Z,\map(X_i, Y))\\
&\cong \mathrm{Hom}_{\mathbf{Top}}(Z, \lim_{I^{\mathrm{op}}} \map(X_i, Y)),
\end{align*}
so the claim follows from the Yoneda lemma.
%Throughout this proof spaces are not CGWH unless explicitly mentioned.
%For categorical reasons $\psi$ is bijective and continuous, so 
%we only need to show its inverse is continuous. We first show
%that $\psi(M(K,V))$ is open in $\lim_{I,0}\map_0(X,Y)$. We choose compact sets 
%$$K_{j}\subset X_{i_j}, j=1,\ldots n,$$ such that
%$K = \cup_{j=1}^n \psi_{i_j}(K_{j})$. Then we
%have 
%$$\psi(M(K,V)) = \left(\prod_{j = i}^n M(K_{i_j}, V)\times \prod_{i\in I\setminus \{i_1,\ldots, i_n\}}
%\map_0(X_i, Y)\right)\cap \lim_{I, 0} \map_0(X_i, Y),$$
%showing the claim. This means that $\psi: \map_0(X,Y) \to \lim_{I,0} \map_0(X, Y)$
%is a homeomorphism. 
%
%We now consider the following composition:
%$$
%\lim_I \map(X_i, Y)
%\xrightarrow{\mathrm{id}}
%\lim_{I,0} \map_0(X_i, Y)
%\xrightarrow{\psi^{-1}}
%\map_0(X,Y).
%$$
%The left map is continuous since it is the product of the maps
%$$\lim_I \map(X_i, Y)\xrightarrow{\pi_j} \map(X_j, Y)\xrightarrow{\mathrm{id}}
%\map_0(X_j, Y)$$
%and since $\mathrm{id}: kA\to A$ is continuous 
%for any space $A$. 
%The continuity of $$\psi^{-1}: \lim_I \map(X_i,Y)\to \map(X, Y)$$
% now follows from the fact that for any CGWH space $A$ and any space $B$,
%a map $f:A\to kB$ is continuous iff $f:A\to B$ is continuous 
%\cite[Corollary~1.10, \pno~2]{stricklandcgwh}.
\end{proof}

We remark that we can write $\RR = \colim_{k\in \ZZ} [k, k+1]$
and $[0,1] = \colim_{i=1,\ldots, n} [\frac{i-1}{n}, \frac in]$. 
%In both cases the
%hypothesis of the previous lemma is satisfied.
In the following, we denote by $p:\RR\to \TT$  the 
standard covering, i.e. $p(x) = \mathrm{exp}(2\pi i x)$. Furthermore,
we define
\[
\map_p(\RR,\RR) = \{f\in \map(\RR, \RR): p(f(x+k)) = p(f(x)) \text{ for all }
x\in \RR, k\in \ZZ\}.
\]

\begin{prop}\label{prop:loopspacesection}
Let $f\in \map_p(\RR,\RR)$ and denote by $\eta(f):\TT\to \TT$ the unique map
that makes 
$$
\begin{tikzcd}
\RR \arrow{r}{f}\arrow{d}{p}
&\RR\arrow{d}{p}\\
\TT\arrow{r}{\eta(f)}
&\TT
\end{tikzcd}
$$
commute. Then $\eta:\map_p(\RR,\RR)\rightarrow \mathcal{L}\TT$ is continuous.
Furthermore,  for any $f\in \mathcal{L}\TT$ there exists a neighborhood $U$
of $f$ and a local section $s:U\rightarrow \map_p(\RR,\RR)$ of $\eta$.
\end{prop}
\begin{proof}
Let $q_1:\TT\setminus\{1\}\rightarrow (0,1), q_{-1}:\TT\setminus\{-1\}\to \left(-\frac12, \frac12\right)$
be local inverses of $p$. Then the adjoint
$$\tilde \eta:\map_p(\RR,\RR)\times \TT\to \TT$$
of $\eta$ is locally given by
$$\map_p(\RR,\RR)\times \TT\setminus\{1\}\xrightarrow{\id\times q_1}
\map_p(\RR,\RR)\times (0,1)\xrightarrow{\mathrm{ev}}\RR
\xrightarrow{p} \TT$$
and
$$\map_p(\RR,\RR)\times \TT\setminus\{-1\}\xrightarrow{\id\times q_{-1}}
\map_p(\RR,\RR)\times \left(-\frac12,\frac12\right)\xrightarrow{\mathrm{ev}}\RR
\xrightarrow{p} \TT,$$
showing that $\tilde \eta$, hence $\eta$ is continuous.

Now let $f\in \map(\TT,\TT)$ and $\tilde f:\RR\to \RR$ a lift of $p\circ f$.
We choose a subdivision $$\left[\frac{i-1}{n}, \frac{i}{n}\right], i = 1,\ldots ,n$$ of $[0,1]$
such that for each $i$ we have $f(p([\frac{i-1}{n}, \frac{i}{n}]))\subset \TT\setminus\{1\}$ 
or $f(p([\frac{i-1}{n}, \frac{i}{n}]))\subset \TT\setminus\{-1\}$. We put $K_i = p([\frac{i-1}{n}, \frac{i}{n}])$
and 
$$U_i = 
\begin{cases}
\TT\setminus\{1\} &\text{ if } f(K_i)\subset \TT\setminus\{1\},\\
\TT\setminus\{-1\} &\text{ if } f(K_i)\subset \TT\setminus\{-1\}.
\end{cases}$$
We choose $a_i\in \frac12 \ZZ$ such that 
$$
\begin{tikzcd}
\left [\frac{i-1}{n}, \frac{i}{n} \right ]
\arrow{r}{\tilde f}
\arrow{d}{p}
&(a_i, a_i+1)\arrow{d}{p_i}\\
K_i\arrow{r}{f}
&U_i
\end{tikzcd}
$$
commutes. Here $p_i$ denotes the restriction of $p$ to $(a_i, a_i+1)$, which is a homeomorphism.
Since $f(p(\frac{i}{n}))\in \TT\setminus\{1, -1\}$, we have either
$\tilde f(\frac{i}{n})\in \left(a_i, a_i+\frac12\right)$ or $\tilde f(\frac{i}{n})\in \left(a_i +\frac12, a_i +1\right)$, so we can define
$$\tilde V_i = 
\begin{cases}
 \left(a_i, a_i+\frac12\right) &\text{ if } \tilde f(\frac{i}{n})\in \left(a_i, a_i+\frac12\right)\\
 \left(a_i +\frac12, a_i +1\right) &\text{ if } 
 \tilde f(\frac{i}{n})\in \left(a_i +\frac12, a_i +1\right)
\end{cases}$$
Finally, we put $V_i = p(\tilde V_i)$. The definition ensures that 
$\tilde V_i\subset (a_{i}, a_{i}+1)$.

Recall from the introduction that $M(K_i, U_i) = \{f\in \map(\TT,\TT):f(K_i)\subset U_i\}$,
which is an open subset of $\map(\TT,\TT)$ and consider the diagram
$$
\begin{tikzcd}
M(K_i, U_i)\arrow{r}\arrow{d}
&\map(K_i, U_i)\arrow{d}\\
\map(\TT,\TT)\arrow{r}
&\map(K_i, \TT),
\end{tikzcd}
$$
where the left vertical map is the inclusion, the right vertical map is induced
by the inclusion $U_i\to \TT$ and the lower horizontal map is induced
by the inclusion $K_i\to \TT$. By lemma \ref{lem:mapinclusion} the upper
horizontal map is continuous. Using this we can define $t_i:M(K_i, U_i) \to 
\map\left(\left[\frac{i-1}{i}, \frac{i}{n}\right], \RR\right)$ as the composition
\[
\begin{tikzcd}
M(K_i, U_i)\arrow[hookrightarrow]{r}
&\map(K_i, U_i)
\arrow{r}{(p_i^{-1})_\ast} 
&\map(K_i, (a_i, a_i +1))
\arrow{d}{p^\ast}\\
&\map\left(\left[\frac{i-1}{i}, \frac{i}{n}\right], \RR\right)
&\map\left(\left[\frac{i-1}{i}, \frac{i}{n}\right], (a_i, a_i +1)\right).
\arrow{l}
\end{tikzcd}
\]
If $g\in U(f):=\bigcap_{i = 1}^n M(K_i, U_i)\cap M(\{p(\frac{i}{n})\}, V_i)$, then 
for $i\ge 2$ we have
$$t_i(g)\left(\frac{i-1}{n}\right)\in \tilde V_{i-1} \subset(a_{i}, a_i+1)$$ 
and  
$$t_i(g)\left(\frac{i-1}{n}\right)\in (a_i, a_i+1).$$ By construction $p$ maps 
$t_i(g)(\frac{i-1}{n})$ and $t_i(g)(\frac{i-1}{n})$ to the same element
and since $p$ is injective on $(a_i, a_i+1)$ they are equal. By lemma \ref{lem:maprepresentable}
we obtain a map
$$s_0:U(f)
\to \lim_{i\in \{1,\ldots, n\}} \map\left(\left[\frac{i-1}{n},
\frac{i}{n}\right],\RR\right)\cong \map([0,1],\RR).$$
For $k\in \ZZ\setminus \{0\}$ we define
$$s_k:U(f)\to \map([k, k+1],\RR)$$
by
$$s_k(g)(x) = s_0(g)(x-k) + k\mathrm{deg}(g),$$
which is continuous since the degree map is continuous. This
definition ensures that $s_k(g)(k+1) = s_{k+1}(g)(k+1)$ for all $k$,
so by lemma \ref{lem:maprepresentable} we obtain a map
$$s:U(f)\to \lim_{k\in \ZZ} \map([k, k+1], \RR)\cong \map(\RR,\RR)$$
such that $p\circ s(g) = g\circ p$.
\end{proof}








\begin{prop}
The image of $\varphi$ is a $\TT$-deformation retract of $\mathcal{L}\TT$.
\end{prop}

\begin{proof}
We define a set map $\psi:\mathcal{L}\TT\times I\to \map_p(\RR,\RR)$
by
\begin{equation}\label{eq:loopspacelift}
\psi(f,t)(x) = (1-t)\tilde f(x) + t\left(\int_0^1 \tilde f(y)\,dy + \mathrm{deg}(f)x\right),
\end{equation}
where $\tilde f$ is a lift of $f\circ p$.
Using that two such lifts differ by an integer, one readily verifies that
$p(\psi(f,t)(x))$ does not depend on the choice of $\tilde f$.
We then define
\begin{equation}
H:\mathcal{L}\TT\times I\to \mathcal{L}\TT, (f,t)\mapsto (-1)^{\mathrm{deg}(f)}\eta(\psi(f,t)).
\end{equation}
By definition of $\eta$, we have for all $z\in \TT$
\begin{equation}\label{eq:loopspacedeformationretraction}
\eta(\psi(f,t))(z) = p(\psi(f,t)(\tilde z)),
\end{equation}
where $p(\tilde z) = z$. This shows that the definition of $H$ also does not
depend on the choice of $\tilde f$.

We use this fact to show $H$ is continuous. We choose a local section
$s:U\to \map_p(\RR,\RR)$ of $\eta$. We then define
$\psi_s:U\to \map_p(\RR,\RR)$ as
\begin{equation}
\psi_s(f,t)(x) = (1-t) s(f)(x) + t\left(\int_0^1  s(f)(y)\,dy + \mathrm{deg}(f)x\right)
\end{equation}
and claim that $\psi_s$ is continuous. 
Note that the 
topology of $\map([0,1], \RR)$ coincides with the topology induced
by the metric $d(f,g) = \sup_{x\in [0,1]} |f(x) - g(x)|$ 
(see \cite[Proposition~2.13, \pno~5]{stricklandcgwh}),
so that
$$\int_0^1: \map([0,1], \RR)\to \RR, f\mapsto \int_0^1 f(x) \, dx$$
is continuous.
It follows that the composition
\[
U\xrightarrow{s} \map_p(\RR,\RR)\hookrightarrow
\map(\RR,\RR)\to \map([0,1], \RR)\xrightarrow{\int_0^1}\RR,
\]
where the third map is induced by the inclusion $[0,1]\hookrightarrow \RR$,
is continuous. This yields continuity of $\psi_s$, since $s$ and
the degree map are continuous. As $H$ does not
depend on the choice of lifts, the restriction of $H$
to $U$ is given by $H(f,t) = (-1)^{\mathrm{deg}(f)}\eta(\psi_s(f,t))$
and continuity of $\eta$ implies that $H$ is continuous.

Next, we show $H$ is $\TT$-equivariant. We fix $(f,t)\in \mathcal{L}\TT\times I$
and $z\in \TT$. We choose $\tilde z\in \RR$ with $p(\tilde z) = z$
and put $a_{\tilde z}(x) = x+\tilde z$. Then $\tilde f\circ a_{\tilde z}$
is a lift of $f\circ m_z$. Using that $\tilde f(x+1) - \tilde f(x) = \mathrm{deg}(f)$ for 
all $x\in \RR$ we calculate:
\begin{align*}
\int_0^1 (\tilde f\circ a_{\tilde z})(x) \, dx &= \int_0^1 \tilde f(x+\tilde z) \, dx\\
&=\int_{\tilde z}^{1+\tilde z} \tilde f(x) \, dx\\
&=\int_0^{1} \tilde f(x) \, dx + \int_1^{1+\tilde z} \tilde f(x) \, dx - \int_0^{\tilde z} \tilde f(x) \, dx\\
&=\int_0^{1} \tilde f(x) \, dx + \int_0^{\tilde z} \tilde f(x+1) - \tilde f(x) \, dx\\
&=\int_0^{1} \tilde f(x) \, dx + \mathrm{deg}(f)\tilde z.
\end{align*}
Plugging this into \eqref{eq:loopspacelift} we see that up to a choice
we have
\begin{align*}
\psi(f\circ m_z,t)(x) &= (1-t)\tilde f(x+\tilde z) + t\left(\int_0^1 \tilde f(y)\,dy + \mathrm{deg}(f)(x+\tilde z)\right)\\
&= \psi(f,t)(\tilde z + x)
\end{align*}
Finally, if $w\in \TT$ and $\tilde w\in \RR$ such that $p(\tilde w) = w$,
we calculate using \eqref{eq:loopspacedeformationretraction}:
\begin{align*}
H(f\circ m_z,t)(w) &= (-1)^{\mathrm{deg}(f)} \eta(\psi(f\circ m_z,t))(w) 
=(-1)^{\mathrm{deg}(f)} p(\psi(f\circ m_z,t)(\tilde w)) \\
&= (-1)^{\mathrm{deg}(f)} p(\psi(f,t)(\tilde z + \tilde w))
= (-1)^{\mathrm{deg}(f)} \eta(\psi(f,t))(p(\tilde z + \tilde w))\\
&=(-1)^{\mathrm{deg}(f)}  \eta(\psi(f,t))(zw)
= (H(f,t)\circ m_z)(w),
\end{align*}
yielding the claim.

Finally, we show that $H$ is a deformation retraction. Using \eqref{eq:loopspacedeformationretraction},
\eqref{eq:loopspacelift} and the identity $p\circ \tilde f = f\circ p$, one immediately verifies
that $H(-,0)$ is the identity. Furthermore, using \eqref{eq:loopspacedeformationretraction}
we obtain
\[
H(f,1)(z) = (-1)^{\mathrm{deg}(f)}\cdot p\left(\int_0^1 f(y)\,dy\right)\cdot z^{\mathrm{deg}(f)},
\]
showing that $H(f,1)$ lies in the image of $\varphi$. Finally, we choose for 
$\varphi_i(z)$ the lift $x\mapsto ix +\tilde z$, where $p(\tilde z) = z$
and using  \eqref{eq:loopspacedeformationretraction} again,
a straightforward calculation shows $H(\varphi_i(z),1) = \varphi_i(z)$.

%We first construct a map $\psi: \mathcal{L}\TT\to \TT$ such that
%\begin{equation}\label{eq:psiequivariance}
%\psi(f\circ m_z) = z^{\mathrm{deg}(f)}\psi(f)
%\end{equation}
%for all $z\in \TT$.
%%Consider the covering map
%%$$p:\RR\to \RR, x\mapsto e^{2\pi ix}.$$
%For $f\in \mathcal{L}\TT$, let $\tilde f:\RR\to \RR$
%be a lift of $f\circ p$, i.e. $p\circ \tilde f = f\circ p$.
%Then we put $\psi(f) = p\left( \int_0^1 \tilde f(x)\, dx \right)$. This does not
%depend on the choice of $\tilde f$ and we need to verify
%that $\psi$ satisfies \eqref{eq:psiequivariance}. Let $z\in \TT$
%and choose $y\in \RR$ such that $p(y) = z$. We put $a_y(x) = x+y$
%and using that $\tilde f(x+1) - f(x) = \mathrm{deg}(f)$ for 
%all $x\in \RR$ we calculate:
%\begin{align*}
%\int_0^1 (\tilde f\circ a_y)(x) \, dx &= \int_0^1 \tilde f(x+y) \, dx\\
%&=\int_y^{1+y} \tilde f(x) \, dx\\
%&=\int_0^{1} \tilde f(x) \, dx + \int_1^{1+y} \tilde f(x) \, dx - \int_0^{y} \tilde f(x) \, dx\\
%&=\int_0^{1} \tilde f(x) \, dx + \int_0^{y} \tilde f(x+1) - \tilde f(x) \, dx\\
%&=\int_0^{1} \tilde f(x) \, dx + \mathrm{deg}(f)y.
%\end{align*}
%We obtain \eqref{eq:psiequivariance} by observing that $\tilde f\circ a_y$
%is a lift of $f\circ m_z\circ p$. A similar calculation shows
%\begin{equation}\label{eq:psistandardloop}
%\psi(\varphi_i(z)) = (-1)^iz.
%\end{equation}
%Note that the 
%topology of $\map([0,1], \RR)$ coincides with the topology induced
%by the metric $d(f,g) = \sup_{x\in [0,1]} |f(x) - g(x)|$ 
%(see \cite[Proposition~2.13, \pno~5]{stricklandcgwh}),
%so that
%$$\int_0^1: \map([0,1], \RR)\to \RR, f\mapsto \int_0^1 f(x) \, dx$$
%is continuous. For $f\in \mathcal{L}\TT$, we choose according to
%proposition~\ref{prop:loopspacesection} a local section
%$$s:U\to \map_p(\RR,\RR)$$
%of $\eta$. Then  $\psi$ is locally given as the composition
%$$U\xrightarrow{s} \map_p(\RR,\RR)\hookrightarrow
%\map(\RR,\RR)\to \map([0,1], \RR)\xrightarrow{\int_0^1}\RR
%\xrightarrow{p} \TT,$$
%showing the continuity of $\psi$ at $f$.
%
%We now consider the map
%$$\TT\times [0,1]\to \TT, (z,t)\mapsto z^t:=p(t\tilde z),$$
%where $\tilde z \in p^{-1}(z)$. Using local sections of $p$ 
%one sees that this is continuous. Also, $z^tz^s = z^{t+s}$
%and $z^0 = 1, z^1 = z$ for all $z\in \TT, s,t\in [0,1]$
%We can now define
%$$H:\mathcal{L}\TT\times [0,1] \to \mathcal{L}\TT$$
%as the adjoint of the map
%$$\mathcal{L}\TT\times [0,1] \times \TT\to \TT, (f,t,z)\mapsto 
%( f(z))^t ((-1)^{\mathrm{deg}(f)} \psi(f) z^{\mathrm{deg(f)}})^{1-t}.$$
%Using \eqref{eq:psistandardloop} one sees
%this gives the desired deformation retraction.
\end{proof}
\begin{thm}\label{thm:cycretract}
Let $\mathcal{F}$ be the family of finite subgroups of $\TT$. Then there is a
multiplicative $\mathcal{F}$-homotopy equivalence between
$|CN_\bullet(\ZZ)|$ and $\coprod_{i\in \ZZ} \TT_i$.
\end{thm}

\section{Edgewise subdivision}
We introduce the edgewise subdivision of \cite{bhmcyclotomic}, which for a cyclic space $X_\bullet$
allows us to study the actions of finite groups on $|X_\bullet|$
at the simplicial level. We note that the simplex category has
a monoidal structure given by concatenation, i.e. $[n]\vee [m] = [n+m-1]$.

\begin{mydef}
The $a$-fold subdivision functor $SD_a:\Delta\to \Delta$ is defined by
\[
SD_a([n]) = \bigvee_{i=1}^a [n] = [a(n+1)-1], SD_a(f) = f\vee\cdots\vee f.
\]
For any simplicial object $X_\bullet:\Delta^{\mathrm{op}}\to \mathcal{C}$
its $a$-fold edgewise subdivision is defined as $sd_aX_\bullet = X_\bullet
\circ SD_a^{\mathrm{op}}$. The  $a$-fold edgewise subdivision
of a cyclic object $X_\bullet$ is the  $a$-fold edgewise subdivision of its
underlying simplicial object. 
%It has a cyclic structure given by
%\begin{equation}\label{eq:subdivisionsimplicialaction}
%sd_aX_k = X_{a(k+1) - 1}\xrightarrow{t_{a(k+1)-1}^a} X_{a(k+1) - 1}
%=sd_aX_k.
%\end{equation}
\end{mydef}
The edgewise subdivision allows us to define a simplicial $C_a$-action on $sd_a X_\bullet$ via 
\begin{equation}
sd_aX_k = X_{a(k+1) - 1}\xrightarrow{t_{a(k+1)-1}^{k+1}} X_{a(k+1) - 1}
=sd_aX_k.
\end{equation}
This follows from 
the cyclic identities \eqref{eq:cyclicidentities}. The next
result is a consequence of \cite[Lemma~1.6, \pno~469]{bhmcyclotomic}
and the subsequent discussion there.
\begin{lem}
If $X_\bullet$ is a cyclic set,  $|sd_aX_\bullet|$ has a canonical
$\TT$-action, which extends the simplicial $C_a$-action.
\end{lem}


\begin{prop}\label{prop:edgewisesubdivision}
There is a natural homeomorphism of $\TT$-spaces
$D_a:|sd_aX_\bullet|\to |X_\bullet|$ induced by
\[X_{a(k+1) -1}\times \Delta^{k}\to X_{a(k+1) -1}\times \Delta^{a(k+1)-1}, 
(x,p)\mapsto (x,\frac1a(p, \ldots, p)).\]
\end{prop}

\begin{proof}
That $D_a$ is a homeomorphism is \cite[Lemma~1.1, \pno~468]{bhmcyclotomic}.
The equivariance follows from \cite[Lemma~1.11, \pno~470]{bhmcyclotomic} and 
the subsequent remarks there.
\end{proof}

\begin{cor}
There is a natural homeomorphism $|(sd_aX_\bullet)^{C_a}|\cong |X_\bullet|^{C_a}$.
\end{cor}

\begin{proof}
The simplicial space $(sd_aX_\bullet)^{C_a}$ is an equalizer
and by \cite[Corollary~11.6, \pno~105]{mayloopspaces} equalizers
commute with geometric realization.
\end{proof}

For a (pointed) topological monoid $M$, the fixed points $sd_aCN(M)_\bullet^{C_a}$ are 
easy to understand. On the level of spaces, the diagonal
\[
CN(M)_k = M^{k+1}\xrightarrow{\Delta} (M^{k+1})^a = sd_aCN(M)_k
\]
yields a homeomorphism
\[
\Delta: CN(M)_k\xrightarrow{\cong} sd_aCN(M)_k^{C_a}.
\]
The induced map on the realization is not equivariant. 
Instead, we have the following proposition.

\begin{prop}
The map $\Delta:|CN(M)_\bullet|\to \rho_{a}^\ast |sd_aCN(M)_\bullet|^{C_a}$
is a natural homeomorphism of $\TT$-spaces.
\end{prop}

\begin{proof}
It is certainly true that $\Delta$ is a homeomorphism, 
since it is the realization of an isomorphism of simplicial spaces.
The equivariance follows from \cite[Lemma~1.10, \pno~470]{bhmcyclotomic}.
\end{proof}

%\begin{lem}\label{lem:modifiedequivariance}
%Let $X_\bullet$ and $Y_\bullet$ be cyclic spaces and
%$f_\bullet :X_\bullet \to Y_\bullet$ a simplicial map such that
%\[
%f_\bullet\circ t_X = (t_Y)^a\circ f_\bullet.
%\]
%Then $f:|X_\bullet|\to |Y_\bullet|$ satisfies
%\[
%f(zx) = z^af(x) 
%\]
%for all $z\in \TT$.
%\end{lem}

For any $a$, let $P_a:\mathcal{L}X\to \mathcal{L}X$ be the $a$-th power map,
i.e. $P_a(\sigma)(z) = \sigma(z^a)$. This is not equivariant, but it induces
for $X = \TT$ an equivariant homeomorphism
\[
P_a:\mathcal{L}\TT\xrightarrow{\cong} \rho_a^{\ast}(\mathcal{L}\TT)^{C_a}.
\]
\begin{prop}\label{prop:cyclicnervefixedpoints}
The following diagram commutes up to homotopy:
\[
\begin{tikzcd}[column sep = large]
\lvert CN(\ZZ)_\bullet\rvert
\arrow{r}{\gamma}\arrow{d}{\Delta}
&\mathcal{L}B\ZZ
\arrow{r}{\mathcal{L}\theta}\arrow{d}{P_a}
&\mathcal{L}\TT
\arrow{d}{P_{a}}
&\TT_j\arrow[swap]{l}{\varphi_j}\arrow[equal]{d}\\
\rho_{a}^\ast \lvert CN(\ZZ)_\bullet\rvert^{C_{a}}
\arrow{r}{\rho_{a}^\ast\gamma^{C_{a}}}
&\rho_{a}^\ast (\mathcal{L} B\ZZ)^{C_{a}}
\arrow{r}{\rho_a^\ast \mathcal{L}\theta^{C_a}}
&\rho_{a}^\ast (\mathcal{L} \TT)^{C_{a}}
&\rho_{a}^\ast \TT_{aj}\arrow[swap]{l}{\rho_{a}^\ast \varphi_{aj}}
\end{tikzcd}
\]
\end{prop}
\begin{proof}
The right two squares obviously commute.
For the commutativity of the left square see the proof of 
\cite[Proposition~2.6, \pno~473]{bhmcyclotomic}.
\end{proof}

\begin{prop}\label{prop:cyclicnervegrading}
The map
\[
f:\ZZ\to CN(\ZZ)_0\to |CN(\ZZ)_\bullet|\xrightarrow{\gamma} \mathcal{L}B\ZZ\xrightarrow{\mathcal{L}\theta} \mathcal{L}\TT,
\]
where $\theta$ is the map from \eqref{eq:circleclassifying}, maps $1$ to the identity.
\end{prop}

\begin{proof}
Since the union of all finite subgroups is dense in $\TT$, it is enough to
show that $f(1)$ maps every element of finite order to itself. Fix $a\in \NN$
and let $g = \mathrm{exp}(\frac{2\pi i}{a} )$ be a generator of $C_a$. 
Let $\tilde \gamma$ denote the adjoint of $\gamma$. We 
note that the composition
\[\ZZ\to CN(\ZZ)_0\to |CN(\ZZ)_\bullet|
\]
 maps $1$ to $[(1,1)]$
and we show that
\[
|CN(\ZZ)_\bullet|\times \TT\xrightarrow{\tilde \gamma} B\ZZ\xrightarrow{\theta} \TT
\]
maps $([(1,1)], g^i)$ to $g^i$.
Using the definition
of $CN(\ZZ)_\bullet$ we see that $[(1,0,\ldots, 0),(\frac1a,\ldots, \frac1a)] = [(1,1)]$.
Recall from proposition \ref{prop:edgewisesubdivision} that the homeomorphism
\[
D_a:|sd_a CN(\ZZ)_\bullet|\to | CN(\ZZ)_\bullet|
\]
is $\TT$-equivariant and maps $[((1,0,\ldots, 0),1)]$ to 
$[(1,0,\ldots, 0),(\frac1a,\ldots, \frac1a)]$.
Since the $C_a$ action on the left hand side is simplicial, we obtain
\[
g^i\cdot [(1,0,\ldots, 0),\left(\frac1a,\ldots, \frac1a\right)] 
= [(0,\ldots, 0,1,0,\ldots, 0),\left(\frac1a,\ldots, \frac1a\right)],
\]
where $1$ is in the $i$-th coordinate. Hence
\[
\tilde \gamma(([(1,1)], g^i)) = 
\begin{cases}
[(0,\ldots, 0),(\frac1a,\ldots, \frac1a)] &\mathrm{ if } \, i = 0,\\
[(0,\ldots, 0,1,0,\ldots, 0),(\frac1a,\ldots, \frac1a)] &\mathrm{ if }\, i > 0,
\end{cases}
\]
and \eqref{eq:circleclassifying} maps this to $g^i$.
\end{proof}



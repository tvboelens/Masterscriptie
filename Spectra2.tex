\chapter{Equivariant homotopy theory.}
\section{Unstable equivariant homotopy theory.}
Throughout this section $G$ denotes a topological group. The categories
$\mathbf{Top}_G$ and $\mathbf{Top}_{\ast G}$ have objects
(pointed) left $G$-spaces and morphisms (pointed) continuous
maps. In the pointed case we require that $G$ acts trivially on
the basepoint. Note also that we do not require maps to be equivariant.
We equip the (smash) product with the diagonal action. If $X$ and $Y$ are
$G$-spaces, we let $G$ act on $\map(X,Y)$ by conjugation. Just as in the
non-equivariant case, the product and the mapping space
give $\mathbf{Top}_G$ the structure of a closed symmetric monoidal
category with unit the one-point space. Similarly, the smash product
and pointed mapping space give $\mathbf{Top}_{\ast G}$
a closed symmetric monoidal structure with unit $S^0$.

\subsection{Equivariant CW complexes and universal spaces.}


\begin{mydef}
Let $A$ and $X$ be $G$-spaces. We say $X$ is obtained by attaching equivariant $n$-cells
if there exists a family $\{H_j:j\in J\}$ of closed subgroups and a pushout in $\mathbf{Top}^G$:
\begin{equation}\label{eq:cwpushout}
\begin{tikzcd}
\underset{j\in J}{\coprod} G/H_j\times S^{n-1}
\arrow{r}{\varphi}\arrow[hookrightarrow]{d}
&A\arrow{d}\\
\underset{j\in J}{\coprod} G/H_j\times D^{n}
\arrow{r}{\Phi}
&X.
\end{tikzcd}
\end{equation}
\end{mydef}
In \eqref{eq:cwpushout} $G$ acts trivially on $S^{n-1}$ and $D^n$. The action on $G/H_j$ is
by left multiplication. One can verify that if $X$ is obtained from $A$ by attaching equivariant cells, there is
a canonical $G$-homeomorphism $X/A\cong \bigvee_{j\in J} G/H_{j+}\wedge S^n$.
\begin{mydef}
Let $(X,A)$ be a pair of $G$-spaces with $A$ Hausdorff. We say it is a relative $G$-CW-complex
if there is a filtration
\[
A = X_{-1}\subset X_0\subset \ldots\subset X_n\subset \ldots \subset X
\]
such that the following holds:
\begin{enumerate}
\item $X = \cup_{n = -1}^\infty X_n$.
\item $X_n$ is obtained from $X_{n-1}$ by attaching equivariant $n$-cells.
\item $X = \colim_n X_n$.
\end{enumerate}
The cellular dimension of $X$ is the smallest integer $n$ such that $X_n =X$. We refer
to the space $X_n$ as the $n$-skeleton.
In the case $A=\emptyset$ we say $X$ is a $G$-CW-complex.
\end{mydef}

We now turn to families of subgroups and universal spaces.
\begin{mydef}
A family of subgroups of the group $G$ is a set $\mathcal{F}$
of closed subgroups such that the following holds:
\begin{enumerate}
\item If $H\in \mathcal{F}$ and $K$ is conjugate to $H$, then  $K\in \mathcal{F}$.
\item If $H\in \mathcal{F}$ and $K\subset H$, then  $K\in \mathcal{F}$.
\end{enumerate}
\end{mydef}


\begin{mydef}
Let $\mathcal{F}$ be a family of subgroups of the group $G$. A universal $G$-space
for $\mathcal{F}$ is a $G$-CW-complex $E\mathcal{F}$ such that $E\mathcal{F}^H$
is empty if $H\in \mathcal{F}$ and $E\mathcal{F}^H$
is contractible if $H\not \in \mathcal{F}$.
\end{mydef}

\begin{thm}
Let $G$ be a compact Lie group and $\mathcal{F}$ a family of subgroups.
Then an universal space for $\mathcal{F}$ exists and is unique up to
$G$-homotopy equivalence.
\end{thm}

\subsection{Change of groups.}
\begin{mydef}
Let $\alpha:\Gamma\to G$ be a group homomorphism and $X$ a (pointed) $G$-space.
We denote by $\alpha^\ast X$ the $\Gamma$-space with underlying space $X$
and $\Gamma$-action given by $(\gamma,x)\mapsto \alpha(\gamma)\cdot x$. This operation 
yields the restriction functors
\begin{align}
\begin{split}
\alpha^\ast:\mathbf{Top}^G&\to \mathbf{Top}^{\Gamma},\\
\alpha^\ast: \mathbf{Top}^G_\ast&\to \mathbf{Top}^{\Gamma}_\ast.
\end{split}
\end{align}
In the case that $H$ is a subgroup and $\alpha:H\to G$ is the inclusion, we often
write $\mathrm{res}^G_H$ instead of $\alpha^\ast$.
\end{mydef}
The restriction functors admit left adjoints called induction. For our purposes
it suffices to treat the pointed case. We give an explicit construction.

\begin{mydef}

\end{mydef}
\section{Symmetric Spectra}

In the following we denote by $\Sigma_n$ the symmetric group. If we regard $S^{n}$ as the one-point compactification
of $\RR^n$, the canonical left action of $\Sigma_n$ on $\RR^n$ extends naturally to a basepoint preserving left action
on $S^{n}$.
\begin{mydef}
A symmetric spectrum is a sequence of pointed spaces $\{X_n\}_{n\ge 0}$, together with basepoint preserving left actions
of $\Sigma_n$ on $X_n$ and maps $\sigma_n:X_n\wedge S^1\rightarrow X_{n+1}$ such that (fill in details here)
\end{mydef}

\begin{bsp}[Sphere spectrum]
\end{bsp}

\begin{bsp}[Suspension spectrum]
Let $X$ be a pointed space. The associated suspension spectrum $\Sigma^\infty X$ is given
by $(\Sigma^\infty X)_n= X\wedge S^n$. The $n$-th structure map is defined to be the canonical homeomorphism
$(X\wedge S^n)\wedge S^1\cong X\wedge S^{n+1}$.
\end{bsp}
\begin{mydef}
Smash product of symmetric spectra.
\end{mydef}

\begin{mydef}
Ring spectra and module spectra over a ring spectrum.
\end{mydef}

\begin{bsp}[Eilenberg-MacLane spectrum]

\end{bsp}

\section{Equivariant Orthogonal Spectra}
In this section we collect basic definitions and facts about equivariant 
stable homotopy theory. The standard reference is \cite{mandellmay},
but see also \cite{schwedeequivariant} and \cite[Section~3-7]{rvadams}
in the case of finite groups. 
Throughout this section we assume that $G$ is a compact Lie group.
We are mostly interested in the case of the circle group $\TT$ and its
finite cyclic subgroups. 

\begin{mydef}
An orthogonal $G$-spectrum $T$ is a $\mathbf{Top}_{\ast G}$-functor
\[
T:\mathbf{Th}_G\to \mathbf{Top}_{\ast G}.
\]
A naive orthogonal $G$-spectrum $T$ is a $\mathbf{Top}_{\ast G}$-functor
\[
T:\mathbf{Th}^{\mathrm{triv}}_G\to \mathbf{Top}^G_{\ast}.
\]
A map of (naive) orthogonal $G$-spectra is a natural transformation of
$\mathbf{Top}_{\ast G}$-functors. We denote the category of orthogonal
$G$-spectra by $\mathbf{Sp}^G$ and the category of
naive orthogonal $G$-spectra by $\mathbf{NSp}^G$.
\end{mydef}
We will often drop the adjective orthogonal and simply use the expression
$G$-spectrum. The category $\mathbf{Th}_G$ is the $\mathbf{Top}_{\ast G}$-category of
Thom complexes  as described in  \cite[Section~3]{rvadams} (this description
is also valid for compact Lie groups).
We denote by $\mathbf{Th}_G^{\mathrm{triv}}$ the full subcategory
with objects the trivial $G$-representations. Then any $G$-spectrum
$T$ has an underlying naive $G$-spectrum obtained by restriction,
which we denote by $\iota^\ast T$.


%We will define orthogonal $G$-spectra
%as $\mathbf{Top}_\ast^G$-enriched functors taking values
%in $\mathbf{Top}_{\ast G}$. We first describe the indexing category.
%
%We take a $G$\textit{-representation} to mean a real inner product
%space $U$ on which $G$ acts smoothly by linear isometries. We only
%consider $G$-representations of finite or countably infinite
%dimension. For $G$-representations $U$ and $V$ we denote
%by $\mathbf{L}(U,V)$ the set of all linear isometries and topologize
%it as a subspace of the unpointed mapping space $\map(U,V)$. 
%We let $\xi(U,V)\to \mathbf{L}(U,V)$
%be the vector bundle with total space
%\[
%\xi(U,V) = \{(f,v)\in \mathbf{L}(U,V)\times V:f(U)\perp v\}
%\]
%and bundle map projection onto the first factor. We let $G$
%act on $\xi(U,V)$ via 
%\begin{equation}\label{eq:thomspaceaction}
%(g,(f,v))\mapsto (gf(g^{-1}-),gv).
%\end{equation}
%
%\begin{mydef}
%Let $U$ and $V$ be finite dimensional $G$-representations.
%We define $\mathbf{Th}(U,V)$ as the Thom space of the bundle
%$\xi(U,V)\to \mathbf{L}(U,V)$. Explicitly, it is the quotient space
%of the fibrewise one-point compactification obtained by collapsing
%all the points at infinity to a basepoint.
%It has a $G$-action induced by \eqref{eq:thomspaceaction}.
%\end{mydef}
%
%Composition of isometries induces $G$-equivariant bundle maps
%\[
%\xi(V,W)\times \xi(U,V)\to \xi(U,W), ((f,w),(f',v))\mapsto (f'\circ f, w + f'(v))
%\]
%and this descends to a map
%\begin{equation}\label{eq:thomspacecomposition}
%\circ:\mathbf{Th}(V,W)\wedge \mathbf{Th}(U,V)\to \mathbf{Th}(U,W).
%\end{equation}
%This map has an identity, given by
%\begin{equation}\label{eq:thomspaceidentity}
%(\mathrm{id},0)\in \mathbf{Th}(U,U).
%\end{equation}
%\begin{mydef}
%The $\mathbf{Top}_{\ast G}$-category $\mathbf{Th}_G$ has as objects
%finite dimensional $G$-representations. For finite dimensional
%$G$-representations $U$ and $V$ the morphism space
%is the pointed $G$-space $\mathbf{Th}(U,V)$.  The composition is
%given by \eqref{eq:thomspacecomposition} and the identity by \eqref{eq:thomspaceidentity}.
%The $\mathbf{Top}_{\ast G}$-category $\mathbf{Th}_G^{\mathrm{triv}}$ is the subcategory
%with objects the trivial finite dimensional $G$-representations and the same
%morphism spaces.
%\end{mydef}

% We now proceed to explain that any $G$-spectrum has an 
%underlying naive $G$-spectrum. Let
%\[
%\iota: \mathbf{Th}^{\mathrm{triv}}_G\to \mathbf{Th}_G
%\]
%denote the inclusion functor. If $T$ is an orthogonal $G$-spectrum,
%$\iota^\ast T$ naturally takes values in $\mathbf{Top}^G_{\ast}$,
%since if $U$ and $V$ are trivial $G$-representations, the $G$-action
%on $\mathbf{Th}(U,V)$ is trivial, hence the $G$-equivariant map
%\[
%\mathbf{Th}(U,V)\to \map(T(U),T(V))
%\]
%factors through $\map(T(U),T(V))^G$, which are exactly the equivariant maps.
%\begin{mydef}
%Let $T$ be an orthogonal $G$-spectrum. Its underlying naive spectrum
%is the naive orthogonal $G$-spectrum $\iota^\ast T$.
%\end{mydef}
Naive and non-naive spectra are related by the following result. It will play an important
role in our discussion of change of group functors.
\begin{thm}\label{thm:naiveisnotnaive}
The functor
\[
\iota^\ast: \mathbf{Sp}^G\to \mathbf{NSp}^G
\]
is an equivalence of categories. 
Its inverse
\[
\iota_!: \mathbf{NSp}^G\to \mathbf{Sp}^G
\]
is given by mapping a naive $G$-spectrum $T$
to its enriched left Kan extension along $\iota:\mathbf{Th}^{\mathrm{triv}}_G\to 
\mathbf{Th}_G$.
In particular, $G$-spectra
with isomorphic underlying naive $G$-spectra are isomorphic.
\end{thm}

Every $T$ spectrum comes equipped with a structure map
\begin{equation}\label{eq:structuremap}
\sigma_{U,V}: S^V\wedge T(U)\to T(U\oplus V),
\end{equation}
as well as an internal structure map
\begin{equation}\label{eq:internalstructuremap}
\sigma_{U\subseteq V}: S^V\wedge T(U)\to T(V)
\end{equation}
for finite dimensional $G$-representations $U$ and $V$.
A precise definition is given in \cite[Definition~4.6, \pno~1505]{rvadams}.
Before we proceed, we give two important examples of $G$-spectra.
\begin{bsp}
\begin{enumerate}
\item The $G$\textit{-sphere spectrum} $\spherespectrum$
is defined by $\spherespectrum(U) = S^U$. On morphisms
it is given by sending $(f,v)\in \mathbf{Th}(U,V)$
to the one-point compactification of the map
\[
U\to V, u\mapsto f(u)+v.
\]
\item Let $X$ be a $G$-space. Its \textit{suspension spectrum}
$\Sigma^\infty X$ is defined by $(\Sigma^\infty X)(U) = S^U\wedge X$.
On morphisms it is defined analogously to the sphere spectrum.
\end{enumerate}
\end{bsp}

The previous examples are special cases of a more general construction,
which we will frequently use.
\begin{mydef}
Let $X$ be a pointed $G$-space and $T$ a $G$-spectrum. Their
smash product $T\wedge X$ is the composition
\[
\mathbf{Th}_G\xrightarrow{T}\mathbf{Top}_{\ast G}
\xrightarrow{-\wedge X} \mathbf{Top}_{\ast G}.
\]
The mapping spectrum $\map(X,T)$ is defined as the composition
\[
\mathbf{Th}_G\xrightarrow{T}\mathbf{Top}_{\ast G}
\xrightarrow{\map(X,-)} \mathbf{Top}_{\ast G}.
\]
These operations assemble to $\mathbf{Top}_{\ast G}$-functors
\begin{align*}
-\wedge X: \mathbf{Sp}^G&\to \mathbf{Sp}^G,\\
\map(X,-):\mathbf{Sp}^G &\to \mathbf{Sp}^G,
\end{align*}
which form an adjoint pair.
In the case of a finite dimensional $G$-representation $U$
we write $\Sigma^U T$ and $\Omega^U T$ instead
of $T\wedge S^U$ and $\map(S^U,T)$ respectively. We refer to the corresponding functors
as the suspension and loop functors.
\end{mydef}
\begin{mydef}
A $G$-homotopy is a map of $G$-spectra $f:T\wedge I_+\to T'$. Just as for spaces,
we have the notion of $G$-homotopic maps of spectra and $G$-homotopy
equivalences.
\end{mydef}




%\begin{prop}
%The categories of naive orthogonal $G$-spectra and of orthogonal spectra with $G$-action are isomorphic.
%\end{prop}
%
%\begin{proof}
%Let $T$ be a naive orthogonal $G$-spectrum. We put $X_n=T(\RR^n)$. To define the $O(n)\times G$-action, note that $f\in O(n)$ implies
%$(f,0)\in \mathcal{Th}(\RR^n, \RR^n)$ and that $$T(f,0): X_n\rightarrow X_n$$ is $G$-equivariant. This means we can define the action by
%$$O(n)\times G\times X_n\rightarrow X_n, (f,g,x)\rightarrow T(f,0)(gx).$$
%
%Next, we define the structure map
%$$\sigma_n: X_n\wedge S^1\rightarrow X_{n+1}$$
%as the adjoint of the map
%$$S^1\rightarrow \mathcal{Th}(\RR^n,\RR^{n+1})\rightarrow \text{Map}_{\mathcal{T}^G}(T(\RR^n),T(\RR^{n+1}))=\text{Map}_{\mathcal{T}^G}(X_n,X_{n+1}).$$
%Then $\sigma_n$ is $G$-equivariant and the iterated structure maps are as well. 
%
%Conversely, if $X$ is an orthogonal spectrum with a $G$-action, we will a naive orthogonal spectrum $T$. Note that on objects it suffices to define
%$T(R^n)$ for all $n\in N$. For if $V$ is an inner product space of dimension $n$ and $f:V\rightarrow \RR^n$ an isometric isomorphism, it
%is easy to see that $(f,0)\in \mathcal{Th}(V,\RR^n)$ is an isomorphism.
%\end{proof}

\subsection{Homotopy theory of equivariant orthogonal spectra.}
The definition of the homotopy groups of a spectrum requires the notion of a 
(complete) $G$-universe \cite[Definition~II.2.2, \pno~30]{mandellmay}.
In the setup of \cite{mandellmay} the universe is built into the definition
of a $G$-spectrum, but we find it more convenient to define $G$-spectra
without universes. The downside is that we need a choice of universe
to define homotopy groups. For complete $G$-universes, this choice is
irrelevant, see for example the discussion in \cite[\pno~1510-1511]{rvadams}.
\begin{mydef}
Let $T$ be a $G$-spectrum and $H$ a closed subgroup of $G$. Choose
a complete $G$-universe $\mathcal{U}$ and denote by $\mathcal{P}(\mathcal{U})$
the poset of all finite dimensional subrepresentations of $\mathcal{U}$. Define
\begin{equation}\label{eq:equivarianthomotopygroup}
\pi_q^H(T) = 
\begin{cases}
\underset{U\in \mathcal{P}(\mathcal{U})}{\colim} \pi_q\left(\left(\Omega^UT(U)\right)^H\right) &\text{ if } q\ge 0,\\
\underset{U\in \mathcal{P}(\mathcal{U})}{\colim} \pi_q\left(\left(\Omega^UT(U\oplus \RR^{-q})\right)^H\right) &\text{ if } q< 0.
\end{cases}
\end{equation}
The structure maps for the colimits above are obtained by first applying $H$-fixed points and then $\pi_q(-)$
to the maps
\begin{align}
\Omega^UT(U)&\xrightarrow{\Omega^U \tilde \sigma_{U\subseteq V}} \Omega^U\Omega^{V-U} T(V)\cong \Omega^{V} T(V)
&\text{ if } q\ge 0,\label{eq:homotopygroupstructuremap1}\\
\Omega^UT(U\oplus \RR^{-q})&\xrightarrow{\Omega^U \tilde \sigma_{U\oplus \RR^{-q}\subseteq V\oplus \RR^{-q}}} 
\Omega^U\Omega^{V-U} T(V\oplus \RR^{-q})\cong \Omega^{V} T(V\oplus \RR^{-q})
&\text{ if } q<0 \label{eq:homotopygroupstructuremap2},
\end{align}
after noting that $V\oplus \RR^{-q}-U\oplus \RR^{-q} = V-U$.
\end{mydef}
\begin{mydef}
Let $f:T\to T'$ be a map of $G$-spectra.
\begin{enumerate}[(i)]
\item We say $f$ is a weak equivalence of $G$-spectra
if the map
\[
f_\ast:\pi_q^H(T)\to \pi_q^H(T')
\]
is an isomorphism for all $q\in \ZZ$ and all closed subgroups $H$
of $G$.
\item We say $f$ is a level equivalence if $f:T(U)\to T'(U)$
is a weak equivalence of $G$-spaces for any finite dimensional
$G$-representation $U$.
\end{enumerate}
\end{mydef}
For brevity we will often drop the adjective weak when
talking about weak equivalences of $G$-spectra.
We will often denote an equivalence of $G$-spectra by
$T\xrightarrow{\sim} T'$ or by $T\simeq T'$. In equivariant
stable homotopy theory one is mostly interested in the weak
equivalences. The usefulness of the level equivalences comes
from the following basic result from \cite[Lemma~III.3.3, \pno~45]{mandellmay}.
\begin{lem}
A level equivalence of $G$-spectra is a weak equivalence of $G$-spectra.
\end{lem}

The category $\mathbf{Sp}^G$ can be given the structure of a model
category \cite[Theorem~III.4.2, \pno~47]{mandellmay} such that the weak equivalences
of $G$-spectra are exactly the weak equivalences of the model structure. 
It is referred to as the \textit{stable model structure}
and admits Quillen's
small object argument. By \cite[Theorem~2.1.14, \pno~33]{hoveymodelcats} 
this implies the existence of a fibrant replacement 
functor $Q$ and a cofibrant replacement functor $\Gamma$
together with weak equivalences of $G$-spectra
\[
r_T: T\xrightarrow{\sim} QT
\]
and
\[
\gamma_T:\Gamma T\xrightarrow{\sim} T,
\]
which are both natural in $T$. The next result justifies the expression
stable model structure.


\begin{lem}\label{lem:spectrumsuspensionshift}
\begin{enumerate}[(i)]
\item For any finite dimensional $G$-representation $U$
the unit $\eta:T\to \Omega^U\Sigma^UT$ is an equivalence
of $G$-spectra.
\item There is a natural isomorphism $\pi^H_q(\Sigma T)\cong \pi_{q+1}(T)$
\end{enumerate}
\end{lem}
\begin{proof}
The first assertion is \cite[Lemma~3.8, \pno~46]{mandellmay}. For (ii)
we show there is a natural isomorphism $\pi_q^H(\Omega^1 T)\cong \pi_{q+1}^H(T)$
and apply (i). The natural isomorphisms 
\[
(\Omega^U\Omega^1T(U))^H\cong\Omega^1(\Omega^UT(U))^H
\]
and 
\[
(\Omega^U\Omega^1T(U\oplus \RR^{-q}))^H\cong\Omega^1(\Omega^UT(U\oplus\RR^{-q}))^H
\]
commute with \eqref{eq:homotopygroupstructuremap1} and  \eqref{eq:homotopygroupstructuremap2} 
and by adjointness we have the isomorphisms
\[
\pi_q(\Omega^1(\Omega^UT(U))^H)\cong \pi_{q+1}((\Omega^UT(U))^H)
\]
and
\[
\pi_q(\Omega^1(\Omega^UT(U\oplus\RR^{-q}))^H)\cong \pi_{q+1}((\Omega^UT(U\oplus\RR^{-q}))^H).
\]
The claim now follows by passing to the colimits.
\end{proof}

The fibrant objects in the stable model structure can be easily
characterized and possess good homotopical properties.
\begin{mydef}
Let $T$ be a $G$-spectrum.
\begin{enumerate}[(i)]
\item We call $T$ good if the structure map \eqref{eq:structuremap}
is a closed embedding.
\item $T$ is a $G$-$\Omega$-spectrum if the adjoint
\begin{equation}
\tilde \sigma_{U\subseteq V}: T(U)\to \Omega^{V-U}T(V)\label{eq:adjointinternalstructuremap}
\end{equation}
of the internal structure map \eqref{eq:internalstructuremap} is
a weak equivalence of $G$-spaces.
\end{enumerate}
\end{mydef}

\begin{lem}
Let $T$ be a $G$-spectrum.
\begin{enumerate}[(i)]
\item $T$ is a $G$-$\Omega$-spectrum iff it is fibrant
in the stable model structure on $\mathbf{Sp}^G$.
\item If $T$ is a $G$-$\Omega$-spectrum, there is a natural isomorphism
\begin{equation}\label{eq:homotopygroupomegaspectrum}
\pi_q^H(T)\cong
\begin{cases}
\pi_q\left(T(0)^H\right) &\text{ if } q\ge 0,\\
\pi_q\left(T(\RR^{-q})^H\right) &\text{ if } q<0.
\end{cases}
\end{equation}
\end{enumerate}
\end{lem}
\begin{proof}
For (i) see \cite[Proposition~III.4.12, \pno~50]{mandellmay}.
The proof of \cite[Lemma~III.3.3, \pno~45]{mandellmay} implies that
\eqref{eq:homotopygroupstructuremap1} and \eqref{eq:homotopygroupstructuremap2}
are weak equivalences of $G$-spaces if $T$ is a $G$-$\Omega$-spectrum,
hence the structure maps for the colimit in \eqref{eq:equivarianthomotopygroup}
are all isomorphisms. This yields (ii).
\end{proof}
In the remainder of this subsection we explain how to obtain a reduced homology
theory from a $G$-spectrum. For the axioms of a reduced homology
theory we refer to \cite[\pno~110]{mayconcise}. We fix a $G$-spectrum $T$
and a subgroup $H$. Given a pointed $G$-CW-complex $X$
we put
\begin{equation}\label{eq:spectrumhomology}
\tilde E_q(X) = \pi_q^H(T\wedge X).
\end{equation}
To apply the above definition to non-equivariant CW-complexes, we equip
these with the trivial action.
We have the following result from \cite[Theorem~III.3.5]{mandellmay}.
\begin{lem}\label{lem:spectrumwedge}
Let $\{T_i:i\in I\}$ be a family of $G$-spectra. The inclusion
$\iota_j:T_j\to \bigvee_{i\in I}$ of each summand into the wedge
induces for each subgroup $H$ and each integer $q\in \ZZ$ an isomorphism 
\begin{equation}
\bigoplus_{i\in I} \pi_q^H(T_i)\xrightarrow{\cong} \pi_q^H\left(\bigvee_{i\in I} T_i\right).
\end{equation}
\end{lem}

Lemma \ref{lem:spectrumwedge} and \ref{lem:spectrumsuspensionshift} together with the isomorphism
\[
T\wedge \bigvee_{i\in I} X_i\cong \bigvee_{i\in I} T\wedge X_i
\] imply that \eqref{eq:spectrumhomology} satisfies the additivity and suspension axioms. The weak equivalence axiom
is a consequence of the fact that if $X\to Y$ is a weak equivalence of $G$-spaces,
$T\wedge X\to T\wedge Y$ is a level equivalence. To see that the exactness
axiom also holds we take brief detour into cofiber sequences of $G$-spectra.

\begin{mydef}
Let $f:T\to T'$ a map of $G$-spectra.
\begin{enumerate}[(i)]
\item The cofiber of $f$ is the $G$-spectrum $Cf$ such that 
$(Cf)(U)$ is the cofiber of $f:T(U)\to T'(U)$ for
any finite dimensional $G$-representation $U$.
It comes equipped with a natural inclusion 
$i:T'\to Cf$.
\item We say $T\to T'\to T''$ is a cofiber sequence
if there is a map of $G$-spectra $f:X\to Y$
and a diagram
\[
\begin{tikzcd}
T\arrow{r}\arrow{d}[anchor = center, rotate = 90, yshift = -1ex]{\sim}
&T' \arrow{d}[anchor = center, rotate = 90, yshift = -1ex]{\sim}\arrow{r}
&T''\arrow{d}[anchor = center, rotate = 90, yshift = -1ex]{\sim}\\
X\arrow{r}{f}
&Y\arrow{r}{i}
&Cf,
\end{tikzcd}
\]
which commutes up to $G$-homotopy, such that the vertical
arrows are equivalences of $G$-spectra.
\item The homotopy fiber of $f$ is the $G$-spectrum
$Ff$ such that $(Ff)(U)$ is the homotopy fiber of the map
$f:T(U)\to T'(U)$ for any finite dimensional $G$-representation $U$.
It has a natural projection $p:Ff\to X$.
\item We say $T\to T'\to T''$ is a fiber sequence
if there is a map of $G$-spectra $f:X\to Y$
and a diagram
\[
\begin{tikzcd}
T\arrow{r}\arrow{d}[anchor = center, rotate = 90, yshift = -1ex]{\sim}
&T' \arrow{d}[anchor = center, rotate = 90, yshift = -1ex]{\sim}\arrow{r}
&T''\arrow{d}[anchor = center, rotate = 90, yshift = -1ex]{\sim}\\
Ff\arrow{r}{p}
&X\arrow{r}{f}
&Y,
\end{tikzcd}
\]
which commutes up to $G$-homotopy, such that the vertical
arrows are equivalences of $G$-spectra.
\end{enumerate}
\end{mydef}
Note that if $f:X\to Y$ is a map of $G$-spaces and $T$ a $G$-spectrum,
then we have the isomorphisms 
\[
T\wedge Cf\cong C(\mathrm{id}\wedge f) \text{ and } T\wedge Ff\cong F(\mathrm{id}\wedge f),
\]
which imply that if $X\to Y\to Z$ is a (co)fiber sequence of $G$-spaces,
$T\wedge X\to T\wedge Y\to T\wedge Z$ is a (co)fiber sequence of $G$-spectra.
The following result is \cite[Theorem~III.3.5, \pno~45]{mandellmay}.
Of course, by functoriality the analogous result also holds for arbitrary (co)fiber
sequences.

\begin{thm}\label{thm:(co)fibersequences}
Let $f:T\to T'$ be a map of $G$-spectra. For any subgroup $H$
there are natural long exact sequences
\[
\ldots\to\pi_q^H(T)\xrightarrow{f_\ast} \pi_q^H(T')
\xrightarrow{i_\ast} \pi_q^H(Cf)\xrightarrow{\partial}
\pi_{q-1}^H(T)\to \ldots
\]
and
\[
\ldots \to \pi_q^H(Ff)\xrightarrow{p_\ast} \pi_q^H(T)\xrightarrow{f_\ast} \pi_q^H(T')
\xrightarrow{\partial} \pi_{q-1}^H(Ff)\to \ldots
\]
of abelian groups.
\end{thm}
This immediately implies that \eqref{eq:spectrumhomology} satisfies the exactness axiom,
that is it defines a reduced homology theory. Standard results on homology theories
(for example \cite[Theorem~10.8.1 and Proposition~10.8.4, \pno~271-272]{diecktop})
then yield the following proposition.

\begin{prop}\label{prop:colimitshomotopygroups}
Let $X_0\to X_1\to\ldots \to X_n\to \ldots$ be a sequence of cofibrations of pointed $G$-spaces
and let $X = \colim_n X_n$. Then for each $G$-spectrum $T$, each integer $q\in \ZZ$
and each subgroup $H$ of $G$ the structure maps $X_n\to X$ induce an isomorphism
\begin{equation}
\underset{n}{\colim} \,\pi_q^H(T\wedge X_n)\xrightarrow{\cong} \pi_q^H(T\wedge X).
\end{equation}

\end{prop}


\subsection{Change of groups.}
Let $\alpha:\Gamma\to G$ be a group homomorphism.
Analagously to the case of spaces, there are change of group
functors
\begin{align}\label{eq:spectrumrestrictioninduction}
\begin{split}
\alpha^\ast:\mathbf{Sp}^G  &\to \mathbf{Sp}^\Gamma,\\
\alpha_\ast:\mathbf{Sp}^\Gamma  &\to \mathbf{Sp}^G.
\end{split}
\end{align}
We refer to \cite[Construction~6.5 and Construction~6.7, \pno~1519]{rvadams}
for details.
%We do this first for naive spectra.
%\begin{mydef}
%If $T$ is a naive $G$-spectrum we define its \textit{restriction along}
%$\alpha$ as the composite
%\[
%\mathbf{Th}^{\mathrm{triv}}_\Gamma = \mathbf{Th}^{\mathrm{triv}}_G\xrightarrow{T}
%\mathbf{Top}_{\ast G}\xrightarrow{\alpha^\ast}\mathbf{Top}_{\ast \Gamma}.
%\]
%If $T$ is a naive $\Gamma$-spectrum its \textit{induction along} $\alpha$ is the composite
%\[
%\mathbf{Th}^{\mathrm{triv}}_G = \mathbf{Th}^{\mathrm{triv}}_\Gamma\xrightarrow{T}
%\mathbf{Top}_{\ast \Gamma}\xrightarrow{\alpha_\ast}\mathbf{Top}_{\ast G}.
%\]
%By abuse of notation, we denote the resulting functors as
%\begin{align}
%\alpha^\ast:\mathbf{NSp}^G  &\to \mathbf{NSp}^\Gamma,\\
%\alpha_\ast:\mathbf{NSp}^\Gamma  &\to \mathbf{NSp}^G.
%\end{align}
%\end{mydef}
As in the case of spaces, if $H$ is a subgroup of $G$ and $\alpha:H\to G$ denotes the inclusion,
we often write $\mathrm{res}^G_H$ and $\mathrm{ind}^G_H$ instead of
$\alpha^\ast$ and $\alpha_\ast$.
%Using theorem \ref{thm:naiveisnotnaive} we can extend the previous definition to non-naive
%spectra.
%\begin{mydef}
%For a $G$-spectrum $T$ we define its \textit{restriction along} $\alpha$ as
%the $\Gamma$-spectrum $\iota_!\alpha^\ast\iota^\ast T$.
%Given a $\Gamma$-spectrum $T$ we define its \textit{induction along} $\alpha$ as
%the $\Gamma$-spectrum $\iota_!\alpha_\ast\iota^\ast T$.
%We abuse notation and denote the resulting functors by \eqref{eq:spectrumrestrictioninduction}.
%\end{mydef}
%Again, if $H$ is a subgroup of $G$ and $\alpha:H\to G$ denotes the inclusion,
%we often write $\mathrm{res}^G_H$ and $\mathrm{ind}^G_H$ instead of
%$\alpha^\ast$ and $\alpha_\ast$. 
The next lemma records basic
homotopical properties of the restriction functors.

\begin{lem}
Let $T$ be a $G$-spectrum and $H$ a subgroup of $G$.
\begin{enumerate}[(i)]
\item $\pi_q^H(T)\cong \pi_q^H(\mathrm{res}^G_H T)$.
\item If $f:T\to T'$ is an equivalence of $G$-spectra, then $\mathrm{res}^G_H f:
\mathrm{res}^G_HT\to \mathrm{res}^G_H T'$ is an equivalence of
$H$-spectra.
\item If $\alpha:\Gamma\to G$ is an isomorphism, then for any $G$-spectrum $T$
and any integer $q\in \ZZ$
we have a natural isomorphism $\pi_q^H(T)\cong \pi_q^{\alpha^{-1}(H)}(\alpha^\ast T)$.
Consequently a map $f:T\to T'$ is an equivalence of $G$-spectra iff
$\alpha^\ast f:\alpha^\ast T\to \alpha^\ast T'$ is an equivalence of $\Gamma$-spectra.
\end{enumerate}
\end{lem}


Next, we turn to the notion of fixed-point spectra. These will play a major role in this thesis. 
First, we introduce some notation. We denote by
$j:\mathbf{Th}_{\{1\}}\to \mathbf{Th}_G$ the inclusion.
The same reasoning as for 
$\iota^\ast T$ shows that $j^\ast T$ takes values in $\mathbf{Top}_\ast^G $.
Now let $H$ be a subgroup of $G$.
Then $\mathbf{Th}_G^{H\mathrm{-triv}}$ denotes the full
subcategory of $\mathbf{Th}_G$ which has as objects
$G$-representations $U$ such that $H$ acts trivially on $U$.
Moreover, we denote by $\mathbf{Top}_{\ast G}^H$ the subcategory
of $\mathbf{Top}_{\ast G}$ with the same objects, but with morphisms
$H$-equivariant maps. If we additionally assume that $H$ is normal,
then the projection $p:G\to G/H$ induces a functor
$p^\ast:\mathbf{Th}_{G/H}\to\mathbf{Th}_G^{H\mathrm{-triv}}$
and one checks again that the composition
\[
\mathbf{Th}_{G/H}\xrightarrow{p^\ast}\mathbf{Th}_G^{H\mathrm{-triv}}
\xrightarrow{T} \mathbf{Top}_{\ast G}
\]
takes values in $\mathbf{Top}_{\ast G}^H$.
\begin{mydef}
Let $T$ be a $G$-spectrum.
\begin{enumerate}[(i)]
\item The non-equivariant $H$-fixed point spectrum of $T$, denoted
by $\mathrm{res}_1 T^H$, is given by the composition
\[
\mathbf{Th}_{\{1\}}\xrightarrow{j}\mathrm{Th}_G \xrightarrow{T}
\mathbf{Top}_\ast^G \xrightarrow{(-)^H} \mathbf{Top}_\ast.
\]
\item Suppose additionally that $H$ is normal. We define the $H$-fixed
point spectrum of $T$, denoted by $T^H$, as the $G/H$ spectrum
given by the composition
\[
\mathbf{Th}_{G/H}\xrightarrow{p^\ast}\mathbf{Th}_G^{H\mathrm{-triv}}
\xrightarrow{T} \mathbf{Top}_{\ast G}^H\xrightarrow{(-)^H} \mathbf{Top}_{\ast G/H}.
\]
\end{enumerate}
\end{mydef}

It is evident from the definition that taking fixed point spectra is functorial.
For $G$-$\Omega$-spectra the fixed point functors have good homotopical
properties.
\begin{lem}\label{lem:omegaspectrafixedpoints}
Let $T$ be a $G$-$\Omega$-spectrum.
\begin{enumerate}[(i)]
\item For any integer $q\in \ZZ$ and any subgroup $H$ there is a natural isomorphism $\pi_q^{\{1\}}(\mathrm{res}_1 T^H)\cong \pi_q^H(T)$.
\item  If $H$ is a normal subgroup $T^H$ is a $G/H$-$\Omega$-spectrum. Additionally,
if $K$ is a subgroup containing $H$, there is a natural isomorphism
$\pi_q^{K/H}(T^H)\cong \pi_q^K(T)$ for any integer $q\in \ZZ$.
\item If $T'$ is a $G$-$\Omega$-spectrum, then $f:T\to T'$ is an equivalence of $G$-spectra
iff $f^H: \mathrm{res}_1 T^H\to \mathrm{res}_1 T'^H$ is an equivalence of non-equivariant spectra
for all subgroups $H$. Moreover, if $f$ is an equivalence of $G$-spectra and $H$ is normal, the induced
map $f^H:T^H\to T'^H$ is an equivalence of $G/H$-spectra.
\end{enumerate}
\end{lem}

\begin{proof}
That $T^H$ is a $G/H$-$\Omega$-spectrum if $H$ is normal 
is proven in \cite[Lemma~6.20, \pno~24]{rvadams}. The
proof given there also works for compact Lie groups. The rest follows
from \eqref{eq:homotopygroupomegaspectrum} and the isomorphism 
$
(X^H)^{K/H}\cong X^K
$
for any $G$-space $X$. 
\end{proof}

We end this subsection with two technical lemma's on fixed point spectra, which we will need later on.

\begin{lem}\label{lem:wedgefixed}
Let $\{T_i\}_{i\in I}$ be a family of  orthogonal $G$-spectra and $H$ a closed normal subgroup. Then the inclusion
$$T_j\rightarrow \bigvee_{i\in I } T_i$$ induces a map
$$QT_i\rightarrow Q\left(\bigvee_{j \in I } T_j\right).$$ After taking fixed points
this yields a map 
$$\bigvee_{i\in I}(QT_i)^H\rightarrow \left(Q\bigvee_{i \in I } T_i\right)^H.$$
This map is an equivalence of $G/H$-spectra.
\end{lem}

\begin{proof}
Let $K$ be a closed subgroup containing $H$. The claim follows from the
following commutative diagram
$$
\begin{tikzcd}
\pi_n^{K/H}(\bigvee_{i\in I}(QT_i)^H) \arrow{r} 
&\pi_n^{K/H}((Q(\bigvee_{i \in I } T_i))^H) \arrow{d}[swap]{\cong} \\
\bigoplus_{i\in I} \pi_n^{K/H}((QT_i)^H) 
\arrow{u}{\cong}[swap]{\oplus (\iota_{QT_i})_\ast} 
\arrow{d}[swap]{\cong}
&\pi_n^K(Q(\bigvee_{i \in I } T_i))\\
\bigoplus_{i\in I} \pi_n^{K}(QT_i)
& \pi_n^K(\bigvee_{i \in I } T_i) \arrow{u}{\cong}[swap]{(r_{\bigvee_{i \in I } T_i})_\ast}\\
\bigoplus_{i\in I} \pi_n^{K}(T_i) \arrow{u}{\cong}[swap]{\oplus (r_{T_i})_\ast}  
\arrow{ru}{\cong}[swap]{\oplus (\iota_{T_i})_\ast} 
%& \bigoplus_{i \in I} \pi_n^K(T_i) \arrow{u}{\cong}
\end{tikzcd}
$$
where the horizontal map is the map from the statement, $\iota_{QT_i}:QT_i
\rightarrow \bigvee_{j\in I} QT_i$ is the inclusion and the undecorated vertical arrows 
are the isomorphisms from lemma \ref{lem:omegaspectrafixedpoints} (ii).
\end{proof}

\begin{lem}\label{lem:smashfixed}
%\todo{I'm not sure this approach for the proof is correct}
Let $T$ be an orthogonal $G$-$\Omega$-spectrum and let $A$ be a CW-complex with trivial $G$-action.
Then for any closed normal subgroup $H$ the map 
$ T\wedge A\to Q(T\wedge A)
$
induces an equivalence of $G/H$-spectra $T^H\wedge A\simeq Q(T\wedge A)^H$.
\end{lem}

\begin{proof}
For $A=D^n$ the claim is obvious. We now prove the claim  for $A=S^n$ by
induction. We have $S^0\wedge T\cong T$, so the claim follows 
from the fact that $T$ is a $G$-$\Omega$-spectrum. For the inductive step, consider the cofiber sequence
$$S^{n-1}\rightarrow D^n\rightarrow S^n.$$
Smashing with $T$ yields a cofiber sequence of $G$-spectra
$$T\wedge S^{n-1}\rightarrow T\wedge D^n\rightarrow T\wedge S^n$$
and smashing with $T^H$ yields a cofiber sequence of $G/H$-spectra
\[
T^H\wedge S^{n-1}\rightarrow T^H\wedge D^n\rightarrow T^H\wedge S^n.
\]
By theorem \ref{thm:(co)fibersequences} the top row in the following diagram
becomes a long exact sequence after taking homotopy groups:
$$
\begin{tikzcd}
\arrow{r} \arrow{d}[anchor = center, rotate = 90, yshift = -1ex]{\sim}
T^H\wedge S^{n-1}
&T^H\wedge D^{n}
\arrow{r} \arrow{d}[anchor = center, rotate = 90, yshift = -1ex]{\sim}
&T^H\wedge S^{n}
\arrow{d}
\\
(Q(T\wedge S^{n-1}))^H\arrow{r}
&(Q(T\wedge D^n))^H\arrow{r}
&(Q(T\wedge S^{n}))^H.
\end{tikzcd}
$$
Fibrant replacement preserves cofiber sequences, so by lemma \ref{lem:omegaspectrafixedpoints}
and theorem \ref{thm:(co)fibersequences}
the bottom row also becomes a long exact sequence after applying homotopy groups.
By induction hypothesis the first two
vertical arrows are equivalences of $G/H$-spectra, so the third one is as well.

Now let $A$ be a general CW-complex with basepoint $a_0\in A_0$.
Then the claim follows for $A_0$ from the isomorphism $T\wedge A_0\cong \bigvee_{a\in A_0\setminus\{a_0\}} T$ and lemma \ref{lem:wedgefixed}. Using the cofiber sequence
\[A_{n-1}\to A_n\to A_n/A_{n-1}\]
and lemma \ref{lem:wedgefixed}, a similar argument as above shows the claim for every
$n$-skeleton $A_n$. For $A$ the claim follows from the following diagram
\[
\begin{tikzcd}
\underset{n}{\colim}\,\pi_q^{K/H}(T^H \wedge A_n)
\arrow{d}{\cong}\arrow{r}{\cong}
& \underset{n}{\colim}\,\pi_q^{K/H}((Q(T \wedge A_n))^H)
\arrow{d}
&\underset{n}{\colim}\,\pi_q^{K}(T \wedge A_n)
\arrow[swap]{l}{\cong}\arrow{d}{\cong}\\
\pi_q^{K/H}(T^H \wedge A)
\arrow{r}
&\pi_q^{K/H}((Q(T \wedge A))^H)
& \pi_q^{K}(T \wedge A)
\arrow[swap]{l}{\cong},
\end{tikzcd}
\]
where the outer vertical arrows are isomorphisms by proposition \ref{prop:colimitshomotopygroups},
the two horizontal arrows on the right are isomorphisms induced by fibrant replacement and lemma
\ref{lem:omegaspectrafixedpoints} and the upper left horizontal arrow is an isomorphism
by the previously treated cases. 
\end{proof}
\subsection{The smash product of equivariant orthogonal spectra.}
In this subsection we briefly describe the smash product of orthogonal
$G$-spectra. It gives $\mathbf{Sp}^G$ the structure of a symmetric monoidal
category. We are particularly interested in the monoid objects, which
are called \textit{ring spectra}. Our main example of a ring spectrum
is the topological Hochschild spectrum (see proposition \todo{fill in this reference} for a 
precise statement). The homotopy groups of ring
spectra have the structure of a graded ring, a fact we will repeatedly use.

In the following definition we use that for pointed $G$-spaces
$W,X,Y$ and $Z$ the canonical map
\begin{equation}\label{eq:mappingspacesmash}
\mathrm{smash}:\map(X,Y)\wedge\map(W,Z)\to \map(X\wedge W, Y\wedge Z)
\end{equation}
is continuous, since it is the adjoint of the map
\[
\map(X,Y)\wedge\map(W,Z)\wedge X\wedge W\cong
\map(X,Y)\wedge X\wedge \map(W,Z)\wedge W
\xrightarrow{\mathrm{ev}_X\wedge \mathrm{ev}_W}
Y\wedge Z.
\]
 \begin{mydef}
 The \textit{external smash product} of orthogonal $G$-spectra $T$ and $T'$ is
 the $\mathbf{Top}_{\ast G}$-enriched functor given on objects by
 \[
T\;\bar{\wedge} \;T' : \mathbf{Th}_G\times  \mathbf{Th}_G\to \mathbf{Top}_{\ast G}, 
(U,V)\mapsto T(U)\wedge T'(V).
 \]
 and on morphisms by
\[
\begin{tikzcd}
\mathbf{Th}(U,V) \wedge \mathbf{Th}(U',V')
\arrow{r}{T\wedge T'}
&\map(T(U),T(V))\wedge \map(T'(U'), T'(V'))
\arrow{d}{\mathrm{smash}}\\
&\map(T(U)\wedge T'(U'), T(V)\wedge T'(V')).
\end{tikzcd}
\]
 \end{mydef}

\begin{mydef}
The smash product of two orthogonal $G$ spectra $T$ and $T'$ is
the $\mathbf{Top}_{\ast G}$-enriched left Kan extension
of $T\;\bar{\wedge}\; T'$ along $\oplus$. It is denoted by $T\wedge T'$.
\end{mydef}

\begin{prop}
The smash product gives $\mathbf{Sp}^G$ the structure of a symmetric monoidal
category. The unit object is the sphere spectrum $\spherespectrum$.
\end{prop}

\begin{proof}
We only prove the isomorphism $\spherespectrum \wedge T\cong T$. The rest
follows formally from the calculus of Kan extensions. As part of the data of a
left Kan extension we have a $\mathbf{Top}_{\ast G}$-enriched natural transformation
\[
\epsilon: \spherespectrum\;\bar{\wedge}\;T\to (\spherespectrum\wedge T)\circ \oplus.
\]
For any finite dimensional $G$-representation $U$ we then obtain the $G$-equivariant map
\begin{equation}\label{eq:smashproductunit1}
T(U)\cong S^0\wedge T(U)\xrightarrow{\epsilon_{0,U}} (\spherespectrum\wedge T)(U).
\end{equation}
Conversely, the structure maps \eqref{eq:structuremap} of $T$ give a map
\[
S^V\wedge T(U)\xrightarrow{\sigma_{U, V}} T(U\oplus V),
\]
which by the universal property of the Kan extension yields a map
\begin{equation}\label{eq:smashproductunit2}
u_T:\spherespectrum \wedge T\to T.
\end{equation}
One checks that \eqref{eq:smashproductunit1} is a map of $G$-spectra and using
the universal property of Kan extensions one sees that it is inverse to
\eqref{eq:smashproductunit2}. 
\end{proof}


\begin{mydef}
A \textit{(commutative) ring} $G$-spectrum is a (commutative) monoid object in
the symmetric monoidal category $(\mathbf{Sp}^G, \wedge, \spherespectrum)$.
\end{mydef}
In general, it is difficult to check if a $G$-spectrum is a ring spectrum using the above
definition. We indicate a more direct way. For the associativity, we note that
by the calculus of Kan extensions the triple smash product $T\wedge T'\wedge T''$
can also be computed as the left Kan extension of the $\mathbf{Top}_{\ast G}$-functor
\[
T\;\bar{\wedge}\; T'\;\bar{\wedge}\; T'':\mathbf{Th}_G\times \mathbf{Th}_G\times \mathbf{Th}_G\to \mathbf{Top}_{\ast G}
\]
along the $\mathbf{Top}_{\ast G}$-functor
\[
-\oplus-\oplus-: \mathbf{Th}_G\times \mathbf{Th}_G\times \mathbf{Th}_G\to \mathbf{Th}_G.
\]
Then, if we have a multiplication map $\mu:T\wedge T\to T$ induced by a $\mathbf{Top}_{\ast G}$-enriched
natural transformation $\bar \mu:T\;\bar{\wedge}\; T\to T\circ \oplus$, the associativity
is equivalent to the commutativity of the following diagram:
\begin{equation}
\begin{tikzcd}[column sep = large]
T(U)\wedge T(V)\wedge T(W)
\arrow{d}{\bar \mu_{U,V}\wedge \mathrm{id}}\arrow{r}{\mathrm{id}\wedge \bar\mu_{V,W}}
&T(U)\wedge T(V\oplus W)
\arrow{d}{\bar\mu_{U,V\oplus W}}\\
T(U\oplus V)\wedge T(W)
\arrow{r}{\bar\mu_{U\oplus V,W}}
&T(U\oplus V\oplus W).
\end{tikzcd}
\end{equation}
Similarly, given a unit map $u: \spherespectrum\to T$ the unit law is equivalent to the condition that the composition
\begin{equation}\label{eq:ringspectrumunit}
T(U)\cong S^0\wedge T(U)\xrightarrow{u_0\wedge\mathrm{id}} T(0)\wedge T(U)\xrightarrow{\bar\mu_{0,U}} T(U)
\end{equation}
is the identity. Finally, the commutativity of the $\mu$ is equivalent to the commutativity of:
\begin{equation}
\begin{tikzcd}
T(U)\wedge T(V)
\arrow{d}{\mathrm{twist}}\arrow{r}{\bar\mu_{U,V}}
&T(U\oplus V)
\arrow{d}{T(\mathrm{twist})}\\
T(V)\wedge T(U)
\arrow{r}{\bar\mu_{V,U}}
&T(V\oplus U).
\end{tikzcd}
\end{equation}
This can again be seen by invoking the formal properties
of Kan extensions and using \eqref{eq:smashproductunit1}. 
We will use the remarks above in the proof that the 
topological Hochschild spectrum is a ring spectrum.

We now proceed to construct a natural pairing
\begin{equation}\label{eq:homotopygroupspairing}
\beta:\pi_m^H(T)\otimes \pi_n^H(T')\to \pi^H_{m+n}(T\wedge T')
\end{equation}
for any subgroup $H$. Given elements $x\in \pi_m^H(T)$
and $y\in \pi_n(T')$, we choose representatives
$f:S^m\to (\Omega^U T(U))^H$ and $g:S^n\to (\Omega^{U'}T'(U'))^H$.
We have the canonical map
\[
\mathrm{smash}: (\Omega^U T(U))^H\wedge (\Omega^{U'} T'(U'))^H\to
(\Omega^{U\oplus U'}(T(U)\wedge T'(U')))^H
\]
and the natural transformation $\epsilon:T\;\bar\wedge\; T'\to (T\wedge T')\circ \oplus$.
We define $\beta(x,y)\in \pi_{m+n}^H(T\wedge T')$ as the class of the map
\[
\begin{tikzcd}[column sep = huge]
S^{m+n}\arrow{r}{\cong}
&S^m\wedge S^n
\arrow{r}{f\wedge g}
&(\Omega^U T(U))^H\wedge (\Omega^{U'} T'(U'))^H
\arrow{d}{\mathrm{smash}}\\
&(\Omega^{U\oplus U'} ((T\wedge T')(U\oplus U')))^H
&\Omega^{U\oplus U'}(T(U)\wedge T'(U'))^H
\arrow[swap]{l}{\Omega^{U\oplus U'}\epsilon_{U, U'}}
\end{tikzcd}
\]
We need to show this is well defined. It is clearly independent of the homotopy
class of $f$ and $g$, so we only need to show it is independent of
the choice of representatives in the colimit in \eqref{eq:equivarianthomotopygroup}.
To see this, let $V$ and $V'$ be $G$-representations containing $U$ and $U'$
respectively and note the following diagram commutes by naturality of $\epsilon$:
\[
\begin{tikzcd}[column sep=13em]
(\Omega^U T(U))^H\wedge (\Omega^{U'} T'(U'))^H
\arrow{d}{\mathrm{smash}}\arrow{r}{(\Omega^{V-U}\tilde \sigma_{U\subseteq V})^H\wedge(\Omega^{V'-U'}\tilde \sigma'_{U'\subset V'})^H}
&(\Omega^V T(V))^H\wedge (\Omega^{V'} T'(V'))^H
\arrow{d}{\mathrm{smash}}\\
(\Omega^{U\oplus U'}(T(U)\wedge T'(U')))^H
\arrow{d}{(\Omega^{U\oplus U'}\epsilon_{U,U'})^H}
&(\Omega^{V\oplus V'}(T(V)\wedge T'(V')))^H
\arrow{d}{(\Omega^{V\oplus V'}\epsilon_{V,V'})^H}\\
(\Omega^{U\oplus U'} ((T\wedge T')(U\oplus U')))^H
\arrow{r}{(\Omega^{V\oplus V'-U\oplus U'}\sigma^{T\wedge T'}_{U\oplus U'\subseteq V\oplus V'})^H}
&(\Omega^{V\oplus V'} ((T\wedge T')(V\oplus V')))^H.
\end{tikzcd}
\]
Using the definition we see that the assignment $\beta(x,y)$ is additive in each
variable, so that \eqref{eq:homotopygroupspairing} really gives a pairing.
Moreover, if $T$ is a ring spectrum, we can compose with the map induced
by the multiplication:
\begin{equation}\label{eq:homotopygroupmultiplication}
\cdot:\pi_m^H(T)\otimes \pi_n^H(T)\xrightarrow{\beta}\pi_{m+n}^H(T\wedge T)
\xrightarrow{\mu_\ast} \pi_{m+n}^H(T).
\end{equation}
\begin{prop}
Let $T$ be a ring $G$-spectrum and $H$ a subgroup.
\begin{enumerate}[(i)]
\item The homotopy groups $\pi_\ast^H(T)$
form a graded ring with multiplication \eqref{eq:homotopygroupmultiplication}.
\item If $T$ is additionally commutative, $\pi_\ast^H(T)$ is graded anti-commutative,
i.e. $x\cdot y = (-1)^{mn}y\cdot x$ for $x\in \pi_m^H(T)$ and $y\in \pi_n^H(T)$.
\end{enumerate}
\end{prop}

\begin{proof}
(i) Since $\mu$ is associative, the multiplication on $\pi_\ast^H(T)$ is as well. Since \eqref{eq:homotopygroupspairing}
is additive, we obtain distributivity. It remains to show the existence of a unit. By the Freudenthal suspension
theorem we have $\pi_0^{\{1\}}(\spherespectrum)\cong \ZZ$ and the class of the map
$S^0\to \Omega^{\RR^1}\spherespectrum(\RR^1)$, which sends the non-basepoint to the
identity is a generator. Since $G$ acts trivially on $\Omega^{\RR^1}\spherespectrum(\RR^1)$,
this map is also an element of $\pi_0^H(\spherespectrum)$, so we have a canonical map
$\pi_0^{\{1\}}(\spherespectrum)\to \pi_0^H(\spherespectrum)$. 
Using \eqref{eq:ringspectrumunit} and the definition of the multiplication
we see that
\[
\pi_q^H(T)\cong \pi_q^H(T)\otimes \pi_0^{\{1\}}(\spherespectrum)
\to  \pi_q^H(T)\otimes \pi_0^H(\spherespectrum)
\xrightarrow{\mathrm{id}\otimes (u_T)_\ast} 
\pi_q^H(T)\otimes \pi_q^H(T) \xrightarrow{\mu_\ast}
\pi_q^H(T)
\]
is the identity, which shows that $\pi_0^{\{1\}}(\spherespectrum)\to \pi_0^H(\spherespectrum)
\xrightarrow{(u_T)_\ast} \pi_0^H(T)$ is a unit map. 

(ii) We consider the map
\[
\tau_{m,n}: S^{m+n}\cong S^m\wedge S^n\xrightarrow{\mathrm{twist}}
S^n\wedge S^m\cong S^{n+m}.
\]
Using the definition of $\beta$ we see the following diagram commutes
(compare \cite[Proposition~4.29, \pno~39]{schwedeequivariant}):
\[
\begin{tikzcd}
\pi_m^H(T)\otimes \pi_n^H(T')
\arrow{dd}{\mathrm{twist}}
\arrow{r}{\beta}
&\pi_{m+n}^H(T\wedge T')
\arrow{d}{(\mathrm{twist})_\ast}\\
&\pi_{m+n}^H(T'\wedge T)
\arrow{d}{\tau_{m,n}^\ast}\\
\pi_n^H(T')\otimes \pi_m^H(T)
\arrow{r}{\beta}
&\pi_{m+n}^H(T'\wedge T)
\end{tikzcd}
\]
Now specialize to the case $T = T'$ and use that
precomposing with $\tau_{m,n}$ induces multiplication
by $(-1)^{mn}$ on homotopy groups.
\end{proof}


\begin{lem}\label{lem:pairing}
Let $X$ be a wedge of $n$-spheres and $T$ an orthogonal $G$-spectrum. 
Then there is a natural isomorphism
\begin{equation}\label{eq:pairingspheres}
\pi_{n}(X)\otimes \pi_q^{\{1\}}(T)\xrightarrow{\cong} \pi_{ q + n}(T\wedge X).
\end{equation}
\end{lem}
\begin{proof}
We define \eqref{eq:pairingspheres} in the same way as \eqref{eq:homotopygroupspairing}.
By lemma \ref{lem:spectrumwedge} we only need to treat the case $X = S^n$
to show it is an isomorphism. However, by checking the definitions we find
that 
\[
\pi_q^{\{1\}}(T)\cong \pi_n(S^n)\otimes \pi_q^{\{1\}}(T)\to \pi_{q+n}^{\{1\}} (T\wedge S^n) =  \pi_{q+n}^{\{1\}} (\Sigma^n T)
\]
is exactly the isomorphism from lemma \ref{lem:spectrumsuspensionshift} (ii).
\end{proof}
\subsection{Restriction and transfer.}



\begin{lem}\label{lem:switchfixedpoints}
Let $C\subset \TT$ be a finite subgroup, $T$ a $C$-spectrum
and $A$ a $C$-CW complex.
%Let $\iota:C\to \TT$ denote the inclusion.
The $C$-action on $ T\wedge A$ induces
a $C$-action on the homotopy groups of $T\wedge A$.
Suppose the $C$-action on $A$ is free away from
the basepoint.
Then the map $\pi_q^C( T\wedge A)\to 
\pi_q^{\{1\}}(T\wedge A)$ induces an isomorphism
\[\pi_q^C( T\wedge A)\xrightarrow{\cong} 
\left(\pi_q^{\{1\}}(  T\wedge A)\right)^C\]
\end{lem}
\todo{A lot of the notation in the proof needs to be cleaned up}
\begin{proof}
We first show that the map of the statement lands in 
$\left(\pi_q^{\{1\}}(  T\wedge A)\right)^C$. This map
is $C$-equivariant, so it suffices to show that $C$
acts trivially on $\pi_q^C( T\wedge A)$.
We fix a cofibrant replacement functor $\Gamma$ for
the stable model structure on the category of $C$-spectra
\cite[Theorem III.4.2, \pno~47]{mandellmay}. 
%Since $G$ acts freely on $X$,
%the projection
%\[\left(\Gamma T\wedge X\right)\wedge EG_+\to \Gamma T\wedge X\]
%is a level equivalence. 
By \cite[Lemma III.4.14, \pno~50]{mandellmay} $\Gamma T\wedge A$
is cofibrant and by \cite[Theorem III.3.11, \pno~47]{mandellmay} the map
$\gamma_T\wedge\mathrm{id_A}: \Gamma T\wedge A
\to T\wedge A$ is an equivalence of $C$-spectra.
Finally, \cite[Lemma~5.10, \pno~1514]{rvadams} implies
that $\Gamma T\wedge A$ is good.
%so by \cite[Theorem~6.16, \pno~1521]{rvadams} $ \Gamma T$
%is good and 
%$\gamma_T:
% \Gamma T\to  T$ is an equivalence of $C$-spectra.
%Applying \cite[Lemma~5.10, \pno~1514]{rvadams} again yields that
%also $ \Gamma T\wedge A$ is good and 
%$\gamma_T\wedge\mathrm{id}:
% \Gamma T\wedge A\to  T\wedge A$ is an equivalence of
%$C$-spectra.

Let $\mathcal{U}$ be a complete $C$-universe.
We use the bifunctorial
replacement functor $Q$ of \cite[Theorem~1.1, \pno~1494]{rvadams}. Our assumptions 
imply that the projection 
\[ \Gamma T\wedge A\wedge EC_+\to  \Gamma T\wedge A\]
 is a level equivalence of $C$-spectra, so by \cite[Main Theorem~1.7, \pno~1496]{rvadams} 
there is a natural equivalence of non-equivariant spectra
\[A:\left(\Gamma T\wedge A\right)\underset{C}{\wedge} EC_+\xrightarrow{\sim} Q^{\mathcal{U}}(\Gamma T\wedge A)^C.\]
%Moreover, by the proof of \cite[Corollary~1.8, \pno~5]{rvadams} the projection
%\[p:\left(\Gamma T\wedge X\right)\underset{G}{\wedge} EG_+\xrightarrow{\sim}
%\left(\Gamma T\wedge X\right)/G\]
%is an equivalence of nonequivariant spectra. 
We obtain a zig-zag of isomorphisms
\[
\begin{tikzcd}
\pi_q^C( T\wedge A)
&\pi_q^C( \Gamma T\wedge A)\arrow{l}[swap]{(\gamma_{ T}\wedge\mathrm{id}_A)_\ast}
\arrow{r}{(r_{ \Gamma T\wedge A})_\ast}
&\pi_q^C\left(  Q^{\mathcal{U}}( \Gamma T\wedge A) \right)
\\
&\pi_q\left(  \left( \Gamma T\wedge A\right)\underset{C}{\wedge} EC_+\right)
\arrow{r}{A_\ast}
&\pi_q\left(   Q^{\mathcal{U}}(\Gamma T\wedge A)^C \right)\arrow{u}{\cong},
\end{tikzcd}
\]
which are natural in both $T$ and $A$.
%The upper left arrow is an isomorphism by \cite[Lemma~5.13, \pno~20]{rvadams},
%because $\gamma_T:T\to \Gamma T$ is an equivalence of $G$-spectra. 
That the
upper right arrow and the vertical arrow
are isomorphisms follows from \cite[Theorem~1.1, \pno~1494]{rvadams}.
Because of the naturality it suffices to show the $C$-action on 
$\pi_q\left(  \left( \Gamma T\wedge A\right)\underset{C}{\wedge} EC_+\right)$
is trivial. For any $g\in C$, the action of $g$ is induced by the map
\begin{equation}\label{eq:universalspaceaction}
\mathrm{id}_{\Gamma T\wedge A}\wedge \ell_g: 
\left( \Gamma T\wedge A\right)\underset{C}{\wedge} EC_+ 
\to  \left( \Gamma T\wedge A\right)\underset{C}{\wedge} EC_+,
\end{equation}
where
\[\ell_g:EC\to EC\]
denotes left multiplication by $g$. 
We show that \eqref{eq:universalspaceaction} is (non-equivariantly) homotopic to the identity
for a suitable model of $EC$. Let $\mathcal{F}$ be the family
of all subgroups $G$ of $\TT$ such that $G\cap C = 1$
and let $E\mathcal{F}$ be the universal $\TT$-space
for $\mathcal{F}$. If $\iota:C\to \TT$ denotes the inclusion,
then $\iota^\ast E\mathcal{F} = EC$ and left multiplication
by elements of $\TT$ makes sense.
We choose $\varphi\in I$ such that $\mathrm{exp}(2\pi i \varphi) = g$
and define
\[h:EC_+\wedge I_+\to EC_+, x\wedge t\mapsto \mathrm{exp}(2\pi i (1-t)\varphi)\cdot x.\]
After smashing with $\Gamma T\wedge A$ we obtain a 
well-defined homotopy
\[H: \left(\left( \Gamma T\wedge A\right)\underset{C}{\wedge} EC_+\right) \wedge I_+
\to  \left( \Gamma T\wedge A\right)\underset{C}{\wedge} EC_+\]
from \eqref{eq:universalspaceaction} to the identity as required.

We now show the map from the statement is an isomorphism.
We first treat the case $A = C_+$. For any $C$-space X
there is a natural $C$-equivariant homeomorphism
\begin{equation}\label{eq:untwist}
X\wedge C_+\xrightarrow{\cong} \mathrm{res}^C_1 X \wedge C_+ = 
\mathrm{ind}^C_1 \mathrm{res}^C_1 X, x\wedge g\mapsto g^{-1}x\wedge g
\end{equation}
and this gives rise to the natural isomorphism
\[T\wedge C_+\cong \mathrm{ind}^C_1 \mathrm{res}^C_1 T,\]
hence there is a commutative diagram
\begin{equation}\label{eq:confusingdiagram}
\begin{tikzcd}
\pi_q^C(T\wedge C_+)\arrow{d}{\cong}\arrow{r}{\mathrm{res}^C_1}
&\pi_q^{\{1\}}(T\wedge C_+)\arrow{d}{\cong}\arrow{r}{\cong}
&\pi_q(\iota^\ast (T\wedge C_+))\arrow{d}{\cong}
\\
\pi_q^C(\mathrm{ind}^C_1 \mathrm{res}^C_1 T)\arrow{r}{\mathrm{res}^C_1}
&\pi_q^{\{1\}}(\mathrm{ind}^C_1 \mathrm{res}^C_1 T)\arrow{r}{\cong}
&\pi_q(\iota^\ast \mathrm{ind}^C_1 \mathrm{res}^C_1 T).
\end{tikzcd}
\end{equation}
Now, we have $\iota^\ast \mathrm{ind}^C_1 \mathrm{res}^C_1 T = \iota^\ast T\wedge C_+$
and the $C$-action on $\pi_q(\iota^\ast \mathrm{ind}^C_1 \mathrm{res}^C_1 T)$
induced by the right vertical isomorphism is the same as the $C$-action
induced by
\[\mathrm{id}_{\iota^\ast T}\wedge \mu:\iota^\ast T\wedge C_+\wedge C_+\to \iota^\ast T\wedge C_+,\]
where $\mu:C_+\wedge C_+\to C_+$ denotes the multiplication. We show that 
the map 
\begin{equation}\label{eq:restriction}
\pi_q^C(\mathrm{ind}^C_1 \mathrm{res}^C_1 T)
\to \left(\pi_q(\iota^\ast \mathrm{ind}^C_1 \mathrm{res}^C_1 T)\right)^C
\end{equation}
is an isomorphism. This is enough, since the vertical maps and the upper right map in \eqref{eq:confusingdiagram}
are $C$-equivariant. Non-equivariantly, we have the isomorphism 
\[
\iota^\ast \mathrm{ind}^C_1 \mathrm{res}^C_1 T\cong \bigvee_{g\in C} \iota^\ast T,\]
hence also the $C$-equivariant isomorphism
\[
\pi_q(\iota^\ast \mathrm{ind}^C_1 \mathrm{res}^C_1 T)\cong \bigoplus_{g\in C} \pi_q(\iota^\ast T),
\]
where $C$ acts on the right hand side by permuting the summands.
Then the fixed points on the right hand side are the diagonal, so the projection
$\mathrm{pr}:\iota^\ast \mathrm{ind}^C_1 \mathrm{res}^C_1 T = \iota^\ast T\wedge C_+\to 
\iota^\ast T$ induces an isomorphism
\[
\left(\pi_q(\iota^\ast \mathrm{ind}^C_1 \mathrm{res}^C_1 T)\right)^C\xrightarrow{\cong}
\pi_q(\iota^\ast T),
\]
hence it is enough to show that 
\[
\pi_q^C(\mathrm{ind}^C_1 \mathrm{res}^C_1 T)\xrightarrow{\mathrm{res}^C_1}
\to \pi_q(\iota^\ast \mathrm{ind}^C_1 \mathrm{res}^C_1 T)
\xrightarrow{\mathrm{pr}_\ast} \pi_q(\iota^\ast T)
\]
is an isomorphism, which is \cite[Proposition 4.11, \pno~30]{schwedeequivariant}.

The case $A= S^n\wedge C_+$ follows directly, since $T\wedge S^n\wedge C_+ = 
\Sigma^n T\wedge C_+$. This also yields the case $A = \bigvee_{j\in J} S^n\wedge C_+$.
We now treat the general case. Since the $C$-action
is free, we have 
$A_0 \cong \bigvee_{j\in J} C_+$ and $A_n/A_{n-1}\cong \bigvee_{j\in J'} S^n\wedge C_+$,
so using the cofiber sequence
\[A_{n-1}\to A_n\to A_n/A_{n-1} \]
the claim follows for $A_n$ by induction.
%\[\pi_q\left( \left(\Gamma T\wedge X\right)/G \right) \xleftarrow{\cong}
%\pi_q\left(  \left(\Gamma T\wedge X\right)\underset{G}{\wedge} EG_+\right)
%\xrightarrow{\cong} \pi_q\left(   \mathcal{Q}^{\mathcal{U}}(\Gamma T\wedge X)^G \right)
%\xleftarrow{\cong} \pi^{G}_q(\Gamma T\wedge X) \xleftarrow{\cong}
%\pi_q^G(T\wedge X)\]
\end{proof}


\begin{lem}\label{lem:separatefixed}
\begin{enumerate}[(i)]
Let $n,\nu \in \NN_0$ such that $ \nu\le n$.
\item If $X$ is $\TT$-space, then there is a natural isomorphism 
$$\rho_{ p^n }^\ast X^{ C_{ p^n } } \cong \rho_{ p^{ n - \nu } }^\ast
\left( \rho_{ p^\nu }^\ast X^{ C_{ p^\nu } }\right)^{ C_{ p^{ n - \nu } } }.$$
\item For any $\TT$-spectrum $T$ there is a natural isomorphism
$$\rho_{ p^n }^\ast T^{ C_{ p^n } } \cong \rho_{ p^{ n - \nu } }^\ast
\left(\rho_{ p^\nu }^\ast T^{ C_{ p^\nu } }\right)^{ C_{ p^{ n - \nu } } }.$$
\end{enumerate}
\end{lem}

\begin{proof}
The second statement follows immediately from the first. To see the first claim, 
use the adjunctions between restriction and (co)induction and prove the claim
by Yoneda lemma. %Still need to fill out the details here, but it shouldn't be hard
\end{proof}

%\begin{lem}\label{lem:cycretract}
%\todo{Fix this statement to involve the free loop space}
%This will be the lemma with cyclic nerve and deformation retraction onto the circle group.
%
%The diagram
%\begin{equation}\label{eq:multiplicativity}
%\begin{tikzcd}
%\TT/C_{|i|+}\wedge \TT/C_{|j|+}\arrow{r}\arrow{d}{\simeq}
%&\TT/C_{|i+j|+}\arrow{d}{\simeq}\\
%\left| N^{ \text{cy} }( \Pi, i )_\bullet \right|\wedge \left| N^{ \text{cy} }( \Pi, j )_\bullet \right|
%\arrow{r}{\mu}
%& \left| N^{ \text{cy} }( \Pi, i+j )_\bullet \right|,
%\end{tikzcd}
%\end{equation}
%where the upper horizontal arrow is defined by $zC_{|i|}\wedge wC_{|j|}\mapsto
%zw^{ \frac{1}{|i + j|} }C_{|i + j|}$, is homotopy commutative.
%\end{lem}



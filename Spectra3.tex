\chapter{Equivariant homotopy theory.}
\section{Unstable equivariant homotopy theory.}\label{sec:unstablehomotopytheory}
Throughout this section $G$ denotes a topological group. The categories
$\mathbf{Top}_G$ and $\mathbf{Top}_{\ast G}$ have objects
(pointed) left $G$-spaces and morphisms (pointed) continuous
maps. In the pointed case we require that $G$ acts trivially on
the basepoint. Note also that we do not require maps to be equivariant.
We equip $S^n, I$ and $D^n$ with the trivial action 
and the (smash) product of two  (pointed) $G$-spaces with the diagonal action. If $X$ and $Y$ are
$G$-spaces, we let $G$ act on $\map(X,Y)$ by conjugation. Just as in the
non-equivariant case, $\mathbf{Top}_G$ is a cartesian closed category. Similarly, the smash product
and pointed mapping space give $\mathbf{Top}_{\ast G}$
a closed symmetric monoidal structure with unit $S^0$.

We denote by $\mathbf{Top}^G$ and $\mathbf{Top}_\ast^G$ the corresponding
subcategories with the same object, but $G$-equivariant maps. The corresponding
mapping spaces are obtained as the fixed points $\map(X,Y)^G$.
In this case we also have that $\mathbf{Top}^G$ is cartesian closed
and $\mathbf{Top}_\ast^G$ is closed symmetric monoidal with unit $S^0$.

\subsection{Equivariant CW-complexes and universal spaces.}
There is an equivariant analogue of CW-complexes. We first need an equivariant
generalization of the notion of attaching $n$-cells.
\begin{mydef}
Let $A$ and $X$ be $G$-spaces. We say $X$ is obtained by attaching equivariant $n$-cells
if there exists a family $\{H_j:j\in J\}$ of closed subgroups and a pushout in $\mathbf{Top}^G$:
\begin{equation}\label{eq:cwpushout}
\begin{tikzcd}
\underset{j\in J}{\coprod} G/H_j\times S^{n-1}
\arrow{r}{\varphi}\arrow[hookrightarrow]{d}
&A\arrow{d}\\
\underset{j\in J}{\coprod} G/H_j\times D^{n}
\arrow{r}{\Phi}
&X.
\end{tikzcd}
\end{equation}
\end{mydef}
In \eqref{eq:cwpushout} $G$ acts trivially on $S^{n-1}$ and $D^n$. The action on $G/H_j$ is
by left multiplication. One can verify that if $X$ is obtained from $A$ by attaching equivariant cells, there is
a canonical $G$-homeomorphism $X/A\cong \bigvee_{j\in J} G/H_{j+}\wedge S^n$.
\begin{mydef}
Let $(X,A)$ be a pair of $G$-spaces with $A$ Hausdorff. We say it is a relative $G$-CW-complex
if there is a filtration
\[
A = X_{-1}\subset X_0\subset \ldots\subset X_n\subset \ldots \subset X
\]
such that the following holds:
\begin{enumerate}
\item $X = \cup_{n = -1}^\infty X_n$.
\item $X_n$ is obtained from $X_{n-1}$ by attaching equivariant $n$-cells.
\item $X = \colim_n X_n$.
\end{enumerate}
The cellular dimension of $X$ is the smallest integer $n$ such that $X_n =X$. We refer
to the space $X_n$ as the $n$-skeleton.
In the case $A=\emptyset$ we say $X$ is a $G$-CW-complex.
\end{mydef}

We now turn to families of subgroups and universal spaces.
\begin{mydef}
A family of subgroups of the group $G$ is a set $\mathcal{F}$
of closed subgroups of $G$ such that the following holds:
\begin{enumerate}
\item If $H\in \mathcal{F}$ and $K$ is conjugate to $H$, then  $K\in \mathcal{F}$.
\item If $H\in \mathcal{F}$ and $K\subset H$, then  $K\in \mathcal{F}$.
\end{enumerate}
\end{mydef}


\begin{mydef}
Let $\mathcal{F}$ be a family of subgroups of the group $G$. A universal $G$-space
for $\mathcal{F}$ is a $G$-CW-complex $E\mathcal{F}$ such that $E\mathcal{F}^H$
is empty if $H\in \mathcal{F}$ and $E\mathcal{F}^H$
is contractible if $H\not \in \mathcal{F}$.
\end{mydef}
The following theorem is proven in \cite[Section~I.6]{diecktg}.
\begin{thm}
Let $G$ be a compact Lie group and $\mathcal{F}$ a family of subgroups.
Then a universal space for $\mathcal{F}$ exists and is unique up to
$G$-homotopy equivalence.
\end{thm}

\subsection{Change of groups.}
Let $\alpha:\Gamma\to G$ be a group homomorphism. Given a $G$-space
$X$, $\Gamma$ acts on $X$ via $\alpha$. This defines the restriction
functors 
\begin{align*}
\alpha^\ast : \mathbf{Top}^G&\to \mathbf{Top}^\Gamma,\\
\alpha^\ast : \mathbf{Top}^G_\ast&\to \mathbf{Top}^{\Gamma}_\ast.
\end{align*}
In the case that $\alpha:H\to G$ is the inclusion of a subgroup,
we will often write $\mathrm{res}^G_H\,$ instead of $\alpha^\ast$.
The restriction functors have left adjoints called induction functors.
We only need the pointed case. First, we introduce some notation.

A $G$-$\Gamma$-space is a pointed space $X$ equipped with
a left $G$-action and a right $\Gamma$-action, which
are compatible in the sense that $g(x\gamma) = (gx)\gamma$
for all $x\in X, g\in G$ and $\gamma\in \Gamma$. If $Y$ is a pointed
left $G\times \Gamma$-space,
we denote by $X\underset{\Gamma}{\wedge} Y$
the quotient of $X\wedge Y$ obtained by identifying
$x\gamma \wedge y$ with $x\wedge \gamma y$ and equip it
with the diagonal $G$-action.
Of course,  $G_+$ is a pointed $G$-$\Gamma$-space if we give it
the right $\Gamma$-action
\[
G_+\times \Gamma\to G, (g,\gamma)\mapsto g\alpha(\gamma)
\]
and we can put $\alpha_\ast Y = G_+\underset{\Gamma}{\wedge} Y$
for any $\Gamma$-space $Y$. Here we regard $Y$
as a $G\times \Gamma$ space with the action
\[
G\times \Gamma\times Y\xrightarrow{\mathrm{pr}_\Gamma\times \mathrm{id}}
\Gamma\times Y\xrightarrow{\mathrm{action}} Y.
\]This defines the induction
functor $\alpha_\ast:\mathbf{Top}^\Gamma_\ast\to 
\mathbf{Top}_\ast^G$. Again, if
$\alpha:H\to G$ is the inclusion of a subgroup, we
often write $\mathrm{ind}^G_H$ instead of $\alpha_\ast$.

For a (pointed) $G$-space $X$ and a subgroup $H$ we have
the fixed points 
\[
X^H = \{x\in X:gx = x \text{ for all } g\in H\}.
\]
This is of course still a $G$-space. If $H$ is additionally normal,
$X^H$ has a canonical $G/H$-action and in this case we
will always interpret $X^H$ as a $G/H$-space.


\subsection{Homotopy theory of equivariant spaces.}
Finally, we discuss some aspects of homotopy theory of $G$-spaces we need.
\begin{mydef} Let $X$ and $Y$ be pointed $G$ spaces.
\begin{enumerate}[(i)]
\item A $G$-homotopy is a $G$-map $h:X\wedge I_+\to Y$. We then define
$G$-homotopy equivalences and $G$-homotopies between $G$-maps
just as in the non-equivariant case.
If $\mathcal{F}$ is a family of subgroups and $f:X\to Y$ a $G$-map, 
then we say $f$ is an $\mathcal{F}$-homotopy equivalence
if it is an $H$-homotopy equivalence for all $H\in \mathcal{F}$.
\item A $G$-map $f:X\to Y$ is a weak equivalence of $G$-spaces if the induced map on
fixed points
\[f^H:X^H\to Y^H\]
is a weak equivalence of non-equivariant spaces for all closed subgroups $H$.
If $\mathcal{F}$ is a family of subgroups, 
then we say $f$ is an $\mathcal{F}$-equivalence
if $\mathrm{res}^G_H\, f: \mathrm{res}^G_H\, X\to \mathrm{res}^G_H\, Y$ 
is a weak equivalence of $H$-spaces for all $H\in \mathcal{F}$.
\end{enumerate}
\end{mydef}
Of course, we have the basic fact that $G$-homotopy equivalences are weak
equivalences of $G$-spaces. The analogous observation also holds
for families of subgroups.

We finally discuss (co)fiber sequences of $G$-spaces. We have corresponding
notions of $G$-fibrations and $G$-cofibrations by replacing
homotopy with $G$-homotopy and map with $G$-map in the non-equivariant definition.
The same arguments as usual show that cofibrations are stable under pushouts
along $G$-maps and fibrations are stable under pullbacks along $G$-maps.
This implies that the inclusion of a subcomplex is a cofibration.
If $f:X\to Y$ is a $G$-map, then its homotopy fiber $Ff$ and homotopy
cofiber $Cf$ are $G$-spaces.
\begin{mydef}
\begin{enumerate}[(i)]
\item A sequence $X\to Y\to Z$
is a \textit{cofiber sequence} of $G$-spaces
if there is a $G$-map $f:A\to B$
and a diagram
\[
\begin{tikzcd}
X\arrow{r}\arrow{d}[anchor = center, rotate = 90, yshift = -1ex]{\sim}
&Y' \arrow{d}[anchor = center, rotate = 90, yshift = -1ex]{\sim}\arrow{r}
&Z''\arrow{d}[anchor = center, rotate = 90, yshift = -1ex]{\sim}\\
A\arrow{r}{f}
&B\arrow{r}{i}
&Cf,
\end{tikzcd}
\]
which commutes up to $G$-homotopy, such that the vertical
arrows are weak equivalences of $G$-spaces.
\item We say $X\to Y\to Z$ is a fiber sequence of $G$-spaces
if there is a $G$-map $f:A\to B$
and a diagram
\[
\begin{tikzcd}
X\arrow{r}\arrow{d}[anchor = center, rotate = 90, yshift = -1ex]{\sim}
&Y \arrow{d}[anchor = center, rotate = 90, yshift = -1ex]{\sim}\arrow{r}
&Z\arrow{d}[anchor = center, rotate = 90, yshift = -1ex]{\sim}\\
Ff\arrow{r}{p}
&A\arrow{r}{f}
&B,
\end{tikzcd}
\]
which commutes up to $G$-homotopy, such that the vertical
arrows are weak equivalences of $G$-spaces.
\end{enumerate}
\end{mydef}


\section{Equivariant Orthogonal Spectra}
In this section we collect basic definitions and facts about equivariant 
stable homotopy theory. Most of the material we use can be found in \cite{mandellmay}
and \cite{hillhopkinsravenel},
but see also \cite{schwedeequivariant} and \cite[Section~3-7]{rvadams}
in the case of finite groups. We mostly follow \cite{rvadams} and remark
that almost all definitions given there are also valid for compact Lie groups.
Throughout this section we assume that $G$ is a compact Lie group.
We are mostly interested in the case of the circle group $\TT$ and its
finite cyclic subgroups. 
We first fix some notation. We denote by $\mathbf{Th}_G$ the $\mathbf{Top}_{\ast}^G$-category of
Thom complexes  as described in  \cite[Section~3]{rvadams} (this description
is also valid for compact Lie groups). It has as objects
the finite dimensional $G$-representations. We give a brief
description of the morphism
spaces. We denote by $\mathbf{L}(U,V)$ the space of linear isometries
between finite dimensional $G$-representations $U$ and $V$
and  consider the bundle
\begin{equation}\label{eq:thombundle}
\xi(U,V)\to \mathbf{L}(U,V)
\end{equation}
with total space
\[
\xi(U,V) = \{(f,v)\in \mathbf{L}(U,V)\times V: f(U)\perp v\}
\]
and bundle map projection onto the first factor. Then
the morphism space 
$\mathbf{Th}(U,V)$ is defined as the Thom space of the
bundle \eqref{eq:thombundle}.
Finally, we denote by $\mathbf{Th}_G^{\mathrm{triv}}$ the full subcategory
with objects the trivial $G$-representations.

\begin{mydef}
An orthogonal $G$-spectrum $T$ is a $\mathbf{Top}_{\ast}^G$-functor
\[
T:\mathbf{Th}_G\to \mathbf{Top}_{\ast G}.
\]
A naive orthogonal $G$-spectrum $T$ is a $\mathbf{Top}_{\ast}^G$-functor
\[
T:\mathbf{Th}^{\mathrm{triv}}_G\to \mathbf{Top}^G_{\ast}.
\]
A map of (naive) orthogonal $G$-spectra is a natural transformation of
$\mathbf{Top}_{\ast}^G$-functors. We denote the category of orthogonal
$G$-spectra by $\mathbf{Sp}^G$ and the category of
naive orthogonal $G$-spectra by $\mathbf{NSp}^G$.
\end{mydef}
We will often drop the adjective orthogonal and simply use the expression
$G$-spectrum. Also, we see that if $\iota: \mathbf{Th}_G^{\mathrm{triv}}
\to \mathbf{Th}_G$ denotes the inclusion, then any $G$-spectrum
$T$ has an underlying naive $G$-spectrum $\iota^\ast T$.


%We will define orthogonal $G$-spectra
%as $\mathbf{Top}_\ast^G$-enriched functors taking values
%in $\mathbf{Top}_{\ast G}$. We first describe the indexing category.
%
%We take a $G$\textit{-representation} to mean a real inner product
%space $U$ on which $G$ acts smoothly by linear isometries. We only
%consider $G$-representations of finite or countably infinite
%dimension. For $G$-representations $U$ and $V$ we denote
%by $\mathbf{L}(U,V)$ the set of all linear isometries and topologize
%it as a subspace of the unpointed mapping space $\map(U,V)$. 
%We let $\xi(U,V)\to \mathbf{L}(U,V)$
%be the vector bundle with total space
%\[
%\xi(U,V) = \{(f,v)\in \mathbf{L}(U,V)\times V:f(U)\perp v\}
%\]
%and bundle map projection onto the first factor. We let $G$
%act on $\xi(U,V)$ via 
%\begin{equation}\label{eq:thomspaceaction}
%(g,(f,v))\mapsto (gf(g^{-1}-),gv).
%\end{equation}
%
%\begin{mydef}
%Let $U$ and $V$ be finite dimensional $G$-representations.
%We define $\mathbf{Th}(U,V)$ as the Thom space of the bundle
%$\xi(U,V)\to \mathbf{L}(U,V)$. Explicitly, it is the quotient space
%of the fibrewise one-point compactification obtained by collapsing
%all the points at infinity to a basepoint.
%It has a $G$-action induced by \eqref{eq:thomspaceaction}.
%\end{mydef}
%
%Composition of isometries induces $G$-equivariant bundle maps
%\[
%\xi(V,W)\times \xi(U,V)\to \xi(U,W), ((f,w),(f',v))\mapsto (f'\circ f, w + f'(v))
%\]
%and this descends to a map
%\begin{equation}\label{eq:thomspacecomposition}
%\circ:\mathbf{Th}(V,W)\wedge \mathbf{Th}(U,V)\to \mathbf{Th}(U,W).
%\end{equation}
%This map has an identity, given by
%\begin{equation}\label{eq:thomspaceidentity}
%(\mathrm{id},0)\in \mathbf{Th}(U,U).
%\end{equation}
%\begin{mydef}
%The $\mathbf{Top}_{\ast G}$-category $\mathbf{Th}_G$ has as objects
%finite dimensional $G$-representations. For finite dimensional
%$G$-representations $U$ and $V$ the morphism space
%is the pointed $G$-space $\mathbf{Th}(U,V)$.  The composition is
%given by \eqref{eq:thomspacecomposition} and the identity by \eqref{eq:thomspaceidentity}.
%The $\mathbf{Top}_{\ast G}$-category $\mathbf{Th}_G^{\mathrm{triv}}$ is the subcategory
%with objects the trivial finite dimensional $G$-representations and the same
%morphism spaces.
%\end{mydef}

% We now proceed to explain that any $G$-spectrum has an 
%underlying naive $G$-spectrum. Let
%\[
%\iota: \mathbf{Th}^{\mathrm{triv}}_G\to \mathbf{Th}_G
%\]
%denote the inclusion functor. If $T$ is an orthogonal $G$-spectrum,
%$\iota^\ast T$ naturally takes values in $\mathbf{Top}^G_{\ast}$,
%since if $U$ and $V$ are trivial $G$-representations, the $G$-action
%on $\mathbf{Th}(U,V)$ is trivial, hence the $G$-equivariant map
%\[
%\mathbf{Th}(U,V)\to \map(T(U),T(V))
%\]
%factors through $\map(T(U),T(V))^G$, which are exactly the equivariant maps.
%\begin{mydef}
%Let $T$ be an orthogonal $G$-spectrum. Its underlying naive spectrum
%is the naive orthogonal $G$-spectrum $\iota^\ast T$.
%\end{mydef}
Naive and non-naive spectra are related by the following result. It plays an important
role in the construction of change of group functors. Also, sometimes it is more convenient
to construct maps on the underlying naive spectra and to extend them to maps
of $G$-spectra. A proof can be found in \cite[Proposition~A.19, \pno~147]{hillhopkinsravenel}.
\begin{thm}\label{thm:naiveisnotnaive}
The functor
\[
\iota^\ast: \mathbf{Sp}^G\to \mathbf{NSp}^G
\]
is an equivalence of categories. 
Its inverse
\[
\iota_!: \mathbf{NSp}^G\to \mathbf{Sp}^G
\]
is given by mapping a naive $G$-spectrum $T$
to its enriched left Kan extension along $\iota:\mathbf{Th}^{\mathrm{triv}}_G\to 
\mathbf{Th}_G$.
In particular, $G$-spectra
with isomorphic underlying naive $G$-spectra are isomorphic.
\end{thm}

Every $T$ spectrum comes equipped with a structure map
\begin{equation}\label{eq:structuremap}
\sigma_{U,V}: S^V\wedge T(U)\to T(U\oplus V),
\end{equation}
as well as an internal structure map
\begin{equation}\label{eq:internalstructuremap}
\sigma_{U\subseteq V}: S^{V-U}\wedge T(U)\to T(V)
\end{equation}
for finite dimensional $G$-representations $U$ and $V$.
A precise definition is given in \cite[Definition~4.6, \pno~1505]{rvadams}.
Before we proceed, we give two important examples of $G$-spectra.
\begin{bsp}
\begin{enumerate}[(i)]
\item The $G$\textit{-sphere spectrum} $\spherespectrum$
is defined by $\spherespectrum(U) = S^U$. On morphisms
it is given by sending $(f,v)\in \mathbf{Th}(U,V)$
to the one-point compactification of the map
\[
U\to V, u\mapsto f(u)+v.
\]
\item Let $X$ be a $G$-space. Its \textit{suspension spectrum}
$\Sigma^\infty X$ is defined by $(\Sigma^\infty X)(U) = S^U\wedge X$.
On morphisms it is defined by smashing the identity of $X$ with the
maps of the previous example.
\end{enumerate}
\end{bsp}

The previous examples are special cases of a more general construction,
which we will frequently use.
\begin{mydef}
Let $X$ be a pointed $G$-space and $T$ a $G$-spectrum. Their
smash product $T\wedge X$ is the composition
\[
\mathbf{Th}_G\xrightarrow{T}\mathbf{Top}_{\ast G}
\xrightarrow{-\wedge X} \mathbf{Top}_{\ast G}.
\]
The mapping spectrum $\map(X,T)$ is defined as the composition
\[
\mathbf{Th}_G\xrightarrow{T}\mathbf{Top}_{\ast G}
\xrightarrow{\map(X,-)} \mathbf{Top}_{\ast G}.
\]
These operations assemble to $\mathbf{Top}_{\ast}^G$-functors
\begin{align*}
-\wedge X: \mathbf{Sp}^G&\to \mathbf{Sp}^G,\\
\map(X,-):\mathbf{Sp}^G &\to \mathbf{Sp}^G,
\end{align*}
which form an adjoint pair.
In the case of a finite dimensional $G$-representation $U$
we write $\Sigma^U T$ and $\Omega^U T$ instead
of $T\wedge S^U$ and $\map(S^U,T)$ respectively. We refer to the corresponding functors
as the suspension and loop functors.
\end{mydef}
\begin{mydef}
A $G$-homotopy is a map of $G$-spectra $f:T\wedge I_+\to T'$. Just as for spaces,
we have the notion of $G$-homotopic maps of spectra and $G$-homotopy
equivalences.
\end{mydef}




%\begin{prop}
%The categories of naive orthogonal $G$-spectra and of orthogonal spectra with $G$-action are isomorphic.
%\end{prop}
%
%\begin{proof}
%Let $T$ be a naive orthogonal $G$-spectrum. We put $X_n=T(\RR^n)$. To define the $O(n)\times G$-action, note that $f\in O(n)$ implies
%$(f,0)\in \mathcal{Th}(\RR^n, \RR^n)$ and that $$T(f,0): X_n\rightarrow X_n$$ is $G$-equivariant. This means we can define the action by
%$$O(n)\times G\times X_n\rightarrow X_n, (f,g,x)\rightarrow T(f,0)(gx).$$
%
%Next, we define the structure map
%$$\sigma_n: X_n\wedge S^1\rightarrow X_{n+1}$$
%as the adjoint of the map
%$$S^1\rightarrow \mathcal{Th}(\RR^n,\RR^{n+1})\rightarrow \text{Map}_{\mathcal{T}^G}(T(\RR^n),T(\RR^{n+1}))=\text{Map}_{\mathcal{T}^G}(X_n,X_{n+1}).$$
%Then $\sigma_n$ is $G$-equivariant and the iterated structure maps are as well. 
%
%Conversely, if $X$ is an orthogonal spectrum with a $G$-action, we will a naive orthogonal spectrum $T$. Note that on objects it suffices to define
%$T(R^n)$ for all $n\in N$. For if $V$ is an inner product space of dimension $n$ and $f:V\rightarrow \RR^n$ an isometric isomorphism, it
%is easy to see that $(f,0)\in \mathcal{Th}(V,\RR^n)$ is an isomorphism.
%\end{proof}

\subsection{Homotopy theory of equivariant orthogonal spectra.}
The definition of the homotopy groups of a spectrum requires the notion of a 
(complete) $G$-universe \cite[Definition~II.2.2, \pno~30]{mandellmay}.
\begin{mydef}
A $G$-universe is a $G$-representation $\mathcal{U}$ of countably infinite dimension
such that $\mathcal{U}^G \neq 0$ and for any finite dimensional $G$-representation
$U$ that embeds in $\mathcal{U}$, also $\oplus_{n\in \NN}U$ embeds in $\mathcal{U}$. 
We say $\mathcal{U}$ is complete if
any finite dimensional $G$-representation embeds in $\mathcal{U}$.
\end{mydef}
In the setup of \cite{mandellmay} the universe is built into the definition
of a $G$-spectrum, but we find it more convenient to define $G$-spectra
without universes. The downside is that we need a choice of universe
to define homotopy groups. For complete $G$-universes, this choice is
irrelevant, see for example the discussion in \cite[\pno~1510-1511]{rvadams}.
\begin{mydef}
Let $T$ be a $G$-spectrum and $H$ a closed subgroup of $G$. Choose
a complete $G$-universe $\mathcal{U}$ and denote by $\mathcal{P}(\mathcal{U})$
the poset of all finite dimensional subrepresentations of $\mathcal{U}$. Define
\begin{equation}\label{eq:equivarianthomotopygroup}
\pi_q^H(T) = 
\begin{cases}
\underset{U\in \mathcal{P}(\mathcal{U})}{\colim} \pi_q\left(\left(\Omega^UT(U)\right)^H\right) &\text{ if } q\ge 0,\\
\underset{U\in \mathcal{P}(\mathcal{U})}{\colim} \pi_{0}\left(\left(\Omega^UT(U\oplus \RR^{-q})\right)^H\right) &\text{ if } q< 0.
\end{cases}
\end{equation}
The structure maps for the colimits above are obtained by first applying $H$-fixed points and then $\pi_q(-)$
to the maps
\begin{align}
\Omega^UT(U)&\xrightarrow{\Omega^U \tilde \sigma_{U\subseteq V}} \Omega^U\Omega^{V-U} T(V)\cong \Omega^{V} T(V)
&\text{ if } q\ge 0,\label{eq:homotopygroupstructuremap1}\\
\Omega^UT(U\oplus \RR^{-q})&\xrightarrow{\Omega^U \tilde \sigma_{U\oplus \RR^{-q}\subseteq V\oplus \RR^{-q}}} 
\Omega^U\Omega^{V-U} T(V\oplus \RR^{-q})\cong \Omega^{V} T(V\oplus \RR^{-q})
&\text{ if } q<0 \label{eq:homotopygroupstructuremap2},
\end{align}
where we denote by $\tilde \sigma_{U\subseteq V}$  the adjoint of \eqref{eq:internalstructuremap}
and use that $V\oplus \RR^{-q}-U\oplus \RR^{-q} = V-U$.
\end{mydef}
\begin{mydef}
Let $f:T\to T'$ be a map of $G$-spectra.
\begin{enumerate}[(i)]
\item We say $f$ is a weak equivalence of $G$-spectra
if the map
\[
f_\ast:\pi_q^H(T)\to \pi_q^H(T')
\]
is an isomorphism for all $q\in \ZZ$ and all closed subgroups $H$
of $G$.
\item We say $f$ is a level equivalence if $f:T(U)\to T'(U)$
is a weak equivalence of $G$-spaces for any finite dimensional
$G$-representation $U$.
\item More generally, for a family of subgroups $\mathcal{F}$,
we say $f$ is an $\mathcal{F}$-equivalence if
\[
f_\ast:\pi_q^H(T)\to \pi_q^H(T')
\] 
is an isomorphism for all $H\in \mathcal{F}$.
\end{enumerate}
\end{mydef}
For brevity we will often drop the adjective weak when
talking about weak equivalences of $G$-spectra.
We will often denote an equivalence of $G$-spectra by
$T\xrightarrow{\sim} T'$ or by $T\simeq T'$. In equivariant
stable homotopy theory one is mostly interested in the weak
equivalences. The usefulness of the level equivalences comes
from the following basic result from \cite[Lemma~III.3.3, \pno~45]{mandellmay}.
\begin{lem}
A level equivalence of $G$-spectra is a weak equivalence of $G$-spectra.
\end{lem}

The category $\mathbf{Sp}^G$ can be given the structure of a model
category \cite[Theorem~III.4.2, \pno~47]{mandellmay} such that the weak equivalences
of $G$-spectra are exactly the weak equivalences of the model structure. 
It is referred to as the \textit{stable model structure}
and admits Quillen's
small object argument. By \cite[Theorem~2.1.14, \pno~33]{hoveymodelcats} 
this implies the existence of a fibrant replacement 
functor $Q$ and a cofibrant replacement functor $\Gamma$
together with weak equivalences of $G$-spectra
\[
r_T: T\xrightarrow{\sim} QT
\]
and
\[
\gamma_T:\Gamma T\xrightarrow{\sim} T,
\]
which are both natural in $T$. The next result justifies the expression
stable model structure.


\begin{lem}\label{lem:spectrumsuspensionshift}
\begin{enumerate}[(i)]
\item For any finite dimensional $G$-representation $U$
the unit $\eta:T\to \Omega^U\Sigma^UT$ is an equivalence
of $G$-spectra.
\item There is a natural isomorphism of abelian groups $\Sigma:\pi_{q}^H(T)\xrightarrow{\cong} \pi^H_{q+1}(\Sigma T) $.
\end{enumerate}
\end{lem}
\begin{proof}
The first assertion is \cite[Lemma~3.8, \pno~46]{mandellmay}. For (ii)
we show there is a natural isomorphism $\pi_q^H(\Omega^1 T)\cong \pi_{q+1}^H(T)$
and apply (i). The natural isomorphisms of abelian groups
\[
(\Omega^U\Omega^1T(U))^H\cong\Omega^1(\Omega^UT(U))^H
\]
and 
\[
(\Omega^U\Omega^1T(U\oplus \RR^{-q}))^H\cong\Omega^1(\Omega^UT(U\oplus\RR^{-q}))^H
\]
commute with \eqref{eq:homotopygroupstructuremap1} and  \eqref{eq:homotopygroupstructuremap2} 
and by adjointness we have the isomorphisms of abelian groups
\[
\pi_q(\Omega^1(\Omega^UT(U))^H)\cong \pi_{q+1}((\Omega^UT(U))^H)
\]
and
\[
\pi_q(\Omega^1(\Omega^UT(U\oplus\RR^{-q}))^H)\cong \pi_{q+1}((\Omega^UT(U\oplus\RR^{-q}))^H).
\]
The claim now follows by passing to the colimits.
\end{proof}
\begin{rem}\label{rem:spectrumsuspensionshift}
Tracing through the definitions, one obtains the following easier description
of the suspension isomorphism $\Sigma:\pi_{q}^H(T)\xrightarrow{\cong} \pi^H_{q+1}(\Sigma T) $.
Given $x\in \pi_q^H(T)$ and a representative
$f:S^q\to (\Omega^UT(U))^H$, $\Sigma (x)$ is the class of
\[
S^q\wedge S^1\xrightarrow{f\wedge \mathrm{id}}(\Omega^UT(U))^H\wedge S^1
\to (\Omega^UT(U)\wedge S^1)^H, 
\]
where the second map is obtained by applying $H$-fixed points to the adjoint of
the map
\[
\Omega^UT(U)\wedge S^U\wedge S^1\xrightarrow{ev\wedge \mathrm{id}} T(U)\wedge S^1.
\]
\end{rem}
The fibrant objects in the stable model structure can be easily
characterized and possess good homotopical properties.
\begin{mydef}
Let $T$ be a $G$-spectrum.
\begin{enumerate}[(i)]
\item We call $T$ good if the structure map \eqref{eq:structuremap}
is a closed embedding.
\item $T$ is a $G$-$\Omega$-spectrum if the adjoint
\begin{equation}
\tilde \sigma_{U\subseteq V}: T(U)\to \Omega^{V-U}T(V)\label{eq:adjointinternalstructuremap}
\end{equation}
of the internal structure map \eqref{eq:internalstructuremap} is
a weak equivalence of $G$-spaces.
\end{enumerate}
\end{mydef}

\begin{lem}
Let $T$ be a $G$-spectrum.
\begin{enumerate}[(i)]
\item $T$ is a $G$-$\Omega$-spectrum iff it is fibrant
in the stable model structure on $\mathbf{Sp}^G$.
\item If $T$ is a $G$-$\Omega$-spectrum, there is a natural isomorphism
of abelian groups
\begin{equation}\label{eq:homotopygroupomegaspectrum}
\pi_q^H(T)\cong
\begin{cases}
\pi_q\left(T(0)^H\right) &\text{ if } q\ge 0,\\
\pi_0\left(T(\RR^{-q})^H\right) &\text{ if } q<0.
\end{cases}
\end{equation}
\end{enumerate}
\end{lem}
\begin{proof}
For (i) see \cite[Proposition~III.4.12, \pno~50]{mandellmay}.
The proof of \cite[Lemma~III.3.3, \pno~45]{mandellmay} implies that
\eqref{eq:homotopygroupstructuremap1} and \eqref{eq:homotopygroupstructuremap2}
are weak equivalences of $G$-spaces if $T$ is a $G$-$\Omega$-spectrum,
hence the structure maps for the colimit in \eqref{eq:equivarianthomotopygroup}
are all isomorphisms. This yields (ii).
\end{proof}
In the remainder of this subsection we explain how to obtain a reduced homology
theory from a $G$-spectrum. For the axioms of a reduced homology
theory we refer to \cite[\pno~110]{mayconcise}. We fix a $G$-spectrum $T$
and a subgroup $H$. Given a pointed $G$-CW-complex $X$
we put
\begin{equation}\label{eq:spectrumhomology}
\tilde E_q(X) = \pi_q^H(T\wedge X).
\end{equation}
To apply the above definition to non-equivariant CW-complexes, we equip
these with the trivial action.
We have the following result from \cite[Theorem~III.3.5]{mandellmay}.
\begin{lem}\label{lem:spectrumwedge}
Let $\{T_i:i\in I\}$ be a family of $G$-spectra. The inclusion
$\iota_j:T_j\to \bigvee_{i\in I} T_i$ of each summand into the wedge
induces for each subgroup $H$ and each integer $q\in \ZZ$ an isomorphism 
\begin{equation}
\bigoplus_{i\in I} \pi_q^H(T_i)\xrightarrow{\cong} \pi_q^H\left(\bigvee_{i\in I} T_i\right).
\end{equation}
\end{lem}

Lemma \ref{lem:spectrumwedge} and \ref{lem:spectrumsuspensionshift} together with the isomorphism
\[
T\wedge \bigvee_{i\in I} X_i\cong \bigvee_{i\in I} T\wedge X_i
\] imply that \eqref{eq:spectrumhomology} satisfies the additivity and suspension axioms. The weak equivalence axiom
is a consequence of the fact that if $X\to Y$ is a weak equivalence of $G$-spaces,
$T\wedge X\to T\wedge Y$ is a level equivalence. To see that the exactness
axiom also holds we take brief detour into cofiber sequences of $G$-spectra.

\begin{mydef}\label{def:cofibersequencespectra}
Let $f:T\to T'$ a map of $G$-spectra.
\begin{enumerate}[(i)]
\item The homotopy cofiber of $f$ is the $G$-spectrum $Cf$ such that 
$(Cf)(U)$ is the homotopy cofiber of $f:T(U)\to T'(U)$ for
any finite dimensional $G$-representation $U$.
It comes equipped with a natural inclusion 
$i:T'\to Cf$.
\item We say $T\to T'\to T''$ is a cofiber sequence
if there is a map of $G$-spectra $f:X\to Y$
and a diagram
\[
\begin{tikzcd}
T\arrow{r}\arrow{d}[anchor = center, rotate = 90, yshift = -1ex]{\sim}
&T' \arrow{d}[anchor = center, rotate = 90, yshift = -1ex]{\sim}\arrow{r}
&T''\arrow{d}[anchor = center, rotate = 90, yshift = -1ex]{\sim}\\
X\arrow{r}{f}
&Y\arrow{r}{i}
&Cf,
\end{tikzcd}
\]
which commutes up to $G$-homotopy, such that the vertical
arrows are equivalences of $G$-spectra.
\item The homotopy fiber of $f$ is the $G$-spectrum
$Ff$ such that $(Ff)(U)$ is the homotopy fiber of the map
$f:T(U)\to T'(U)$ for any finite dimensional $G$-representation $U$.
It has a natural projection $p:Ff\to X$.
\item We say $T\to T'\to T''$ is a fiber sequence
if there is a map of $G$-spectra $f:X\to Y$
and a diagram
\[
\begin{tikzcd}
T\arrow{r}\arrow{d}[anchor = center, rotate = 90, yshift = -1ex]{\sim}
&T' \arrow{d}[anchor = center, rotate = 90, yshift = -1ex]{\sim}\arrow{r}
&T''\arrow{d}[anchor = center, rotate = 90, yshift = -1ex]{\sim}\\
Ff\arrow{r}{p}
&X\arrow{r}{f}
&Y,
\end{tikzcd}
\]
which commutes up to $G$-homotopy, such that the vertical
arrows are equivalences of $G$-spectra.
\end{enumerate}
\end{mydef}
Note that if $f:X\to Y$ is a map of $G$-spaces and $T$ a $G$-spectrum,
then we have the isomorphisms 
\[
T\wedge Cf\cong C(\mathrm{id}\wedge f) \text{ and } T\wedge Ff\cong F(\mathrm{id}\wedge f),
\]
which imply that if $X\to Y\to Z$ is a (co)fiber sequence of $G$-spaces,
$T\wedge X\to T\wedge Y\to T\wedge Z$ is a (co)fiber sequence of $G$-spectra.
The following result is \cite[Theorem~III.3.5, \pno~45]{mandellmay}.
Of course, by functoriality the analogous result also holds for arbitrary (co)fiber
sequences.

\begin{thm}\label{thm:(co)fibersequences}
Let $f:T\to T'$ be a map of $G$-spectra. For any subgroup $H$
there are natural long exact sequences
\[
\ldots\to\pi_q^H(T)\xrightarrow{f_\ast} \pi_q^H(T')
\xrightarrow{i_\ast} \pi_q^H(Cf)\xrightarrow{\partial}
\pi_{q-1}^H(T)\to \ldots
\]
and
\[
\ldots \to \pi_q^H(Ff)\xrightarrow{p_\ast} \pi_q^H(T)\xrightarrow{f_\ast} \pi_q^H(T')
\xrightarrow{\partial} \pi_{q-1}^H(Ff)\to \ldots
\]
of abelian groups.
\end{thm}
This immediately implies that \eqref{eq:spectrumhomology} satisfies the exactness axiom,
that is it defines a reduced homology theory. Standard results on homology theories
(for example \cite[Theorem~10.8.1 and Proposition~10.8.4, \pno~271-272]{diecktop})
then yield the following proposition.

\begin{prop}\label{prop:colimitshomotopygroups}
Let $X_0\to X_1\to\ldots \to X_n\to \ldots$ be a sequence of cofibrations of pointed $G$-spaces
and let $X = \colim_n X_n$. Then for each $G$-spectrum $T$, each integer $q\in \ZZ$
and each subgroup $H$ of $G$ the structure maps $X_n\to X$ induce an isomorphism
\begin{equation}
\underset{n}{\colim} \,\pi_q^H(T\wedge X_n)\xrightarrow{\cong} \pi_q^H(T\wedge X).
\end{equation}

\end{prop}


\subsection{Change of groups.}
Let $\alpha:\Gamma\to G$ be a group homomorphism.
Analagously to the case of spaces, there are change of group
functors
\begin{align}\label{eq:spectrumrestrictioninduction}
\begin{split}
\alpha^\ast:\mathbf{Sp}^G  &\to \mathbf{Sp}^\Gamma,\\
\alpha_\ast:\mathbf{Sp}^\Gamma  &\to \mathbf{Sp}^G.
\end{split}
\end{align}
We refer to \cite[Construction~6.5 and Construction~6.7, \pno~1519]{rvadams}
for details.
%We do this first for naive spectra.
%\begin{mydef}
%If $T$ is a naive $G$-spectrum we define its \textit{restriction along}
%$\alpha$ as the composite
%\[
%\mathbf{Th}^{\mathrm{triv}}_\Gamma = \mathbf{Th}^{\mathrm{triv}}_G\xrightarrow{T}
%\mathbf{Top}_{\ast G}\xrightarrow{\alpha^\ast}\mathbf{Top}_{\ast \Gamma}.
%\]
%If $T$ is a naive $\Gamma$-spectrum its \textit{induction along} $\alpha$ is the composite
%\[
%\mathbf{Th}^{\mathrm{triv}}_G = \mathbf{Th}^{\mathrm{triv}}_\Gamma\xrightarrow{T}
%\mathbf{Top}_{\ast \Gamma}\xrightarrow{\alpha_\ast}\mathbf{Top}_{\ast G}.
%\]
%By abuse of notation, we denote the resulting functors as
%\begin{align}
%\alpha^\ast:\mathbf{NSp}^G  &\to \mathbf{NSp}^\Gamma,\\
%\alpha_\ast:\mathbf{NSp}^\Gamma  &\to \mathbf{NSp}^G.
%\end{align}
%\end{mydef}
As in the case of spaces, if $H$ is a subgroup of $G$ and $\alpha:H\to G$ denotes the inclusion,
we often write $\mathrm{res}^G_H\,$ and $\mathrm{ind}^G_H$ instead of
$\alpha^\ast$ and $\alpha_\ast$.
%Using theorem \ref{thm:naiveisnotnaive} we can extend the previous definition to non-naive
%spectra.
%\begin{mydef}
%For a $G$-spectrum $T$ we define its \textit{restriction along} $\alpha$ as
%the $\Gamma$-spectrum $\iota_!\alpha^\ast\iota^\ast T$.
%Given a $\Gamma$-spectrum $T$ we define its \textit{induction along} $\alpha$ as
%the $\Gamma$-spectrum $\iota_!\alpha_\ast\iota^\ast T$.
%We abuse notation and denote the resulting functors by \eqref{eq:spectrumrestrictioninduction}.
%\end{mydef}
%Again, if $H$ is a subgroup of $G$ and $\alpha:H\to G$ denotes the inclusion,
%we often write $\mathrm{res}^G_H\,$ and $\mathrm{ind}^G_H$ instead of
%$\alpha^\ast$ and $\alpha_\ast$. 
The next lemma records basic
homotopical properties of the restriction functors.

\begin{lem}\label{lem:homotopygroupsrestriction}
Let $T$ be a $G$-spectrum and $H$ a subgroup of $G$.
\begin{enumerate}[(i)]
\item If $\alpha:\Gamma\to G$ is an isomorphism, then for any $G$-spectrum $T$
and any integer $q\in \ZZ$
we have a natural isomorphism of abelian groups $\pi_q^H(T)\cong \pi_q^{\alpha^{-1}(H)}(\alpha^\ast T)$.
Consequently a map $f:T\to T'$ is an equivalence of $G$-spectra iff
$\alpha^\ast f:\alpha^\ast T\to \alpha^\ast T'$ is an equivalence of $\Gamma$-spectra.
\setcounter{counter}{\value{enumi}}
\end{enumerate}
Now assume that $G = \TT$ or that $G$ is finite.
\begin{enumerate}[(i)]
\setcounter{enumi}{\value{counter}}
\item $\pi_q^H(T)\cong \pi_q^H(\mathrm{res}^G_H\, T)$.
\item If $f:T\to T'$ is an equivalence of $G$-spectra, then $\mathrm{res}^G_H\, f:
\mathrm{res}^G_H\,T\to \mathrm{res}^G_H\, T'$ is an equivalence of
$H$-spectra.
\end{enumerate}
\end{lem}
\begin{proof}
(i) In the case that $\alpha$ is an isomorphism we have an isomorphism of
$\mathbf{Top}_{\ast}$-categories $(\alpha^{-1})^\ast:\mathbf{Th}_\Gamma\xrightarrow{\cong} \mathbf{Th}_G$.
For any $G$-spectrum $T$ we consider the composition
\begin{equation}\label{eq:spectrumrestrictionalternative}
S:\mathbf{Th}_\Gamma\xrightarrow{(\alpha^{-1})^\ast} \mathbf{Th}_G
\xrightarrow{T}\mathbf{Top}_{\ast G}
\xrightarrow{\alpha^\ast} \mathbf{Top}_{\ast \Gamma}.
\end{equation}
If $\mathcal{U}$ is a complete $\Gamma$-universe, then $(\alpha^{-1})^\ast\mathcal{U}$
is a complete $G$-universe. It is then immediate from the definition that 
$\pi_q^{\alpha^{-1}(H)}(S)\cong \pi_q^H(T)$.
By construction $S$ has the same underlying naive $\Gamma$-spectrum as $\alpha^\ast T$,
hence they are isomorphic by theorem \ref{thm:naiveisnotnaive}, yielding the result.

(ii) By the remarks in \cite[Construction~6.7, \pno~1519]{rvadams}
there is a natural isomorphism $(\mathrm{res}^G_H\, T)
(\mathrm{res}^G_H\, U)\cong \mathrm{res}^G_H\, T(U)$
of $H$-spaces
for any finite dimensional $G$-representation $U$.
There it is assumed $G$ is finite, but the arguments
also work in the case $G = \TT$.

Now suppose $U$ is a finite dimensional $H$-representation.
We can always find a finite dimensional $G$-representation
$V$ such that $U$ embeds into $\mathrm{res}^G_H\,V$.
This means that for a $G$-universe $\mathcal{U}$,
the set of all $\mathrm{res}^G_H\, V$, where $V$
denotes a $G$-representation, is cofinal in $\mathcal{P}(\mathrm{res}^G_H\,\mathcal{U})$.
We argue below $\mathrm{res}^G_H\,\mathcal{U}$ is a complete
$H$-universe, so the claim follows, since the definition
of homotopy groups does not depend on the
choice of universe (see the discussion in \cite[\pno~1510-1511]{rvadams}).


If $G$ is finite, we have a canonical choice for $\mathcal{U}$,
namely the countable direct sum of the regular representation
and it follows immediately that $\mathrm{res}^G_H\, \mathcal{U}$
is a complete $H$-universe. In the case of $G=\TT$ we also 
have a canonical choice of universe, namely
\[
\mathcal{U} = \bigoplus_{j\in \NN_0} \bigoplus_{\NN} \CC(j),
\]
where $\TT$ acts on $\CC(j)$ by
\[
\TT\times \CC(j)\to \CC(j), (z,w)\mapsto z^jw.
\]
In this case we also see that $\mathrm{res}^G_H\,\mathcal{U}$
is a complete $H$-universe.
Finally, (iii) is a direct consequence of (ii).
\end{proof}
\begin{rem}
The second and third statement in the previous lemma
are also true for arbitrary compact Lie groups. The key point
is of course showing that $\mathrm{res}^G_H\,\mathcal{U}$
is a complete $H$-universe. One can do this using a consequence
of the Peter-Weyl theorem \cite[Theorem~III.3.1, \pno~134]{broeckertomdieckliegroups},
which states that for any irreducible finite dimensional
$H$-representation $U$ there is an irreducible finite dimensional
$G$-representation $V$ such that $U\subset \mathrm{res}^G_H\, V$
\cite[Theorem~III.4.5, \pno~137]{broeckertomdieckliegroups}.
\end{rem}

Next, we turn to the notion of fixed point spectra. These will play a major role in this thesis. 
First, we introduce some notation. We denote by
$j:\mathbf{Th}_{\{1\}}\to \mathbf{Th}_G$ the inclusion.
Then $j^\ast T$ takes values in $\mathbf{Top}_\ast^G $.
Now let $H$ be a subgroup of $G$.
Then $\mathbf{Th}_G^{H\mathrm{-triv}}$ denotes the full
subcategory of $\mathbf{Th}_G$ which has as objects
$G$-representations $U$ such that $H$ acts trivially on $U$.
Moreover, we denote by $\mathbf{Top}_{\ast G}^H$ the subcategory
of $\mathbf{Top}_{\ast G}$ with the same objects, but with morphisms
$H$-equivariant maps. If we additionally assume that $H$ is normal,
then the projection $p:G\to G/H$ induces a functor
$p^\ast:\mathbf{Th}_{G/H}\to\mathbf{Th}_G^{H\mathrm{-triv}}$
and one sees that the composition
\[
\mathbf{Th}_{G/H}\xrightarrow{p^\ast}\mathbf{Th}_G^{H\mathrm{-triv}}
\xrightarrow{T} \mathbf{Top}_{\ast G}
\]
takes values in $\mathbf{Top}_{\ast G}^H$.
\begin{mydef}
Let $T$ be a $G$-spectrum and $H$ a subgroup of $G$.
\begin{enumerate}[(i)]
\item The non-equivariant $H$-fixed point spectrum of $T$, denoted
by $\mathrm{res}_1 T^H$, is given by the composition
\[
\mathbf{Th}_{\{1\}}\xrightarrow{j}\mathbf{Th}_G \xrightarrow{T}
\mathbf{Top}_\ast^G \xrightarrow{(-)^H} \mathbf{Top}_\ast.
\]
\item Suppose additionally that $H$ is normal. We define the $H$-fixed
point spectrum of $T$, denoted by $T^H$, as the $G/H$ spectrum
given by the composition
\[
\mathbf{Th}_{G/H}\xrightarrow{p^\ast}\mathbf{Th}_G^{H\mathrm{-triv}}
\xrightarrow{T} \mathbf{Top}_{\ast G}^H\xrightarrow{(-)^H} \mathbf{Top}_{\ast G/H}.
\]
\end{enumerate}
\end{mydef}
\begin{rem}
The construction above is also referred to in the literature as the
\textit{categorical fixed points} or \textit{naive fixed points}, 
see for example \cite[\pno~82]{mandellmay} and \cite[7.1, \pno~66]{schwedeequivariant}.
In general taking fixed points does not behave well on a homotopical level in the
sense that for a $G$-equivalence $f:T\xrightarrow{\sim}T'$ the induced
map $f^H: T^H\to T'^H$ of $G/H$-spectra is not necessarily
a weak equivalence, see \cite[Warning~3.6, \pno~80]{mandellmay}.
We warn the reader that for this reason some authors use the notation 
$T^H$ to denote instead the \textit{derived} fixed points, which are the categorical
fixed points of a weakly equivalent $\Omega$-spectrum.
\end{rem}


It is evident from the definition that taking fixed point spectra is functorial.
In contrast to the general case, fixed point functors have good homotopical
properties for $G$-$\Omega$-spectra.
\begin{lem}\label{lem:omegaspectrafixedpoints}
Let $T$ be a $G$-$\Omega$-spectrum.
\begin{enumerate}[(i)]
\item For any integer $q\in \ZZ$ and any subgroup $H$ there is a natural isomorphism 
of abelian groups $\pi_q^{\{1\}}(\mathrm{res}_1 T^H)\cong \pi_q^H(T)$.
\item  If $H$ is a normal subgroup $T^H$ is a $G/H$-$\Omega$-spectrum. Additionally,
if $K$ is a subgroup containing $H$, there is a natural isomorphism of abelian groups
$\pi_q^{K/H}(T^H)\cong \pi_q^K(T)$ for any integer $q\in \ZZ$.
\item If $T'$ is a $G$-$\Omega$-spectrum, then $f:T\to T'$ is an equivalence of $G$-spectra
iff $f^H: \mathrm{res}_1 T^H\to \mathrm{res}_1 T'^H$ is an equivalence of non-equivariant spectra
for all subgroups $H$. Moreover, if $f$ is an equivalence of $G$-spectra and $H$ is normal, the induced
map $f^H:T^H\to T'^H$ is an equivalence of $G/H$-spectra.
\end{enumerate}
\end{lem}

\begin{proof}
That $T^H$ is a $G/H$-$\Omega$-spectrum if $H$ is normal 
is proven in \cite[Lemma~6.20, \pno~24]{rvadams}. It is assumed there that $G$
is finite, but the proof also works for compact Lie groups. The rest follows
from \eqref{eq:homotopygroupomegaspectrum} and the equality 
$
(X^H)^{K/H}= X^K
$
for any $G$-space $X$. 
\end{proof}

We end this subsection with three technical lemma's on fixed point spectra, which we will need later.

\begin{lem}\label{lem:separatefixed}
Let $n, s \in \NN_0$ such that $ s\le n$.
 For any $\TT$-spectrum $T$ there is a natural isomorphism of $\TT$-spectra
\[
\rho_{ p^n }^\ast T^{ C_{ p^n } } \cong \rho_{ p^{ n - s } }^\ast
\left(\rho_{ p^s }^\ast T^{ C_{ p^s } }\right)^{ C_{ p^{ n - s } } }.
\]
\end{lem}

\begin{proof}
For a $\TT-$space $X$ we have the equality
$\rho_{ p^n }^\ast X^{ C_{ p^n } } = \rho_{ p^{ n - s } }^\ast
\left(\rho_{ p^s }^\ast X^{ C_{ p^s } }\right)^{ C_{ p^{ n - s } } }$.
This implies that $\rho_{ p^n }^\ast T^{ C_{ p^n } } $
and $\rho_{ p^{ n - s } }^\ast
\left(\rho_{ p^s }^\ast T^{ C_{ p^s } }\right)^{ C_{ p^{ n - s } } }$
have the same underlying naive spectra, so the claim
follows from theorem \ref{thm:naiveisnotnaive}.
\end{proof}

\begin{lem}\label{lem:wedgefixed}
Let $\{T_i\}_{i\in I}$ be a family of  orthogonal $G$-spectra and $H$ a closed normal subgroup. Then the inclusion
$$T_j\rightarrow \bigvee_{i\in I } T_i$$ induces a map
$$QT_i\rightarrow Q\left(\bigvee_{j \in I } T_j\right).$$ After taking fixed points
this yields a map 
$$\bigvee_{i\in I}(QT_i)^H\rightarrow \left(Q\bigvee_{i \in I } T_i\right)^H.$$
This map is an equivalence of $G/H$-spectra.
\end{lem}

\begin{proof}
Let $K$ be a closed subgroup containing $H$. The claim follows from the
following commutative diagram
$$
\begin{tikzcd}
\pi_n^{K/H}(\bigvee_{i\in I}(QT_i)^H) \arrow{r} 
&\pi_n^{K/H}((Q(\bigvee_{i \in I } T_i))^H) \arrow{d}[swap]{\cong} \\
\bigoplus_{i\in I} \pi_n^{K/H}((QT_i)^H) 
\arrow{u}{\cong}[swap]{\oplus (\iota_{QT_i})_\ast} 
\arrow{d}[swap]{\cong}
&\pi_n^K(Q(\bigvee_{i \in I } T_i))\\
\bigoplus_{i\in I} \pi_n^{K}(QT_i)
& \pi_n^K(\bigvee_{i \in I } T_i) \arrow{u}{\cong}[swap]{(r_{\bigvee_{i \in I } T_i})_\ast}\\
\bigoplus_{i\in I} \pi_n^{K}(T_i) \arrow{u}{\cong}[swap]{\oplus (r_{T_i})_\ast}  
\arrow{ru}{\cong}[swap]{\oplus (\iota_{T_i})_\ast} 
%& \bigoplus_{i \in I} \pi_n^K(T_i) \arrow{u}{\cong}
\end{tikzcd}
$$
where the top horizontal map is the map from the statement, $\iota_{QT_i}:QT_i
\rightarrow \bigvee_{j\in I} QT_i$ is the inclusion and the undecorated vertical arrows 
are the isomorphisms from lemma \ref{lem:omegaspectrafixedpoints} (ii). The other arrows
are isomorphisms by lemma \ref{lem:spectrumwedge} and by the fact
that $r$ is a weak equivalence.
\end{proof}

\begin{lem}\label{lem:smashfixednonequivariant}
Let $T$ be an orthogonal $G$-$\Omega$-spectrum and let $X$ be a pointed CW-complex with trivial $G$-action.
Then for any closed normal subgroup $H$ the map 
$ T\wedge X\to Q(T\wedge X)
$
induces an equivalence of $G/H$-spectra $T^H\wedge X\xrightarrow{\sim}Q(T\wedge X)^H$.
\end{lem}

\begin{proof}
For $X=D^n$ the claim is obvious, since in this case both sides have trivial homotopy groups. 
We now prove the claim  for $X=S^n$ by
induction. We have $S^0\wedge T\cong T$, so the claim follows 
from the fact that $T$ is a $G$-$\Omega$-spectrum. For the inductive step, consider the cofiber sequence
$$S^{n-1}\rightarrow D^n\rightarrow S^n.$$
Smashing with $T$ yields a cofiber sequence of $G$-spectra
$$T\wedge S^{n-1}\rightarrow T\wedge D^n\rightarrow T\wedge S^n$$
and smashing with $T^H$ yields a cofiber sequence of $G/H$-spectra
\[
T^H\wedge S^{n-1}\rightarrow T^H\wedge D^n\rightarrow T^H\wedge S^n.
\]
By theorem \ref{thm:(co)fibersequences} the top row in the following diagram
becomes a long exact sequence after taking homotopy groups:
$$
\begin{tikzcd}
\arrow{r} \arrow{d}[anchor = center, rotate = 90, yshift = -1ex]{\sim}
T^H\wedge S^{n-1}
&T^H\wedge D^{n}
\arrow{r} \arrow{d}[anchor = center, rotate = 90, yshift = -1ex]{\sim}
&T^H\wedge S^{n}
\arrow{d}
\\
(Q(T\wedge S^{n-1}))^H\arrow{r}
&(Q(T\wedge D^n))^H\arrow{r}
&(Q(T\wedge S^{n}))^H.
\end{tikzcd}
$$
By definition \ref{def:cofibersequencespectra} (iv), fibrant replacement preserves cofiber sequences, so by lemma \ref{lem:omegaspectrafixedpoints}
and theorem \ref{thm:(co)fibersequences}
the bottom row also becomes a long exact sequence after applying homotopy groups.
By induction hypothesis the first two
vertical arrows are equivalences of $G/H$-spectra, so the third one is as well.

Now let $X$ be a general CW-complex, denote by $X_n$ its $n$-skeleton and let  $x_0\in X_0$ be its basepoint.
Then the claim follows for $X_0$ from the isomorphism $T\wedge X_0\cong \bigvee_{x\in X_0\setminus\{x_0\}} T$ and lemma \ref{lem:wedgefixed}. Using the cofiber sequence
\[X_{n-1}\to X_n\to X_n/X_{n-1}\cong\bigvee_{j\in J} S^n\]
and lemma \ref{lem:wedgefixed}, a similar argument as above shows the claim for every
$n$-skeleton $X_n$. For $X$ we consider the following diagram
\[
\begin{tikzcd}
\underset{n}{\colim}\,\pi_q^{K/H}(T^H \wedge X_n)
\arrow{d}{\cong}\arrow{r}{\cong}
& \underset{n}{\colim}\,\pi_q^{K/H}((Q(T \wedge X_n))^H)
\arrow{d}
&\underset{n}{\colim}\,\pi_q^{K}(T \wedge X_n)
\arrow[swap]{l}{\cong}\arrow{d}{\cong}\\
\pi_q^{K/H}(T^H \wedge X)
\arrow{r}
&\pi_q^{K/H}((Q(T \wedge X))^H)
& \pi_q^{K}(T \wedge X)
\arrow[swap]{l}{\cong},
\end{tikzcd}
\]
where the outer vertical arrows are isomorphisms by proposition \ref{prop:colimitshomotopygroups},
the two horizontal arrows on the right are isomorphisms induced by fibrant replacement and lemma
\ref{lem:omegaspectrafixedpoints} and the upper left horizontal arrow is an isomorphism
by the previously treated cases, yielding the claim. 
\end{proof}


\subsection{The smash product of equivariant orthogonal spectra.}
In this subsection we briefly describe the smash product of orthogonal
$G$-spectra. It gives $\mathbf{Sp}^G$ the structure of a symmetric monoidal
category. We are particularly interested in the monoid objects, which
are called \textit{ring spectra}. Our main example of a ring spectrum
is the topological Hochschild spectrum (see proposition \ref{prop:thhringspectrum} for a 
precise statement). The homotopy groups of ring
spectra have the structure of a graded ring, a fact we will repeatedly use.
Our main references for the construction of the smash product are 
\cite[Section~II.2, \pno~33-35]{mandellmay} and \cite[Section~21, \pno~59-60]{mmssdiagramspectra}.
Our construction of the graded ring structure is modeled on the treatment
given in \cite[Chapter~4, \pno~37-40]{schwedeequivariant}.

In the following definition we use that for pointed $G$-spaces
$W,X,Y$ and $Z$ the canonical map
\begin{equation}\label{eq:mappingspacesmash}
\mathrm{smash}:\map(X,Y)\wedge\map(W,Z)\to \map(X\wedge W, Y\wedge Z)
\end{equation}
is continuous, since it is the adjoint of the map
\[
\map(X,Y)\wedge\map(W,Z)\wedge X\wedge W\cong
\map(X,Y)\wedge X\wedge \map(W,Z)\wedge W
\xrightarrow{\mathrm{ev}_X\wedge \mathrm{ev}_W}
Y\wedge Z.
\]
 \begin{mydef}
 The \textit{external smash product} of orthogonal $G$-spectra $T$ and $T'$ is
 the $\mathbf{Top}_{\ast}^G$-enriched functor given on objects by
 \[
T\;\bar{\wedge} \;T' : \mathbf{Th}_G\times  \mathbf{Th}_G\to \mathbf{Top}_{\ast G}, 
(U,V)\mapsto T(U)\wedge T'(V).
 \]
 and on morphisms by
\[
\begin{tikzcd}
\mathbf{Th}(U,V) \wedge \mathbf{Th}(U',V')
\arrow{r}{T\wedge T'}
&\map(T(U),T(V))\wedge \map(T'(U'), T'(V'))
\arrow{d}{\mathrm{smash}}\\
&\map(T(U)\wedge T'(U'), T(V)\wedge T'(V')).
\end{tikzcd}
\]
 \end{mydef}
To define the smash product of $G$-spectra we use
 the $\mathbf{Top}_{\ast}^G$-functor
$\oplus:\mathbf{Th}_G\times \mathbf{Th}_G\to \mathbf{Th}_G$,
which sends finite dimensional
$G$-representations $U$ and $V$ to the direct sum $U\oplus V$
and acts on morphisms by
\[
\mathbf{Th}(U,V)\wedge \mathbf{Th}(U',V')\to \mathbf{Th}(U\oplus U',V\oplus V'), (f, v)\wedge (f',v')\mapsto (f\oplus f', v+v').
\]
\begin{mydef}
The smash product of two orthogonal $G$ spectra $T$ and $T'$ is
the $\mathbf{Top}_\ast^G$-enriched left Kan extension
of $T\;\bar{\wedge}\; T'$ along $\oplus$. It is denoted by $T\wedge T'$.
\end{mydef}

\begin{prop}
The smash product gives $\mathbf{Sp}^G$ the structure of a symmetric monoidal
category. The unit object is the sphere spectrum $\spherespectrum$.
\end{prop}

\begin{proof}
We only prove the isomorphism $\spherespectrum \wedge T\cong T$. The rest
follows formally from the calculus of Kan extensions. We refer to \cite[Theorem~3.3, \pno~20]{day}
and \cite[Theorem~4.1, \pno~26]{day} for details. As part of the data of a
left Kan extension we have a $\mathbf{Top}_{\ast}^G$-enriched natural transformation
\[
\epsilon: \spherespectrum\;\bar{\wedge}\;T\to (\spherespectrum\wedge T)\circ \oplus.
\]
For any finite dimensional $G$-representation $U$ we then obtain the $G$-equivariant map
\begin{equation}\label{eq:smashproductunit1}
T(U)\cong S^0\wedge T(U)\xrightarrow{\epsilon_{0,U}} (\spherespectrum\wedge T)(0\oplus U)
\cong (\spherespectrum\wedge T)(U).
\end{equation}
Conversely, the structure maps \eqref{eq:structuremap} of $T$ give a map
\[
S^V\wedge T(U)\xrightarrow{\sigma_{U, V}} T(U\oplus V),
\]
which by the universal property of the Kan extension yields a map
\begin{equation}\label{eq:smashproductunit2}
u_T:\spherespectrum \wedge T\to T.
\end{equation}
One checks that \eqref{eq:smashproductunit1} is a map of $G$-spectra and using
the universal property of Kan extensions one sees that it is inverse to
\eqref{eq:smashproductunit2}. 
\end{proof}


\begin{mydef}
A \textit{(commutative) ring} $G$-spectrum is a (commutative) monoid object in
the symmetric monoidal category $(\mathbf{Sp}^G, \wedge, \spherespectrum)$.
\end{mydef}
In general, it is difficult to check if a $G$-spectrum is a ring spectrum using the above
definition. We indicate a more direct way. For the associativity, we note that
by the calculus of Kan extensions the triple smash product $T\wedge T'\wedge T''$
can also be computed as the left Kan extension of the $\mathbf{Top}_\ast^G$-functor
\[
T\;\bar{\wedge}\; T'\;\bar{\wedge}\; T'':\mathbf{Th}_G\times \mathbf{Th}_G\times \mathbf{Th}_G\to \mathbf{Top}_{\ast G}
\]
along the $\mathbf{Top}_\ast^G$-functor
\[
-\oplus-\oplus-: \mathbf{Th}_G\times \mathbf{Th}_G\times \mathbf{Th}_G\to \mathbf{Th}_G.
\]
Then, if we have a multiplication map $\mu:T\wedge T\to T$ induced by a $\mathbf{Top}_\ast^G$-enriched
natural transformation $\bar \mu:T\;\bar{\wedge}\; T\to T\circ \oplus$, the associativity
is equivalent to the commutativity of the following diagram:
\begin{equation}
\begin{tikzcd}[column sep = large]
T(U)\wedge T(V)\wedge T(W)
\arrow{d}{\bar \mu_{U,V}\wedge \mathrm{id}}\arrow{r}{\mathrm{id}\wedge \bar\mu_{V,W}}
&T(U)\wedge T(V\oplus W)
\arrow{d}{\bar\mu_{U,V\oplus W}}\\
T(U\oplus V)\wedge T(W)
\arrow{r}{\bar\mu_{U\oplus V,W}}
&T(U\oplus V\oplus W).
\end{tikzcd}
\end{equation}
Similarly, given a unit map $u: \spherespectrum\to T$ the unit law is equivalent to the condition that the composition
\begin{equation}\label{eq:ringspectrumunit}
T(U)\cong S^0\wedge T(U)\xrightarrow{u_0\wedge\mathrm{id}} T(0)\wedge T(U)\xrightarrow{\bar\mu_{0,U}} T(U)
\end{equation}
is the identity. Finally, the commutativity of the $\mu$ is equivalent to the commutativity of:
\begin{equation}
\begin{tikzcd}
T(U)\wedge T(V)
\arrow{d}{\mathrm{twist}}\arrow{r}{\bar\mu_{U,V}}
&T(U\oplus V)
\arrow{d}{T(\mathrm{twist})}\\
T(V)\wedge T(U)
\arrow{r}{\bar\mu_{V,U}}
&T(V\oplus U).
\end{tikzcd}
\end{equation}
This can again be seen by invoking the formal properties
of Kan extensions and using \eqref{eq:smashproductunit1}. 
We will use the remarks above in the proof that the 
topological Hochschild spectrum is a ring spectrum.




We now proceed to construct a natural pairing
\begin{equation}\label{eq:homotopygroupspairing}
\beta:\pi_m^H(T)\otimes \pi_n^H(T')\to \pi^H_{m+n}(T\wedge T')
\end{equation}
for any subgroup $H$. Given elements $x\in \pi_m^H(T)$
and $y\in \pi_n(T')$, we choose representatives
$f:S^m\to (\Omega^U T(U))^H$ and $g:S^n\to (\Omega^{U'}T'(U'))^H$.
We have the canonical map
\[
\mathrm{smash}: (\Omega^U T(U))^H\wedge (\Omega^{U'} T'(U'))^H\to
(\Omega^{U\oplus U'}(T(U)\wedge T'(U')))^H
\]
and the natural transformation $\epsilon:T\;\bar\wedge\; T'\to (T\wedge T')\circ \oplus$.
We define $\beta(x,y)\in \pi_{m+n}^H(T\wedge T')$ as the class of the map
\[
\begin{tikzcd}[column sep = huge]
S^{m+n}\arrow{r}{\cong}
&S^m\wedge S^n
\arrow{r}{f\wedge g}
&(\Omega^U T(U))^H\wedge (\Omega^{U'} T'(U'))^H
\arrow{d}{\mathrm{smash}}\\
&(\Omega^{U\oplus U'} ((T\wedge T')(U\oplus U')))^H
&\Omega^{U\oplus U'}(T(U)\wedge T'(U'))^H
\arrow[swap]{l}{\Omega^{U\oplus U'}\epsilon_{U, U'}}
\end{tikzcd}
\]
We need to show this is well defined. It is clearly independent of the homotopy
class of $f$ and $g$, so we only need to show it is independent of
the choice of representatives in the colimit in \eqref{eq:equivarianthomotopygroup}.
To see this, let $V$ and $V'$ be $G$-representations containing $U$ and $U'$
respectively and note the following diagram commutes by naturality of $\epsilon$:
\[
\begin{tikzcd}[column sep=13em]
(\Omega^U T(U))^H\wedge (\Omega^{U'} T'(U'))^H
\arrow{d}{\mathrm{smash}}\arrow{r}{(\Omega^{V-U}\tilde \sigma_{U\subseteq V})^H\wedge(\Omega^{V'-U'}\tilde \sigma'_{U'\subset V'})^H}
&(\Omega^V T(V))^H\wedge (\Omega^{V'} T'(V'))^H
\arrow{d}{\mathrm{smash}}\\
(\Omega^{U\oplus U'}(T(U)\wedge T'(U')))^H
\arrow{d}{(\Omega^{U\oplus U'}\epsilon_{U,U'})^H}
&(\Omega^{V\oplus V'}(T(V)\wedge T'(V')))^H
\arrow{d}{(\Omega^{V\oplus V'}\epsilon_{V,V'})^H}\\
(\Omega^{U\oplus U'} ((T\wedge T')(U\oplus U')))^H
\arrow{r}{(\Omega^{V\oplus V'-U\oplus U'}\sigma^{T\wedge T'}_{U\oplus U'\subseteq V\oplus V'})^H}
&(\Omega^{V\oplus V'} ((T\wedge T')(V\oplus V')))^H.
\end{tikzcd}
\]
Using the definition we see that the assignment $\beta(x,y)$ is additive in each
variable, so that \eqref{eq:homotopygroupspairing} really gives a pairing.
Moreover, if $T$ is a ring spectrum, we can compose with the map induced
by the multiplication:
\begin{equation}\label{eq:homotopygroupmultiplication}
\cdot:\pi_m^H(T)\otimes \pi_n^H(T)\xrightarrow{\beta}\pi_{m+n}^H(T\wedge T)
\xrightarrow{\mu_\ast} \pi_{m+n}^H(T).
\end{equation}
\begin{prop}
Let $T$ be a ring $G$-spectrum and $H$ a subgroup.
\begin{enumerate}[(i)]
\item The homotopy groups $\pi_\ast^H(T)$
form a graded ring with multiplication \eqref{eq:homotopygroupmultiplication}.
\item If $T$ is additionally commutative, $\pi_\ast^H(T)$ is graded anti-commutative,
i.e. $x\cdot y = (-1)^{mn}y\cdot x$ for $x\in \pi_m^H(T)$ and $y\in \pi_n^H(T)$.
\end{enumerate}
\end{prop}

\begin{proof}
(i) Since $\mu$ and $\beta$ are associative, the multiplication on $\pi_\ast^H(T)$ is as well. Since \eqref{eq:homotopygroupspairing}
is additive, we obtain distributivity. It remains to show the existence of a unit. By the Freudenthal suspension
theorem we have $\pi_0^{\{1\}}(\spherespectrum)\cong \ZZ$ and the class of the map
$S^0\to \Omega^{\RR^1}\spherespectrum(\RR^1)$, which sends the non-basepoint to the
identity is a generator. Since $G$ acts trivially on $\Omega^{\RR^1}\spherespectrum(\RR^1)$,
this map is also an element of $\pi_0^H(\spherespectrum)$, so we have a canonical map
$\pi_0^{\{1\}}(\spherespectrum)\to \pi_0^H(\spherespectrum)$. 
Using \eqref{eq:ringspectrumunit} and the definition of the multiplication
we see that
\[
\pi_q^H(T)\cong \pi_q^H(T)\otimes \pi_0^{\{1\}}(\spherespectrum)
\to  \pi_q^H(T)\otimes \pi_0^H(\spherespectrum)
\xrightarrow{\mathrm{id}\otimes (u_T)_\ast} 
\pi_q^H(T)\otimes \pi_q^H(T) \xrightarrow{\cdot}
\pi_q^H(T)
\]
is the identity, which shows that $\pi_0^{\{1\}}(\spherespectrum)\to \pi_0^H(\spherespectrum)
\xrightarrow{(u_T)_\ast} \pi_0^H(T)$ is a unit map. 

(ii) We consider the map
\[
\tau_{m,n}: S^{m+n}\cong S^m\wedge S^n\xrightarrow{\mathrm{twist}}
S^n\wedge S^m\cong S^{n+m}.
\]
Using the definition of $\beta$ we see the left part of the following diagram commutes
(compare \cite[Proposition~4.29, \pno~39]{schwedeequivariant}) and the right
part commutes by the commutativity of $T$ and naturality of $\tau_{m,n}$:
\[
\begin{tikzcd}
\pi_m^H(T)\otimes \pi_n^H(T)
\arrow{dd}{\mathrm{twist}}
\arrow{r}{\beta}
&\pi_{m+n}^H(T\wedge T)
\arrow[swap]{d}{(\mathrm{twist})_\ast}\arrow{dr}{\mu_\ast}\\
&\pi_{m+n}^H(T\wedge T)
\arrow[swap]{d}{\tau_{m,n}^\ast}\arrow{r}{\mu_\ast}
&\pi^H_{m+n}(T)
\arrow{d}{\tau_{m,n}^\ast}\\
\pi_n^H(T)\otimes \pi_m^H(T)
\arrow{r}{\beta}
&\pi_{m+n}^H(T\wedge T)\arrow{r}{\mu_\ast}
&\pi_{m+n}^H(T)
\end{tikzcd}
\]
Now use that
 $\tau_{m,n}^\ast$ induces multiplication
by $(-1)^{mn}$ on homotopy groups.
\end{proof}


\begin{lem}\label{lem:pairing}
Let $X$ be a wedge of $n$-spheres and $T$ an orthogonal $G$-spectrum. 
Then there is a natural isomorphism of abelian groups
\begin{equation}\label{eq:pairingspheres}
\pi_{n}(X)\otimes \pi_q^{\{1\}}(T)\xrightarrow{\cong} \pi_{ q + n}^{\{1\}}(T\wedge X).
\end{equation}
\end{lem}
\begin{proof}
We define \eqref{eq:pairingspheres} in the same way as \eqref{eq:homotopygroupspairing}.
By lemma \ref{lem:spectrumwedge} we only need to treat the case $X = S^n$
to show it is an isomorphism. However, by checking the definitions we find
that 
\[
\pi_q^{\{1\}}(T)\cong \pi_n(S^n)\otimes \pi_q^{\{1\}}(T)\to \pi_{q+n}^{\{1\}} (T\wedge S^n) =  \pi_{q+n}^{\{1\}} (\Sigma^n T)
\]
is exactly the isomorphism from lemma \ref{lem:spectrumsuspensionshift} (ii).
\end{proof}
\subsection{Monoidal properties of the restriction and fixed point functors.}
The restriction and fixed point functors are lax monoidal
with respect to the smash product of orthogonal spectra.
This implies that they send ring spectra to ring spectra
and preserve maps of ring spectra. We start with fixed points.
Let $H$ be a normal subgroup of $G$. For $G$-spectra
$T$ and $T'$ we have the
$\mathbf{Top}_{\ast}^G$-enriched natural transformation
\[
\epsilon:T\,\overline{\wedge}\, T'\to (T\wedge T')\circ\oplus.
\]
If we plug in $G$-representations $U$ and $V$ one which
$H$ acts trivially and apply $H$-fixed points we obtain
a $\mathbf{Top}_{\ast}^{G/H}$-enriched natural transformation
\[
\epsilon^{G/H}_{U,V}:T(U)^H\wedge T'(V)^H\to (T\wedge T')(U\oplus V)^H, 
\]
which induces a map of $G/H$-spectra $T^H\wedge T'^H\to (T\wedge T')^H$.
Moreover, it follows directly from the definitions that the $H$-fixed points
of the $G$-sphere spectrum give the $G/H$-sphere spectrum and one can 
verify that this data satisfies the associativity and unitality axioms
of a lax monoidal functor.
The case of smashing a ring spectrum with a space is even easier,
since our construction directly implies $T^H\wedge X^H = (T\wedge X)^H$
for any $G$-spectrum $T$ and any $G$-space $X$.

The restriction functors are more complicated. We first remark that
if we replace $\mathbf{Th}_G$ with $\mathbf{Th}_G^{\mathrm{triv}}$
in the construction of the smash product of $G$-spectra, we obtain
a smash product of naive $G$-spectra and this gives $\mathbf{NSp}^G$
the structure of a symmetric monoidal category. By \cite[Proposition~A.19, \pno~147]{hillhopkinsravenel},
the equivalences of theorem \ref{thm:naiveisnotnaive} are strong monoidal.
Now let $\alpha:\Gamma\to G$ be a group homomorphism,
$T$ and $T'$ two $G$-spectra and let $U$ and $V$ be finite dimensional
trivial $\Gamma$-representations. We have the $\mathbf{Top}_\ast^G$-enriched
natural transformation
\[
\epsilon:T\,\overline{\wedge}\, T'\circ \oplus\to T\wedge T',
\]
and applying $\alpha^\ast$ at the space level we get a map
\[
\alpha^\ast T(U)\wedge \alpha^\ast T'(V) = \alpha^\ast (T(U)\wedge T'(V))
\xrightarrow{\alpha^\ast \epsilon_{U,V}} \alpha^\ast (T\wedge T')(U\oplus V),
\]
which is natural in $U$ and $V$, giving rise to a map of naive $\Gamma$-spectra
\[
\alpha^\ast T\wedge \alpha^\ast T'\to \alpha^\ast (T\wedge T').
\]
Finally, by theorem \ref{thm:naiveisnotnaive} and the remark above
we obtain a map of $\Gamma$-spectra
\[
\alpha^\ast T\wedge \alpha^\ast T'\to \alpha^\ast (T\wedge T')
\]
satisfying the associativity and unitality axioms
of a lax monoidal functor, since the maps of the underlying naive spectra do.
For the smash product of a $G$-spectrum $T$ with a $G$-space $X$
we have the equality $\alpha^\ast (T\wedge X) = \alpha^\ast T\wedge \alpha^\ast X$,
since by \cite[Construction~6.5 \pno~1519]{rvadams}
this holds for the underlying naive $\Gamma$-spectra, so
by \cite[Construction~6.7, \pno~1519]{rvadams} it also holds for 
$\alpha^\ast (T\wedge X)$ and $\alpha^\ast T\wedge \alpha^\ast X$.
\subsection{Restriction and transfer.}
The Frobenius and Verschiebung maps are defined as certain restriction and transfer
maps between homotopy groups. We briefly describe their construction and some of their properties.

Suppose we have two subgroups $K\subset H$ of $G$. For any $G$-spectrum $T$
and any finite dimensional $G$-representation $U$ we have the inclusion of the fixed
points $(\Omega^UT(U))^H\hookrightarrow (\Omega^UT(U))^K$. This induces a map
on homotopy groups and this is compatible with the structure map of the colimit in 
\eqref{eq:equivarianthomotopygroup}. This gives us the \textit{restriction map}
\begin{equation}
\mathrm{res}^H_K: \pi_q^H(T)\to \pi_q^K(T).
\end{equation}
We immediately see this is natural in $T$. Moreover, the restriction map commutes
with the natural pairing $\beta$ from \eqref{eq:homotopygroupspairing}. It follows
that the restriction is a multiplicative map if $T$ is a ring spectrum.

The construction of the transfer map for general $G$-spectra is more involved.
We restrict our discussion to the case where $G$ is a finite group
and follow the treatment in \cite[Chapter~4]{schwedeequivariant}.
This suffices for our purposes, since we are interested in transfer maps of
the form
\[
\mathrm{tr}^{C_a}_{C_{ar}}:\pi_q^{C_a}(T)\to \pi_q^{C_{ar}}(T)
\]
for a $\TT$-spectrum $T$ (compare section \ref{sec:wittcomplexes}) and 
lemma \ref{lem:homotopygroupsrestriction} (ii) tells us we can work
with $\mathrm{res}^\TT_{C_{ar}} T$ instead of $T$.

We consider a finite group $G$ and a subgroup $H$. Suppose
$T$ is an $H$-spectrum. For the underlying naive spectrum
$\iota^\ast T$ and a trivial finite dimensional $H$-representation
$U$ we have the equality
\begin{equation}\label{eq:naivespectruminduction}
(\mathrm{ind}^G_H \iota^\ast T)(U) = T(U)\underset{H}{\wedge} G_+,
\end{equation}
which is a direct consequence of \cite[Construction~6.5, \pno~1519]{rvadams}.
We then have the $H$-equivariant map of naive spectra
\[
\mathrm{pr}:(\mathrm{res}^G_H\, \mathrm{ind}^G_H \iota^\ast T)(U)\to 
 \iota^\ast T(U), [x\wedge g]\mapsto
\begin{cases}
gx &\text{ if } g\in H,\\
\ast &\text{ if } g\not \in H.
\end{cases}
\]
This induces a map of $H$-spectra
\[
\mathrm{pr}:\mathrm{res}^G_H\, \mathrm{ind}^G_H  T \to T
\]
and we have the following proposition from \cite[Proposition~4.11, \pno~30]{schwedeequivariant},
which is a consequence of the Wirthm\"uller isomorphism \cite[Theorem~4.9, \pno~28]{schwedeequivariant}.
\begin{prop}
The map
\begin{equation}\label{eq:externaltransfer}
\pi_q^G(\mathrm{ind}^G_H T)\xrightarrow{\mathrm{res}^G_H\,} \pi_q^H(\mathrm{ind}^G_H T)
\cong \pi_q^H(\mathrm{res}^G_H\, \mathrm{ind}^G_H T)\xrightarrow{\mathrm{pr}_\ast}
\pi_q^H(T),
\end{equation}
where the isormorphism is given by lemma  \ref{lem:homotopygroupsrestriction} (ii),
is an isomorphism for all $q\in \ZZ$.
\end{prop}
If $T$ is a $G$-spectrum, then \eqref{eq:naivespectruminduction}  holds
for $\mathrm{res}^G_H\, T$ instead of $T$ and the map
\[
\mathrm{action}: (\mathrm{ind}^G_H \mathrm{res}^G_H\, \iota^\ast T)(U)
\to \iota^\ast T(U), [x\wedge g]\mapsto gx
\]
is $G$-equivariant, hence defines a map of $G$-spectra
\[
\mathrm{action}:\mathrm{ind}^G_H \mathrm{res}^G_H\,  T\to T.
\]
\begin{mydef}
\begin{enumerate}[(i)]
\item For an $H$-spectrum $T$, the \textit{external transfer isomorphism} 
\[
\mathrm{Tr}^G_H: \pi_q^H(T)\to \pi_q^G(\mathrm{ind}^G_H T)
\]
is defined to be the inverse of \eqref{eq:externaltransfer}.
\item Now assume $T$ is a $G$-spectrum. 
The \textit{internal transfer map}
\[
\mathrm{tr}^G_H: \pi_q^H(T)\to \pi_q^G( T)
\]
is defined as the composite
\[
 \pi_q^H(T)\xrightarrow{\mathrm{Tr}^G_H} \pi_q^G(\mathrm{ind}^G_H\,\mathrm{res}^G_H\, T)
 \xrightarrow{\mathrm{action}_\ast} \pi_q^G(T).
\]
\end{enumerate}
\end{mydef}
This ends the construction of the transfer map and it is immediate from the
definition that the transfer is natural in $T$. The transfer has the following transitivity
property.
\begin{prop}\label{prop:transfertransitivity}
Let $G$ be a finite group and $K\subset H$ subgroups of $G$.
Then 
\[
\mathrm{tr}^G_H\circ \mathrm{tr}^H_K = \mathrm{tr}^G_K:
\pi_\ast^K(T)\to \pi_\ast^G(T)
\]
for any $G$-spectrum $T$.
\end{prop}
\begin{proof}
See \cite[Proposition~4.16, \pno~31]{schwedeequivariant}
\end{proof}

We also have the following statement, which records how transfer and restriction
interact. It is a special case of the double coset formula \cite[Proposition~4.20, \pno~35]{schwedeequivariant}.
\begin{prop}\label{prop:doublecoset}
Let $G$ be a finite commutative group. Then for any $G$-spectrum $T$ we have the equality
\[
\mathrm{res}^G_1\circ \mathrm{tr}^G_1 = \sum_{g\in G} \ell_{g\ast}:
\pi_\ast^{\{1\}}(T)\to \pi_\ast^{\{1\}}(T),
\]
where $\ell_g:T\to T$ denotes left translation with $g$.
\end{prop}


We conclude this subsection
with three technical lemma's, which we need later.
\begin{lem}\label{lem:suspensiontransfer}
For any $G$-spectrum $T$, any integer $q\in \ZZ$  and any subgroup $H$, the following diagram
\[
\begin{tikzcd}
\pi_q^H(T)
\arrow{r}{\cong}[swap]{\Sigma}\arrow{d}{\mathrm{tr}^G_H}
&\pi_{q+1}^H(T)
\arrow{d}{\mathrm{tr}^G_H}\\
\pi_q^G(T)
\arrow{r}{\cong}[swap]{\Sigma}
&\pi_{q+1}^G(T),
\end{tikzcd}
\]
where $\Sigma$ denotes the suspension isomorphism from lemma
\ref{lem:spectrumsuspensionshift}, commutes.
\end{lem}
\begin{proof}
Using remark \ref{rem:spectrumsuspensionshift} we find by
direct inspection that the following diagram commutes:
\[
\begin{tikzcd}[column sep = 1.7em]
\pi_q^G(\mathrm{ind}^G_H \mathrm{res}^G_H\, T)
\arrow{r}{\mathrm{res}^G_H\,} \arrow{d}[swap]{\cong}{\Sigma}
&\pi_q^H(\mathrm{ind}^G_H \mathrm{res}^G_H\, T)
\arrow{d}[swap]{\cong}{\Sigma}
&\pi_q^H(\mathrm{res}^G_H\, \mathrm{ind}^G_H \mathrm{res}^G_H\, T)
\arrow{r}{\mathrm{pr}_\ast}\arrow{l}[swap]{\cong}
\arrow{d}[swap]{\cong}{\Sigma}
&\pi_q^H(\mathrm{res}^G_H\, T)
\arrow{d}[swap]{\cong}{\Sigma}\\
\pi_{q+1}^G(\mathrm{ind}^G_H \mathrm{res}^G_H\, T)
\arrow{r}{\mathrm{res}^G_H\,} 
&\pi_{q+1}^H(\mathrm{ind}^G_H \mathrm{res}^G_H\, T)
&\pi_{q+1}^H(\mathrm{res}^G_H\, \mathrm{ind}^G_H \mathrm{res}^G_H\, T)
\arrow{r}{\mathrm{pr}_\ast}\arrow{l}[swap]{\cong}
&\pi_{q+1}^H(\mathrm{res}^G_H\, T).
\end{tikzcd}
\]
This implies $\Sigma$ commutes with the external transfer. By naturality
of $\Sigma$
the following diagram commutes:
\[
\begin{tikzcd}[column sep = large]
\pi_q^G(\mathrm{ind}^G_H\,\mathrm{res}^G_H\, T)
\arrow{r}{\mathrm{action}_\ast} \arrow{d}[swap]{\cong}{\Sigma}
&\pi_q^G(T)
\arrow{d}[swap]{\cong}{\Sigma}\\
\pi_{q+1}^G(\Sigma\mathrm{ind}^G_H\,\mathrm{res}^G_H\, T)
\arrow{r}{\mathrm{action}_\ast}
&\pi_{q+1}^G(\Sigma T).
\end{tikzcd}
\]
The claim follows.
\end{proof}
\begin{lem}\label{lem:switchfixedpoints}
Let $C\subset \TT$ be a finite subgroup, $T$ a $C$-spectrum
and $X = \bigvee_{j\in J} S^n\wedge C_+$.
%Let $\iota:C\to \TT$ denote the inclusion.
The $C$-action on $ T\wedge X$ induces
a $C$-action on the homotopy groups of $T\wedge X$.
%Suppose the $C$-action on $A$ is free away from
%the basepoint.
The map $\mathrm{res}^C_1:\pi_q^C( T\wedge X)\to 
\pi_q^{\{1\}}(T\wedge X)$ restricts to an isomorphism
\[\pi_q^C( T\wedge X)\xrightarrow{\cong} 
\left(\pi_q^{\{1\}}(  T\wedge X)\right)^C\]
\end{lem}
\begin{rem}
In \cite[\pno~22]{hmmixed} it is claimed the above statement
holds for arbitrary CW-complexes with a free $C$-action. We
believe this is false in general.
\end{rem}
\begin{proof}
We note that the map from the statement can equivalently be described
as the map induced on homotopy groups by the inclusion of the
fixed points
\[
Q(T\wedge X)^C\hookrightarrow Q(T\wedge X).
\]
Since the left hand side does not carry a $C$-action, we see
that the map from the statement really lands in
$\left(\pi_q^{\{1\}}(  T\wedge X)\right)^C$.
%We first show that the map of the statement lands in 
%$\left(\pi_q^{\{1\}}(  T\wedge X)\right)^C$. This map
%is $C$-equivariant, so it suffices to show that $C$
%acts trivially on $\pi_q^C( T\wedge X)$.
%We fix a cofibrant replacement functor $\Gamma$ for
%the stable model structure.
%%Since $G$ acts freely on $X$,
%%the projection
%%\[\left(\Gamma T\wedge X\right)\wedge EG_+\to \Gamma T\wedge X\]
%%is a level equivalence. 
%By \cite[Lemma III.4.14, \pno~50]{mandellmay} $\Gamma T\wedge X$
%is cofibrant and by \cite[Theorem III.3.11, \pno~47]{mandellmay} the map
%$\gamma_T\wedge\mathrm{id_X}: \Gamma T\wedge X
%\to T\wedge X$ is an equivalence of $C$-spectra.
%Finally, \cite[Lemma~5.10, \pno~1514]{rvadams} implies
%that $\Gamma T\wedge X$ is good.
%%so by \cite[Theorem~6.16, \pno~1521]{rvadams} $ \Gamma T$
%%is good and 
%%$\gamma_T:
%% \Gamma T\to  T$ is an equivalence of $C$-spectra.
%%Applying \cite[Lemma~5.10, \pno~1514]{rvadams} again yields that
%%also $ \Gamma T\wedge A$ is good and 
%%$\gamma_T\wedge\mathrm{id}:
%% \Gamma T\wedge A\to  T\wedge A$ is an equivalence of
%%$C$-spectra.
%
%Let $\mathcal{U}$ be a complete $C$-universe.
%We use the bifunctorial
%replacement functor $Q$ of \cite[Theorem~1.1, \pno~1494]{rvadams}. Our assumptions 
%imply that the projection 
%\[ \Gamma T\wedge X\wedge EC_+\to  \Gamma T\wedge X\]
% is a level equivalence of $C$-spectra, so by \cite[Main Theorem~1.7, \pno~1496]{rvadams} 
%there is a natural equivalence of non-equivariant spectra
%\[A:\left(\Gamma T\wedge X\right)\underset{C}{\wedge} EC_+\xrightarrow{\sim} Q^{\mathcal{U}}(\Gamma T\wedge X)^C.\]
%%Moreover, by the proof of \cite[Corollary~1.8, \pno~5]{rvadams} the projection
%%\[p:\left(\Gamma T\wedge X\right)\underset{G}{\wedge} EG_+\xrightarrow{\sim}
%%\left(\Gamma T\wedge X\right)/G\]
%%is an equivalence of nonequivariant spectra. 
%We obtain a zig-zag of isomorphisms
%\[
%\begin{tikzcd}
%\pi_q^C( T\wedge X)
%&\pi_q^C( \Gamma T\wedge X)\arrow{l}[swap]{(\gamma_{ T}\wedge\mathrm{id}_X)_\ast}
%\arrow{r}{(r_{ \Gamma T\wedge A})_\ast}
%&\pi_q^C\left(  Q^{\mathcal{U}}( \Gamma T\wedge X) \right)
%\\
%&\pi_q^{\{1\}}\left(  \left( \Gamma T\wedge X\right)\underset{C}{\wedge} EC_+\right)
%\arrow{r}{A_\ast}
%&\pi_q^{\{1\}}\left(   Q^{\mathcal{U}}(\Gamma T\wedge X)^C \right)\arrow{u}{\cong},
%\end{tikzcd}
%\]
%which are natural in both $T$ and $X$.
%%The upper left arrow is an isomorphism by \cite[Lemma~5.13, \pno~20]{rvadams},
%%because $\gamma_T:T\to \Gamma T$ is an equivalence of $G$-spectra. 
%That the
%upper right arrow and the vertical arrow
%are isomorphisms follows from \cite[Theorem~1.1, \pno~1494]{rvadams}.
%Because of the naturality it suffices to show the $C$-action on 
%$\pi_q^{\{1\}}\left(  \left( \Gamma T\wedge X\right)\underset{C}{\wedge} EC_+\right)$
%is trivial. For any $g\in C$, the action of $g$ is induced by the map
%\begin{equation}\label{eq:universalspaceaction}
%\mathrm{id}_{\Gamma T\wedge X}\wedge \ell_g: 
%\left( \Gamma T\wedge X\right)\underset{C}{\wedge} EC_+ 
%\to  \left( \Gamma T\wedge X\right)\underset{C}{\wedge} EC_+,
%\end{equation}
%where
%\[\ell_g:EC\to EC\]
%denotes left multiplication by $g$. 
%We show that \eqref{eq:universalspaceaction} is (non-equivariantly) homotopic to the identity
%for a suitable model of $EC$. Let $\mathcal{F}$ be the family
%of all subgroups $G$ of $\TT$ such that $G\cap C = 1$
%and let $E\mathcal{F}$ be the universal $\TT$-space
%for $\mathcal{F}$. If $\iota:C\to \TT$ denotes the inclusion,
%then $\iota^\ast E\mathcal{F} = EC$ and left multiplication
%by elements of $\TT$ makes sense.
%We choose $\varphi\in I$ such that $\mathrm{exp}(2\pi i \varphi) = g$
%and define
%\[h:EC_+\wedge I_+\to EC_+, x\wedge t\mapsto \mathrm{exp}(2\pi i (1-t)\varphi)\cdot x.\]
%After smashing with $\Gamma T\wedge X$ we obtain a 
%well-defined homotopy
%\[H: \left(\left( \Gamma T\wedge X\right)\underset{C}{\wedge} EC_+\right) \wedge I_+
%\to  \left( \Gamma T\wedge X\right)\underset{C}{\wedge} EC_+\]
%from \eqref{eq:universalspaceaction} to the identity as required.
%
%We now show the map from the statement is an isomorphism.

We first treat the case $X = C_+$. For any $C$-space $Y$
there is a natural $C$-equivariant homeomorphism
\begin{equation}\label{eq:untwist}
Y\wedge C_+\xrightarrow{\cong} \mathrm{res}^C_1 Y \wedge C_+ = 
\mathrm{ind}^C_1 \mathrm{res}^C_1 Y, y\wedge g\mapsto g^{-1}y\wedge g
\end{equation}
and by theorem \ref{thm:naiveisnotnaive} this gives rise to the natural isomorphism of $C$-spectra
\[T\wedge C_+\cong \mathrm{ind}^C_1 \mathrm{res}^C_1 T,\]
hence there is a commutative diagram of $C$-equivariant maps
\begin{equation}\label{eq:confusingdiagram}
\begin{tikzcd}
\pi_q^C(T\wedge C_+)\arrow{d}{\cong}\arrow{r}{\mathrm{res}^C_1}
&\pi_q^{\{1\}}(T\wedge C_+)\arrow{d}{\cong}\\
\pi_q^C(\mathrm{ind}^C_1 \mathrm{res}^C_1 T)\arrow{r}{\mathrm{res}^C_1}
&\pi_q^{\{1\}}(\mathrm{ind}^C_1 \mathrm{res}^C_1 T).
\end{tikzcd}
\end{equation}
%Now, we have $\iota^\ast \mathrm{ind}^C_1 \mathrm{res}^C_1 T = (\mathrm{res}^C_1\,\iota^\ast T)\wedge C_+$
%and the vertical arrow on the right becomes $C$-equivariant if we give
%$\pi_q^{\{1\}}( \mathrm{ind}^C_1 \mathrm{res}^C_1 T)$
%the $C$-action
%induced by
%\[
%\mathrm{id}_{\mathrm{res}^C_1\,\iota^\ast T}\wedge \ell_g:(\mathrm{res}^C_1\,\iota^\ast T)\wedge C_+
%\to(\mathrm{res}^C_1\, \iota^\ast T)\wedge C_+,
%\]
%where $\ell_g:C\to C$ denotes  left multiplication with $g\in C$. 
We show that 
the lower horizontal map in \eqref{eq:confusingdiagram} induces an isomorphism
\begin{equation}\label{eq:restriction}
\mathrm{res}^C_1:\pi_q^C(\mathrm{ind}^C_1 \mathrm{res}^C_1 T)
\xrightarrow{\cong} \left(\pi_q^{\{1\}}( \mathrm{ind}^C_1 \mathrm{res}^C_1 T)\right)^C.
\end{equation}
% This is enough, since the vertical maps and the upper right map in \eqref{eq:confusingdiagram}
%are $C$-equivariant. 
Non-equivariantly, we have the isomorphism 
\[
\mathrm{res}^C_1 \mathrm{ind}^C_1 \mathrm{res}^C_1 T\cong \bigvee_{g\in C} \mathrm{res}^C_1 T,\]
hence also the $C$-equivariant isomorphism
\[
\pi_q^{\{1\}}( \mathrm{ind}^C_1 \mathrm{res}^C_1 T)\cong \bigoplus_{g\in C} \pi_q^{\{1\}}( T),
\]
where $C$ acts on the right hand side by permuting the summands. This follows from lemma
\ref{lem:homotopygroupsrestriction}.
The fixed points on the right hand side are the diagonal, so the projection
$\mathrm{pr}: \mathrm{ind}^C_1 \mathrm{res}^C_1 T \to T$ induces an isomorphism
\[
\mathrm{pr}_\ast:\left(\pi^{\{1\}}_q( \mathrm{ind}^C_1 \mathrm{res}^C_1 T)\right)^C\xrightarrow{\cong}
\pi_q^{\{1\}}( T).
\]
Finally, we have the isomorphism  \eqref{eq:externaltransfer}
\[
\begin{tikzcd}
\pi_q^C(\mathrm{ind}^C_1 \mathrm{res}^C_1 T)
\arrow{r}{\mathrm{res}^C_1}
&\left( \pi_q^{\{1\}}( \mathrm{ind}^C_1 \mathrm{res}^C_1 T)\right)^C
\arrow{r}[swap]{\cong}{\mathrm{pr}_\ast} 
&\pi_q^{\{1\}}( T)
\end{tikzcd}
\]
and the claim follows, since the first map is the map from the statement.

The case $X= S^n\wedge C_+$ follows directly, since $T\wedge S^n\wedge C_+ = 
\Sigma^n T\wedge C_+$. The case $X = \bigvee_{j\in J} S^n\wedge C_+$ follows
from the commutativity of the following diagram
\[
\begin{tikzcd}
\bigoplus_{j\in J} \pi_q^C(T\wedge S^n\wedge C_+)
\arrow{rr}{\cong}\arrow{d}{\cong}
&&\pi_q^C(T\wedge X)
\arrow{d}\\
\bigoplus_{j\in J} \left(\pi_q^{\{1\}}(T\wedge S^n\wedge C_+)\right)^C
\arrow{r}{\cong}
&\left(\bigoplus_{j\in J} \pi_q^{\{1\}}(T\wedge S^n\wedge C_+)\right)^C
\arrow{r}{\cong}
&\left(\pi_q^{\{1\}}(T\wedge X)\right)^C,
\end{tikzcd}
\]
where the lower left isomorphism comes from the fact that the $C$-action
respects the sum decomposition and the other isomorphisms come from
lemma \ref{lem:spectrumwedge}.
%We now treat the general case. Since the $C$-action
%is free, we have 
%$A_0 \cong \bigvee_{j\in J} C_+$ and $A_n/A_{n-1}\cong \bigvee_{j\in J'} S^n\wedge C_+$,
%so using the cofiber sequence
%\[A_{n-1}\to A_n\to A_n/A_{n-1} \]
%the claim follows for $A_n$ by induction.
%For the limit case we consider the following diagram:
%\[
%\begin{tikzcd}
%\underset{n}{\colim}\, \pi_q^C(T\wedge A_n)
%\arrow{r}{\cong}\arrow{d}{\cong}
%& \pi_q^C(T\wedge A)
%\arrow{dd}\\
%\underset{n}{\colim}\, \left( \pi_q^{\{1\}}(T\wedge A_n)\right)^C
%\arrow{d}{\cong}\\
%\left(\underset{n}{\colim}\, \pi_q^{\{1\}}(T\wedge A_n) \right)^C
%\arrow{r}{\cong}
%&\left(\pi_q^{\{1\}}(T\wedge A).\right)^C
%\end{tikzcd}
%\]
%The left upper vertical arrow is an isomorphism by the previous cases
%and the horizontal arrows are isomorphisms by proposition \ref{prop:colimitshomotopygroups}.
%The lower left vertical arrow is an isomorphism by the following general
%fact. If
%\[
%M_0\to M_1\to\ldots\to M_n\to \ldots
%\]
%is a sequence of $C$-modules with $C$-equivariant maps,
%then the canonical map is an isomorphism:
%\[
%\underset{n}{\colim}\, M_n^C\xrightarrow{\cong} (\underset{n}{\colim} \,M_n)^C.
%\]
%\[\pi_q\left( \left(\Gamma T\wedge X\right)/G \right) \xleftarrow{\cong}
%\pi_q\left(  \left(\Gamma T\wedge X\right)\underset{G}{\wedge} EG_+\right)
%\xrightarrow{\cong} \pi_q\left(   \mathcal{Q}^{\mathcal{U}}(\Gamma T\wedge X)^G \right)
%\xleftarrow{\cong} \pi^{G}_q(\Gamma T\wedge X) \xleftarrow{\cong}
%\pi_q^G(T\wedge X)\]
\end{proof}

\begin{lem}\label{lem:smashfixed}
Let $n,s\in \NN$ such that $s<n$, let $T$ be a $C_{p^n}$-$\Omega$-spectrum and 
and let $X$ be a pointed $C_{p^n}$-CW-complex with trivial
$C_{p^s}$-action. Then 
\begin{equation}\label{eq:smashfixedfibrant}
r:T\wedge X\xrightarrow{\sim} Q(T\wedge X)
\end{equation}
induces a $C_{p^n}/C_{p^s}$-equivalence
\[
T^{C_{p^s}}\wedge X\xrightarrow{\sim} Q(T\wedge X)^{C_{p^s}}.
\]
\end{lem}

\begin{proof}
We first treat the case $X = C_{p^n}/C_{p^r+}$ with $s\le r\le n$. 
Furthermore, suppose $C$ is a subgroup of $C_{p^n}$
containing $C_{p^s}$. We need to show
\[
\pi_q^{C/C_{p^s}}(T^{C_{p^s}}\wedge X)\to \pi_q^{C/C_{p^s}}(Q(T\wedge X)^{C_{p^s}})
\]
is an isomorphism for all $q\in \ZZ$. We first assume $C$ is contained in $C_{p^r}$.
By lemma \ref{lem:homotopygroupsrestriction}
(ii) it suffices to show 
\begin{equation}
\mathrm{res}^{C_{p^n}/C_{p^s}}_{C/C_{p^s}}\, T^{C_{p^s}}\wedge X\to
\mathrm{res}^{C_{p^n}/C_{p^s}}_{C/C_{p^s}}\, Q(T\wedge X)^{C_{p^s}}
\end{equation}
is a $C/C_{p^s}$-equivalence. By lemma \ref{lem:homotopygroupsrestriction} (i)
\[
\mathrm{res}^{C_{p^n}}_C\, r: \mathrm{res}^{C_{p^n}}_C\, T\wedge X\to
\mathrm{res}^{C_{p^n}}_C\, Q(T\wedge X)
\]
is an equivalence of $C$-spectra and both $\mathrm{res}^{C_{p^n}}_C\,T$ 
and $\mathrm{res}^{C_{p^n}}_C\, Q(T\wedge X)$ are $C$-$\Omega$-spectra.
We then have the following commutative diagram
\[
\begin{tikzcd}
(\mathrm{res}^{C_{p^n}}_C\, T)^{C_{p^s}}\wedge X
\arrow{r}{\sim}\arrow{d}{\cong}
&(\mathrm{res}^{C_{p^n}}_C\, Q(T\wedge X))^{C_{p^s}}
\arrow{d}{\cong}\\
\mathrm{res}^{C_{p^n}/C_{p^s}}_{C/C_{p^s}}\, T^{C_{p^s}}\wedge \mathrm{res}^{C_{p^n}/C_{p^s}}_{C/C_{p^s}}\, X
\arrow{r}
&\mathrm{res}^{C_{p^n}/C_{p^s}}_{C/C_{p^s}}\, Q(T\wedge X)^{C_{p^s}},
\end{tikzcd}
\]
where the vertical isomorphisms come from theorem \ref{thm:naiveisnotnaive},
since both the source and target have the same underlying naive $C/C_{p^s}$-spectrum,
the upper horizontal arrow is a $C/C_{p^s}$-equivalence by lemma
\ref{lem:smashfixednonequivariant}, because $C$ acts trivially on $X$,
and the lower horizontal map is the map from the statement. This yields the claim.

We now assume that $C$ contains $C_{p^r}$ and $C\neq C_{p^r}$. Then
$C$ acts freely on $X$ away from the basepoint and we have a diagram
\[
\begin{tikzcd}
\pi_q^{C/C_{p^s}}(T^{C_{p^s}}\wedge X)
\arrow{r}\arrow{d}[swap]{\cong}{\mathrm{res}^{C/C_{p^s}}_1}
&\pi_q^{C/C_{p^s}}(Q(T\wedge X)^{C_{p^s}})
\arrow{d}[swap]{\cong}{\mathrm{res}^{C/C_{p^s}}_1}\\
\left(\pi_q^{\{1\}}(T^{C_{p^s}}\wedge X)\right)^{C/C_{p^s}}
\arrow{r}
&\left(\pi_q^{\{1\}}(Q(T\wedge X)^{C_{p^s}})\right)^{C/C_{p^s}},
\end{tikzcd}
\]
where the vertical arrows are isomorphisms by lemma
\ref{lem:switchfixedpoints} and the horizontal arrows are induced by the map
of the statement. Finally, the lower horizontal map is an isomorphism by lemma
\ref{lem:homotopygroupsrestriction} (i),
since  by lemma \ref{lem:smashfixednonequivariant} 
$\mathrm{res}^{C_{p^n}/C_{p^s}}_1\, T^{C_{p^s}}\wedge X
\to \mathrm{res}^{C_{p^n}/C_{p^s}}_1\, Q(T\wedge X)^{C_{p^s}}$ 
is an equivalence of non-equivariant spectra. This proves the statement
for the case $X = C_{p^n}/C_{p^r+}$.

The rest of the proof is now completely analogous to the proof of
lemma \ref{lem:smashfixednonequivariant} with the only difference
being that in this case we use cofiber sequences of the form
\[
S^{m-1}\wedge C_{p^n}/C_{p^r+}\to D^m \wedge C_{p^n}/C_{p^r+}\to
S^m\wedge C_{p^n}/C_{p^r+}
\]
and
\[
X_{m-1}\to X_m\to X_m/X_{m-1}\cong \bigvee_{j\in J} S^m \wedge C_{p^n}/C_{p^{r_j}+},
\]
where $X_m$ denotes the $m$-skeleton of $X$ and $s\le r, r_j\le n$.
\end{proof}
\begin{rem}\label{rem:smashfixed}
In the setting of the previous lemma we consider the commutative square
\[
\begin{tikzcd}
\pi_q^{\{1\}}(T^{C_{p^s}}\wedge X)
\arrow{r}\arrow{d}{\cong}
&\pi_q^{C_{p^s}}(T\wedge X)
\arrow{d}{\cong}\\
\pi_q^{\{1\}}(Q(T\wedge X)^{C_{p^s}})
\arrow{r}{\cong}
&\pi_q^{C_{p^s}}(Q(T\wedge X)),
\end{tikzcd}
\]
where the vertical isomorphism are induced by
$r:T\wedge X\xrightarrow{\sim} Q(T\wedge X)$
and the lower vertical isomorphism is given by
lemma \ref{lem:omegaspectrafixedpoints}. It remains
to explain the upper horizontal map. Note that $\RR^\infty$
is a complete universe for the trivial group
and we let $\mathcal{U}$ be the countable sum of the
regular $C_{p^s}$-representation, so that $\mathcal{U}$
is a complete $C_{p^s}$-universe. Then $\RR^\infty$ embeds in 
$\mathcal{U}$. If $U$ is a trivial $C_{p^s}$-representation,
we have the equality
\begin{equation}\label{eq:remarksmashfixed}
\Omega^U(T(U)^{C_{p^s}}\wedge X) = (\Omega^UT(U)\wedge X)^{C_{p^s}},
\end{equation}
since $C_{p^s}$ acts trivially on $X$. We then obtain the map
$\pi_q^{\{1\}}(T^{C_{p^s}}\wedge X)
\to
\pi_q^{C_{p^s}}(T\wedge X)$ by applying $\pi_q$ to \eqref{eq:remarksmashfixed}
and then passing to colimits over $\mathcal{P}(\RR^\infty)$ and $\mathcal{P}(\mathcal{U})$.
It is then straightforward to check this commutes with the natural pairing
\eqref{eq:homotopygroupspairing}, hence if $T$ is a ring $C_{p^s}$-spectrum and $X$ is a monoid
in $\mathbf{Top}_\ast$, this map becomes a map of graded rings.
\end{rem}


%\begin{lem}\label{lem:cycretract}
%\todo{Fix this statement to involve the free loop space}
%This will be the lemma with cyclic nerve and deformation retraction onto the circle group.
%
%The diagram
%\begin{equation}\label{eq:multiplicativity}
%\begin{tikzcd}
%\TT/C_{|i|+}\wedge \TT/C_{|j|+}\arrow{r}\arrow{d}{\simeq}
%&\TT/C_{|i+j|+}\arrow{d}{\simeq}\\
%\left| N^{ \text{cy} }( \Pi, i )_\bullet \right|\wedge \left| N^{ \text{cy} }( \Pi, j )_\bullet \right|
%\arrow{r}{\mu}
%& \left| N^{ \text{cy} }( \Pi, i+j )_\bullet \right|,
%\end{tikzcd}
%\end{equation}
%where the upper horizontal arrow is defined by $zC_{|i|}\wedge wC_{|j|}\mapsto
%zw^{ \frac{1}{|i + j|} }C_{|i + j|}$, is homotopy commutative.
%\end{lem}

\section{Symmetric Spectra}
The natural input for topological Hochschild homology is a symmetric ring spectrum.
We summarize the basic definitions and examples, but only as needed. In particular,
we will not discuss homotopy theory of symmetric spectra. The material in this
section is taken from \cite{hsssymmetricspectra}.

Recall that $\Sigma_n$ denotes the symmetric group on $n$ elements. It acts on $\RR^n$
by permutation of the basis vectors and this action extends to a basepoint preserving action
on the one-point compactification $S^n$. We recall that we choose the point at infinity as the basepoint.
\begin{mydef}
A symmetric spectrum $X$ is a sequence of pointed spaces $\{X_n\}_{n\ge 0}$, together with an action
of $\Sigma_n$ on $X_n$ and maps $\sigma_n:X_n\wedge S^1\rightarrow X_{n+1}$ such that 
the iterated structure map
\[
X_n\wedge S^m\xrightarrow{\sigma_n\wedge\mathrm{id}} X_{n+1}\wedge S^{m-1}
\xrightarrow{\sigma_{n+1}\wedge \mathrm{id}}\ldots \xrightarrow{\sigma_{n+m-2}\wedge\mathrm{id}} 
X_{n+m-1}\wedge S^1\xrightarrow{\sigma_{n+m-1}}
X_{n+m}
\]
is $\Sigma_n\times \Sigma_m$-equivariant. Here $\Sigma_n\times \Sigma_m$
acts on $X_{n+m}$ via the natural inclusion $\Sigma_n\times \Sigma_m\hookrightarrow \Sigma_{n+m}$.
A map of symmetric spectra $f:X\to Y$ is given by a family of 
$\Sigma_n$-equivariant maps $f_n:X_n\to Y_n$
such that the following diagram commutes for all $n$:
\[
\begin{tikzcd}
X_n\wedge S^1
\arrow{r}{f_n\wedge \mathrm{id}}\arrow{d}{\sigma_n}
&Y_n\wedge S^1
\arrow{d}{\sigma'_n}\\
X_{n+1}\arrow{r}{f_{n+1}}
&Y_{n+1}.
\end{tikzcd}
\]
\end{mydef}

\begin{bsp}[Sphere spectrum]
The symmetric sphere spectrum, which we denote by $\spherespectrum^{\mathrm{sym}}$,
has as $n$-th space the $n$-sphere, $\spherespectrum^{\mathrm{sym}}_n = S^n$.
The $n$-th structure map is given by the canonical homeomorphism $S^n\wedge S^1\cong S^{n+1}$.
\end{bsp}

\begin{bsp}[Suspension spectrum]
Let $X$ be a pointed space. The associated suspension spectrum $\Sigma^\infty X$ is given
by $(\Sigma^\infty X)_n= X\wedge S^n$, where we let $\Sigma_n$ act trivially on $X$.
The $n$-th structure map is given by the canonical homeomorphism
$(X\wedge S^n)\wedge S^1\cong X\wedge S^{n+1}$.
\end{bsp}

\begin{bsp}
Let $T$ be an orthogonal spectrum, i.e. a $G$-spectrum with $G$ the trivial group.
We then put $X_n = T(\RR^n)$. We have the inclusion $\iota_n:\RR^n\to \RR^n\oplus \RR = \RR^{n+1}$
and define the structure map by
\[
X_n\wedge S^1\to X_{n+1}, x\wedge y\to T(\iota_n, y)(x).
\]
Using the naturality of the map $\mathbf{Th}(\RR^n, \RR^{n+m})\to \map(T(\RR^n), T(\RR^{m+n}))$
one sees that the iterated structure maps have the right equivariance properties. 
We call $X$ the underlying symmetric spectrum of $T$.
\end{bsp}

\begin{mydef}[Symmetric ring spectrum]
A symmetric spectrum $\A$ is a \textit{ring spectrum} if it is equipped with the following data.
\begin{enumerate}[(i)]
\item $\Sigma_m\times \Sigma_n$-equivariant multiplication maps
\[
\mu_{m,n}:\A_m\wedge \A_m\to \A_{m+n}
\]
for all $m,n\in \NN_0$, which are associative in the sense that
\[
\begin{tikzcd}[column sep = large]
\A_m\wedge \A_n\wedge \A_q
\arrow{r}{\mu_{m,n}\wedge \mathrm{id}}\arrow{d}{\mathrm{id}\wedge \mu_{n,q}}
&\A_{m+n}\wedge \A_q
\arrow{d}{\mu_{m+n,q}}\\
\A_m\wedge \A_{n+q}
\arrow{r}{\mu_{m, n_q}}
&\A_{m+n+q}
\end{tikzcd}
\]
commutes.
\item A unit map $u:\spherespectrum^{\mathrm{sym}}\to \A$ such
that
\begin{align*}
\A_n\cong \A_n\wedge S^0&\xrightarrow{\mathrm{id}\wedge u_0} \A_n\wedge \A_0\xrightarrow{\mu_{n,0}} \A_n\\
\A_n\cong S^0\wedge \A_n&\xrightarrow{u_0 \wedge\mathrm{id}} \A_0\wedge \A_n\xrightarrow{\mu_{0,n}} \A_n
\end{align*}
is the identity for all $n\in \NN_0$
and the diagram
\[
\begin{tikzcd}[column sep = large]
\A_n\wedge S^1
\arrow{r}{\mathrm{id}\wedge u_1}\arrow{d}{\mathrm{twist}}\arrow[bend left]{rr}{\sigma_n}
&\A_n\wedge \A_1
\arrow{r}{\mu_{n,1}}
&\A_{n+1}
\arrow{d}{\chi_{n,1}}\\
S^1\wedge \A_n
\arrow{r}{u_1\wedge \mathrm{id}}
&\A_1\wedge \A_n
\arrow{r}{\mu_{1,n}}
&\A_{n+1}
\end{tikzcd}
\]
commutes  for all $n\in \NN_0$. Here $\chi_{m,n}\in \Sigma_{m+n}$ is the permutation
defined by
\[
\chi_{m,n}(i) = \begin{cases}
i+n &\text{ for } 1\le i\le m,\\
i-m &\text{ for } m+1\le i\le m+n.
\end{cases}
\]
\end{enumerate}
Additionally, we call $\A$ \textit{commutative} if the following diagram commutes for all $m,n\in \NN_0$:
\[
\begin{tikzcd}
\A_m\wedge \A_n
\arrow{r}{\mu_{m,n}}\arrow{d}{\mathrm{twist}}
&\A_{m+n}\arrow{d}{\chi_{m,n}}\\
\A_n\wedge \A_m\arrow{r}{\mu_{n,m}}
&\A_{m+n}.
\end{tikzcd}
\]
If $\A$ and $\mathbb{B}$ are symmetric ring spectra then a \textit{map of ring spectra}
is a map of symmetric spectra $f:\A\to \mathbb{B}$ such that
\[
\begin{tikzcd}[column sep = large]
\A_m\wedge \A_n
\arrow{r}{f_m\wedge f_n}\arrow{d}{\mu_{m,n}}
&\mathbb{B}_m\wedge \mathbb{B}_n\arrow{d}{\mu'_{m,n}}\\
\A_{m+n}\arrow{r}{f_{m+n}}
&\mathbb{B}_{m+n}
\end{tikzcd}
\]
commutes for all $m,n\in \NN_0$ and $f$ preserves the unit map
in the sense that the following diagram commutes:
\[
\begin{tikzcd}
\spherespectrum^{\mathrm{sym}}
\arrow{r}{u_\A}\arrow[swap]{dr}{u_\mathbb{B}}
&\A\arrow{d}{f}\\
&\mathbb{B}.
\end{tikzcd}
\]
\end{mydef}

\begin{rem}
Just as for equivariant orthogonal spectra there is a smash product of symmetric spectra.
This gives the category of symmetric spectra the structure of a closed symmetric monoidal category
with unit object the sphere spectrum. In this setup the (commutative) ring spectra are exactly the
(commutative) monoid objects. We do not pursue this approach, since the above definition
is shorter and we have no need for the smash product.
\end{rem}

The following example is central in the construction of the topological Hochschild homology
of a ring. Recall that for any abelian group $A$ and any based space $X$ the 
reduced $A$-linearization $A\left[X\right]$ has underlying set
the tensor product of $A$ with the reduced free abelian group generated by $X$.
In other words,
\[
A\left[X\right] = A\otimes \left(\left(\bigoplus_{x\in X}\ZZ\right)/M\right),
\]
where $M$ is the submodule generated by the basepoint of $X$.
It is equipped with the quotient topology with respect to the map
\[
\coprod_{n\in \NN_0} A^n\times X^n\to A\left[X\right], (a_1,\ldots, a_n, x_1,\ldots, x_n)\mapsto \sum_{i = 1}^n a_i\cdot x_i.
\]
Here we give $A$ the discrete topology.
\begin{bsp}[Eilenberg-MacLane spectrum]
For an abelian group $A$ its \textit{Eilenberg-MacLane spectrum} $\mathbb{H}A$ has as $n$-th space
$(\mathbb{H}A)_n = A\left[S^n \right]$ and structure map
\[
\sigma_n:A\left[S^n\right]\wedge S^1\to A\left[S^{n+1}\right], 
\left(\sum_{i = 1}^n a_i\cdot x_i\right)\wedge t\mapsto \left(\sum_{i = 1}^n a_i\cdot (x_i\wedge t)\right).
\]
The $\Sigma_n$-action is induced by the $\Sigma_n$-action on $S^n$.
If $A$ is additionally a (commutative) ring, then $\mathbb{H}A$ is a (commutative) symmetric ring spectrum
with the multiplication map given by
\[
A\left[S^m\right]\wedge A\left[S^n\right]\to A\left[S^{m+n}\right], 
\left(\sum_{i=1}^m a_i\cdot x_i\right)\wedge \left(\sum_{i=1}^n b_i\cdot y_i\right)\mapsto
\left(\sum_{i,j} a_ib_j\cdot(x_i\wedge y_j)\right)
\]
and unit map by
\[
S^n\to A\left[S^n\right], x\mapsto 1\cdot x.
\]
One sees that this construction is functorial and that is a group homomorphism $f:A\to B$
induces a map of symmetric spectra $\mathbb{H}A\to \mathbb{H}B$. This is
a map of ring spectra if $f$ is a ring homomorphism.
Finally, by \cite[Theorem~11.4, \pno~295]{mccord} we have $\tilde H_k(S^n;A)\cong \pi_k\left(A\left[S^n\right]\right)$,
showing that $A\left[S^n\right]$ is an Eilenberg-MacLane space of type $(A,n)$.
\end{bsp}

\chapter*{Introduction}
\addcontentsline{toc}{chapter}{Introduction}
This thesis concerns itself with topological Hochschild homology,
which is a refinement for ordinary Hochschild homology.
More precisely, one can associate to a ring $A$ (or more generally, a symmetric ring spectrum)
 the topological Hochschild spectrum $T(A)$, which is a $\TT$-orthogonal
spectrum, where $\TT$ denotes the circle group. If $A$ is an $\mathbb{F}_p$-algebra,
\cite[Proposition~1.4.6, \pno~14]{hesselholtacta} proves there is a commutative diagram
\[
\begin{tikzcd}
\pi_n^{\{1\}}(T(A))
\arrow{d}{d}\arrow{r}
&\mathrm{HH}_n(A)
\arrow{d}{B}\\
\pi_{n+1}^{\{1\}}(T(A))
\arrow{r}
&\mathrm{HH}_{n+1}(A),
\end{tikzcd}
\]
where $d$ is a differential, $B$ is Connes' operator
on Hochschild homology and the horizontal maps are
linearization operators.

The homotopy groups of $T(A)$ come equipped with
a great deal of algebraic structure. We define
\[\mathrm{TR}^n_q(A;p) = \pi_q^{C_{p^{n-1}}}(T(A)).\]
Then there are operators
\begin{align*}
&F:\mathrm{TR}^n_q(A;p)\rightarrow \mathrm{TR}^{n-1}_q(A;p) &(\mathrm{Frobenius}),\\
&V:\mathrm{TR}^{n-1}_q(A;p)\rightarrow \mathrm{TR}^{n}_q(A;p) &(\mathrm{Verschiebung}),\\
&d:\mathrm{TR}^n_q(A;p)\rightarrow \mathrm{TR}^n_{q+1}(A;p),\\
&R:\mathrm{TR}^n_q(A ;p)\to \mathrm{TR}^{n-1}_q(A;p) &(\mathrm{Restriction}),\\
&[-]_n :A\to \mathrm{TR}^n_0(A;p) &(\text{Teichm\"uller}),
\end{align*}
where $d$ is a differential and these satisfy certain relations (see lemma
\ref{lem:wittcomplexrelations} and lemma \ref{lem:restrictionrelations}
for a precise statement). The differential gives $\mathrm{TR}^n_\ast(A;p)$
the structure of a differential graded ring for each $n$ and the restriction maps are maps of
differential graded rings, giving $\mathrm{TR}^\bullet_\ast(A;p)$
the structure of a pro-differential graded ring.

For ordinary Hochschild homology, the Hochschild-Kostant-Rosenberg
theorem (see for example \cite[Theorem~9.4.7, \pno~322]{weibel})
states that for a smooth $k$-algebra $A$ satisfying some finiteness conditions
there is an isomorphism of differential graded rings between the Hochschild homology
$\mathrm{HH}_{\ast}(A)$ and
the de Rham complex $\Omega^\ast_{A/k}$, where $k$ denotes a noetherian ring.
A similar result holds for $\mathrm{TR}^\bullet_\ast$ if
we assume that $A$ is a smooth $\mathbb{F}_p$-algebra
and we replace the de Rham complex by the de Rham-Witt complex
$W_\bullet\Omega^\ast_A$ of Bloch-Deligne-Illusie (see \cite[Theorem~B, \pno~3]{hesselholtacta}).

Both statements can be proven by the following strategy.
First, one establishes the result for the base case $A = \mathbb{F}_p$.
Then one calculates a polynomial extension formula for both sides,
i.e. one determines $\mathrm{HH}_\ast(A[x_1,\ldots, x_n])$ in terms of
$\mathrm{HH}_\ast (A)$ and $\Omega^\ast_{A[x_1,\ldots, x_n]/k}$
in terms of $\Omega^\ast_{A/k}$ (respectively $\mathrm{TR}^\bullet_\ast(A[x_1,\ldots, x_n];p)$
in terms of $\mathrm{TR}^\bullet_\ast(A;p)$
and $W_\bullet \Omega^\ast_{A[x_1,\ldots, x_n]}$ in terms of
$W_\bullet \Omega^\ast_{A}$). Finally, one proves that both
statements have inheritance properties under \'etale extensions
and uses that any smooth $\mathbb{F}_p$-algebra
is \'etale over a polynomial $\mathbb{F}_p$-algebra.

We focus on the step that computes the polynomial extension.
More precisely, the aim of this thesis is to prove a polynomial extension formula
for the topological Hochschild homology of the group ring of the infinite cyclic group $A[C_\infty]$, 
i.e. we want to compute $\mathrm{TR}^\bullet_\ast(A[C_\infty];p)$
in terms of $\mathrm{TR}^\bullet_\ast(A;p)$.
For Hochschild homology we have by
\cite[Theorem~0.1, \pno~368]{wghochschilddescent}
the isomorphism $\mathrm{HH}_\ast(A[C_\infty])\cong \mathrm{HH}_\ast(A)\otimes_A A[C_\infty]$.
Now, the inclusion of the constant terms $A\to A[C_\infty]$ induces a map
\[f:\mathrm{TR}^\bullet_\ast(A;p)\to \mathrm{TR}^\bullet_\ast(A[C_\infty];p)\]
of pro-differential graded rings.
The naive guess would then be to prove the map
\[
\mathrm{TR}^n_\ast(A;p)\otimes_A A[C_\infty]\to \mathrm{TR}^n_\ast(A[C_\infty];p),
\]
which sends $[1]_n\otimes x$ to $[x]_n$ and $a\otimes 1$ to $f(a)$ is an isomorphism,
but one the left hand side the expression $d([1]_n\otimes x)$ does not make
sense, since $A[C_\infty]$ is not equipped with a differential. Another
reasonable guess is to take the map of differential graded rings
\[
\mathrm{TR}^n_\ast(A;p)\otimes \Omega^\ast_{\ZZ[C_\infty]}\to \mathrm{TR}^n_\ast(A[C_\infty];p),
\]
which sends $a\otimes 1$ to $f(a)$ and $[1]_n\otimes x$ to $[x]_n$. This is almost correct,
but the left hand side does not admit a Verschiebung operator which satisfies
the relations of lemma \ref{lem:wittcomplexrelations}. Instead, we have to enlarge
the source a little bit and obtain the following statement from \cite[Theorem~2, \pno~139]{hesselholtwhitehead}
(but see also
\cite[Theorem~C, \pno~4]{hmmixed}, where the corresponding statement for $A[x]$ is given). 
The main goal of this thesis is to give a complete proof of this theorem.
%For a fixed odd prime $p$, the fixed points of
%topological Hochschild homology are related by maps
%\[
%F,R:T(A)^C_{p^n}\to T(A)^{C_{p^{-1}}},
%\]
%called Frobenius and restriction. These maps can be used
%to construct the topological cyclic homology spectrum
%$TC(A;p)$. One of the main features of topological
%Hochschild homology and topological cyclic homology
%is the existence of trace maps
%
%which play an important role in computations in $K$-theory.
%
%We do not concern ourself with topological cyclic homology
%in this thesis. Instead, our goal is to understand the homotopy
%groups $\pi^{C_{p^n}}_q(T(A[C_\infty]))$ in terms of
%the homotopy groups $\pi^{C_{p^n}}_q(T(A))$
%In fact, we do even more. If $A$ is commutative,
%$T(A)$ is a ring spectrum and the restriction map
%is a map of ring spectra. If we define $\mathrm{TR}^n_q (A;p)= \pi_q^{C_{p^n-1}}(T(A))$,
%then $\mathrm{TR}^\bullet_\ast(A;p)$ has the structure of a pro-graded ring.
%Furthermore, it is possible to equip $\mathrm{TR}^\bullet_\ast(A;p)$ with a
%differential such that it obtains the structure of a pro-differential
%graded ring.
%
%In \cite{hmmixed} $\mathrm{TR}^\bullet_\ast(A;p)$
%is computed in terms of $\mathrm{TR}^\bullet_\ast(A[x];p)$.
%In \cite[Theorem~2, \pno~139]{hesselholtwhitehead} the corresponding
%statement for $\mathrm{TR}^\bullet_\ast(A[C_\infty];p)$ is given without
%proof. The main objective of this thesis is to prove the last statement.



\begin{thm}
%The inclusion of the constant terms $A\to A[C_\infty]$ induces
%a map of pro-differential graded rings
%\[
%f:\mathrm{TR}^\bullet_\ast(A;p)\to \mathrm{TR}^\bullet_\ast(A[C_\infty];p).
%\]
Choose a generator $x\in C_\infty$ and assume
$A$ is a $\ZZ_{(p)}$-algebra. The map $$\bigoplus_{j\in \ZZ} \left(\mathrm{TR}^n_q(A;p)\oplus \mathrm{TR}^n_{q-1}(A;p) \right)
\oplus \bigoplus_{s=0}^{n-1}\bigoplus_{j\in \ZZ\setminus p\ZZ}
\left( \mathrm{TR}^{s}_q(A;p)\oplus \mathrm{TR}^{s}_{q-1}(A;p)\right)
\rightarrow \mathrm{TR}^n_q(A[C_\infty];p)$$
which sends
$$(a_j, b_j , a_{j'}^{(s)} ,b_{j'}^{(s)}),$$
to
\begin{equation}\label{eq:maintheoremsummands1}
f(a_j)[x]^j_n + f(b_j)d([x]^j_n)
+V^{n-s}(f(a_{j'}^{(s)})[x]^{j'}_{s}) + dV^{n-s}(f(b_{j'}^{(s)})[x]^{j'}_{s})
\end{equation}
 is an isomorphism.
\end{thm}
We remark here that the first two summands of \eqref{eq:maintheoremsummands1}
correspond exactly to the elements of $\mathrm{TR}^n_\ast(A;p)\otimes \Omega^\ast_{\ZZ[C_\infty]}$.

We give a brief overview of the structure of this thesis.
Chapter 1 summarizes basic facts from equivariant stable homotopy
theory and fixes notation. This material is well-known, but scattered across the literature.
We also prove several technical statements, which we will need in chapter 4.

Chapter 2 briefly discusses the theory of cyclic spaces, which
form a generalization of simplicial spaces. Cyclic spaces have the
additional feature that their geometric realization come equipped with
a canonical $\TT$-action. It is for this reason that the topological
Hochschild spectrum is a $\TT$-spectrum. We also discuss the cyclic
nerve of a topological monoid, which plays an important part in the 
comparison of $T(A)$ with $T(A[C_\infty])$.
We end this chapter by computing the cyclic nerve of the integers.

In chapter 3 we construct the topological Hochschild spectrum,
the pro-differential graded ring $\mathrm{TR}^\bullet_\ast(A;p)$
and the algebraic operators described above.
Most of this material is contained in the papers
\cite{hesselholtacta}, \cite{hmperfect} and \cite{hmmixed}.
We try to be detailed in order to give a unified treatment.

Chapter 4 finally contains the computation of $\mathrm{TR}^\bullet_\ast(A[C_\infty];p)$
in terms of $\mathrm{TR}^\bullet_\ast(A;p)$.
The proof is modeled for the most part on the proof
given in \cite[Section~3, \pno~18-23]{hmmixed}.
\section*{Notations and conventions.}
\addcontentsline{toc}{section}{Notations and conventions.}
\subsection*{Category theory.}
\addcontentsline{toc}{subsection}{Category theory.}
We use some notions from the theory of enriched categories.
Most of these are basic, e.g. the definition of a category
enriched over a (closed) symmetric monoidal category, enriched
functors and enriched natural transformations. One more
advanced concept we need for our discussion of the 
smash product of spectra is the enriched Kan extension of enriched functors.
We refer to \cite{kellyenriched} for a treatment of enriched categories.

If a category  $\mathcal{V}$ has products and a final object, this gives
$\mathcal{V}$ the structure of a symmetric monoidal category.
In this case we call $\mathcal{V}$ cartesian monoidal.
If $(\mathcal{V},\wedge, \ast)$ is a symmetric monoidal category,
then we usually refer to categories and functors enriched over $\mathcal{V}$
as $\mathcal{V}$-categories and $\mathcal{V}$-functors. 
We denote (enriched) categories by symbols such as $\mathcal{C}, \mathcal{D}
, \mathcal{I}$ and  $\mathcal{J}$.
Specific categories are usually denoted by boldface letters.
We will often use the expressions morphism
and map interchangeably. Finally, if $F:\mathcal{J}\to \mathcal{C}$
and $\alpha:\mathcal{I}\to \mathcal{J}$
are enriched functors, we denote by $\alpha^\ast F$ the composite
\[
\mathcal{I}\xrightarrow{\alpha} \mathcal{J}\xrightarrow{F} \mathcal{C}
\]
and we interpret this as a change of index categories for $F$.

\subsection*{Topology.}
\addcontentsline{toc}{subsection}{Topology.}
By space we mean a compactly generated weak Hausdorff (CGWH) space.
A comprehensive treatment of CGWH spaces
can be found in \cite{stricklandcgwh}.
We denote the category of (pointed) spaces by $\mathbf{Top}$ and $\mathbf{Top}_\ast$.
We then have the basic fact that $\mathbf{Top}$ is cartesian closed
and $\mathbf{Top}_\ast$ is closed symmetric monoidal. In the latter case,
the monoidal structure comes from the smash product and the unit is $S^0$.
The internal hom is given by the (pointed) mapping space, which we denote by
$\map(X,Y)$. For (pointed)
spaces $X$ and $Y$, a compact subset $K\subset X$ and an open subset
$V\subset Y$ we use the notation $M(K,V)$ for the subset $\{f\in \map(X,Y):f(K)\subset V\}$.
Sets of this form are open in $\map(X,Y)$.

Throughout this thesis $G$ will denote a topological group. We will always assume
subgroups to be closed. In the context
of stable homotopy theory we additionally assume $G$ to be a compact Lie group.
Our main cases of interest are the circle group $\TT$ and its finite cyclic subgroups
$C\subset \TT$. If we want to emphasize the order of a cyclic group we write
$C_a$, with $a$ a non-negative integer or $a = \infty$. The circle group
has the property that the map
\[
\TT/C_a\to \TT, z\cdot C_a\mapsto z^a
\]
is a homeomorphism. Its inverse is given by taking $a$-th roots
and we denote this map by $\rho_a:\TT\to \TT/C_a$.

For a compact Lie group $G$ the expression $G$-\textit{representation} refers to
an orthogonal representation of finite or countable dimension, i.e. a real inner product
space $U$ on which $G$ acts smoothly via linear isometries. If $V$ is an inner product
space containing $U$, we denote by $V-U$ the orthogonal complement.
Moreover, if $U$ is a real inner product space of finite dimension, we have the
one-point compactification $S^U$. We take the point at infinity as the basepoint.
If $G$ acts on $U$, this action extends to a $G$-action
on $S^U$. We interpret $S^n$ as the one-point compactification
of $\RR^n$, where $G$ acts trivially. Finally, we have the basic fact 
\[
S^U\wedge S^V\cong S^{U\oplus V}
\]
for finite dimensional inner product spaces $U$ and $V$.

\subsection*{Homotopy colimits and natural modules.}
\label{sec:naturalmodules}
\addcontentsline{toc}{subsection}{Homotopy colimits and natural modules.}
We use the model described in \cite{duggerhocolim} for the homotopy colimit.
This is constructed as follows. Let $F:\mathcal{I}\to \mathbf{Top}$ be a small diagram.
The simplicial replacement of $F$ is the simplicial space $\mathrm{srep}\, F$,
which has $n$-simplices
\begin{equation}\label{eq:simplicialreplacement}
(\mathrm{srep}\, F)_n = \coprod_{i_n\to i_{n-1}\to\ldots\to i_1\to i_0} F(i_n)
\end{equation}
and $\hocolim_{\mathcal{I}} F$ is defined as the geometric realization of this
simplicial space. A description of the face and degeneracy maps can be found
in \cite[\pno~16]{duggerhocolim}. If the target category is $\mathbf{Top}_\ast$ instead
of $\mathbf{Top}$, we replace the coproduct in \eqref{eq:simplicialreplacement}
by a wedge (which is of course the coproduct in $\mathbf{Top}_\ast$).

A natural module is a pair $(\mathcal{I}, F)$, where $\mathcal{I}$ is a small category
and $F:\mathcal{I} \to \mathbf{Top}_\ast$ is a functor. A morphism $(\alpha, f):(\mathcal{I}, F)
\to (\mathcal{J}, G)$ consists of a functor $\alpha:\mathcal{I}\to \mathcal{J}$
and a natural transformation $f:F\to \alpha^\ast G$. In some situations where the indexing
categories are clear we also refer only to the natural transformation $f:F\to \alpha^\ast G$
as the morphism of natural modules. We denote the corresponding
category by $\mathbf{Mod}_{\mathbf{Top}_\ast}$.
There always exists a natural map
\[
\underset{I}{\hocolim} \, \alpha^\ast G\to \underset{J}{\hocolim} \,  G,
\]
and we can compose this with the map induced by $f$ to obtain
a map
\[
\underset{\mathcal{I}}{\hocolim}\, F \xrightarrow{f_\ast}\underset{I}{\hocolim} \, \alpha^\ast G\to \underset{J}{\hocolim} \,  G.
\]
This turns the homotopy colimit into a functor
\[
\hocolim: \mathbf{Mod}_{\mathbf{Top}_\ast}\to \mathbf{Top}_\ast
\]
\subsection*{Algebra.}
\addcontentsline{toc}{subsection}{Algebra.}
We denote by $p$ an \textit{odd} prime. We use the symbols $\NN, \NN_0$ and $\ZZ$
for the natural numbers, non-negative integers and integers respectively. We write
$\ZZ_{(p)}$ for the localization at the prime ideal $(p)$, that is we invert all integers
not divisible by $p$. We usually denote rings by $A$. Unless otherwise stated, ring
means commutative ring with unit. Undecorated tensor products are taken
over $\ZZ$. Finally, we can interpret $\NN_0$ as a category with an arrow
$m\to n$ iff $m\le n$. By a pro-object in a category $\mathcal{C}$
we mean a functor $F:\NN^{\mathrm{op}}\to \mathcal{C}$.
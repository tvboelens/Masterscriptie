\chapter{Topological Hochschild homology for the group ring of the infinite cyclic group.}
\section{Preliminaries on differential graded rings and the de Rham complex.}
We recall some basic facts on differential graded rings, which we will need
in the computation of $\mathrm{TR}^\bullet_\ast(A[C_\infty];p)$. We fix a ground ring
$k$ and assume $A$ is a $k$-algebra. Recall (see for example \cite[\pno~386]{eisenbud})
that the module of K\"ahler
differentials $\Omega^1_{A/k}$ is equipped with a $k$-derivation $\delta:A\to \Omega^1_{A/k}$
satisfying the following universal property. Given an $A$-module $M$ and a $k$-derivation
$d:A\to M$, there is a unique $A$-linear map $f:\Omega^1_{A/k}\to M$ such that
the following diagram commutes:
\[
\begin{tikzcd}
A\arrow{r}{\delta}\arrow{dr}{d}
&\Omega^1_{A/k}\arrow{d}{f}\\
&M.
\end{tikzcd}
\]

From the module of K\"ahler differentials one can construct the de Rham
complex $\Omega^\ast_{A/k}$, which also satisfies a universal property. We describe
this universal property by constructing the de Rham complex as an initial object
of a certain category. We denote by $\mathbf{DG}_{A/k}$ the category with
objects pairs $(E_\ast,\lambda)$, where $E_\ast$ is a differential graded $k$-algebra 
indexed over $\NN_0$
and $\lambda:A\to E_0$ is a map of $k$-algebra's. We then put
$\Omega^\ast_{A/k} = \Lambda^\ast_A \Omega^1_{A/k}$
and equip it with a differential $d:\Lambda^n_A \Omega^1_{A/k}\to
\Lambda^{n+1}_A \Omega^1_{A/k}$ via the formula
\[
d(adx_1\wedge\cdots \wedge dx_n) = da\wedge dx_1\wedge\cdots \wedge dx_n.
\]
One can verify that this makes $\Omega^\ast_{A/k}$ into a differential
graded $k$-algebra. By definition we have $\Omega^0_{A/k} = A$
and by using induction and the universal property of $\Omega^1_{A/k}$
one can show that $(\Omega^\ast_{A/k}, \mathrm{id}_A)$ is an initial
object of $\mathbf{DG}_{A/k}$. We write $\Omega^\ast_{A}$
instead of $\Omega^\ast_{A/\ZZ}$ and we note the formula's
$\Omega^1_{k[x_1,\ldots, x_n]} = \bigoplus_{i=1}^nk[x_1,\ldots, x_n] dx_i$
and $\Omega^1_{S^{-1}A/k} = S^{-1}A\otimes_A \Omega^1_{A/k}$
from \cite[Proposition~16.1, \pno~387]{eisenbud} and
\cite[Proposition~16.9, \pno~397]{eisenbud}. These facts imply
that $\Omega^\ast_{\ZZ[C_\infty]}$ is in degree $0$
given by $\ZZ[C_\infty]$, in degree $1$ by $\ZZ[C_\infty] dx$
and trivial in higher degrees. We will use this fact later. 

Finally, we recall that the tensor product $E_\ast\otimes_k E'_\ast$  of
two differential graded $k$-algebra's
$E_\ast$ and $E'_\ast$ is the tensor product of the underlying
cochain complexes with differential
\[
d(x\otimes y) = dx\otimes y + (-1)^m x\otimes dy,
\]
where $x\in E_m$ and $y\in E'_n$ and multiplication
\[
(x\otimes y)\cdot (w\otimes z) = (-1)^{nq}xw\otimes yz,
\]
where $w\in E_q$ and $z\in E'_r$. The tensor product is
the coproduct in the category of differential graded $k$-algebra's.
\section{The comparison between $T(A)$ and $T(A[C_\infty])$.}
We first relate $T(A)$ to $T(A[C_\infty])$. We do this in a slightly more general
setting.
\begin{mydef}
For a monoid $M$ in $\mathbf{Top}$ and a symmetric ring spectrum
$\A$, we let $\A[M] = \A\wedge M_+$. 
%If $M$ is a monoid in $\mathbf{Top}_\ast$,
%we put $\A[M] = \A\wedge M$.
\end{mydef}
Of course, $\A[M]$ is also a ring spectrum with multiplication map
\[\begin{tikzcd}[column sep = large]
%(\A\wedge M)\wedge(\A\wedge M)&\cong (\A\wedge\A)\wedge (M\wedge M), \\
\A_n\wedge M_+\wedge \A_q\wedge M_+
\arrow{r}[swap]{\cong}{\mathrm{twist}}
&(\A_n\wedge \A_q)\wedge (M\times M)_+
\arrow{r}{\mu_{n,q}\wedge \mu_M}
&\A_{n+q}\wedge M_+
\end{tikzcd}
\]
and unit map $u_{\A\wedge M_+}: \spherespectrum^{\mathrm{sym}}\to \A\wedge M_+$
given in level $n$ by
\[
\begin{tikzcd}[column sep = large]
S^n\arrow{r}{\cong}
&S^n\wedge S^0
\arrow{r}{u_{\A_n}\wedge u_M}
&\A_n\wedge M_+.
\end{tikzcd}
\]
Also, if $M$ is discrete, we have
in the case of a ring $A$  a canonical isomorphism
$(\mathbb{H}A)[M] \cong \mathbb{H}(A[M])$. We now construct a map
\begin{equation}\label{eq:thhpolynomialextension}
\varphi: T(\A)\wedge \lvert CN(M)_\bullet\rvert_+\to T(\A[M])
\end{equation}
for any monoid $M$ in $\mathbf{Top}$, which is a map
of ring spectra if $\A$ is commutative. Let $[\vec{q}\,]\in \mathcal{I}^{k+1}$.
Then we define
\begin{equation}
\varphi_{[\vec{q}\,],k,U} : G_k(\A,S^U)([\vec{q}\,])\wedge CN(M_+)_k
\to G_k(\A[M],S^U)([\vec{q}\,])
\end{equation}
as the adjoint of the map
\begin{equation}
\begin{tikzcd}
 \map(\spherespectrum^{\mathrm{sym}}_{[\vec{q}\,]}, \A_{[\vec{q}\,]}\wedge S^U)
\wedge \spherespectrum^{\mathrm{sym}}_{[\vec{q}\,]}
\wedge (M_+\wedge\cdots\wedge M_+)
\arrow{d}{\mathrm{ev}\wedge \mathrm{id}}\\
\A_{[\vec{q}\,]}\wedge S^U \wedge( M_+\wedge\cdots\wedge M_+)
\arrow{r}{\cong}
&\A[M]_{[\vec{q}\,]}\wedge S^U,
\end{tikzcd}
\end{equation}
where the second map is given by permutation of the smash factors. This is
natural in $[\vec{q}\,]$ and the induced map on homotopy colimits forms
a map of cyclic spaces. After geometric realization and using
the pointed homeomorphism $|CN(M_+)_\bullet|\cong |CN(M)_\bullet|_+$ this yields
\eqref{eq:thhpolynomialextension}. If $\A$ is commutative,
the diagram
\[
\begin{tikzcd}
G_k(\A, S^U)\wedge CN(M_+)_k\wedge G_k(\A, S^V)\wedge CN(M_+)_k
\arrow{r}\arrow{d}{\varphi_{k,U}\wedge \varphi_{k,V}}
&\alpha_k^\ast G_k(\A, S^{U\oplus V})\wedge CN(M_+)_k
\arrow{d}{\varphi_{k, U\oplus V}}\\
G_k(\A[M], S^U)\wedge  G_k(\A[M], S^V)
\arrow{r}
&\alpha_k^\ast G_k(\A[M], S^{U\oplus V}),
\end{tikzcd}
\]
where the horizontal arrows are the smash product of
\eqref{eq:thhmultiplicationnaturalmodules} and the multiplication of $CN(M_+)_k$, 
commutes, showing that $\varphi$ is multiplicative and using \eqref{eq:thhunit}
it immediately follows that
$\varphi$ preserves the unit map. The following theorem is proven in
\cite[Theorem~7.1, \pno~81]{hmperfect}, but see also \cite[Proposition~3, \pno~81]{hesselholthandbook}.
\begin{thm}\label{thm:polynomialextension}
Let $\mathcal{F}$ denote the family of finite subgroups of $\TT$.
For any ring $A$ the map \eqref{eq:thhpolynomialextension}
is a map of ring $\TT$-spectra and an $\mathcal{F}$-equivalence.
\end{thm}


\section{The comparison between $\mathrm{TR}^\bullet_\ast(A;p)$ and
$\mathrm{TR}^\bullet_\ast(A[C_\infty];p)$.}
%We fix a generator $x\in C_\infty$. 
We now recall and prove the main theorem of this thesis.
The canonical inclusion $A\to A[C_\infty]$
induces a map of ring spectra $T(A)\to T(A[C_\infty])$, which in turn induces a map
of differential graded rings
\[
f:\mathrm{TR}_\ast^\bullet(A;p)\to \mathrm{TR}_\ast^\bullet(A[C_\infty];p),
\]
such that $f$ commutes with the Frobenius and Verschiebung. Furthermore,
one can check that the map 
\[
T(A)(U)\to T(A[C_\infty])(U)
\]
commutes with $r_1$, compare \eqref{eq:thhrestriction}, so that $f$ commutes with $R$.
This means $f$ is a map of pro-differential graded rings. Finally,
it follows directly from the construction that the composition of $f$
with the Teichm\"uller map of $A$ is equal to the Teichm\"uller map of $A[C_\infty]$.
Recall from the introduction that we want to prove the following theorem.
\begin{thm}\label{thm:maintheorem}
Choose a generator $x\in C_\infty$. If $A$ is a $\ZZ_{(p)}$-algebra, the map
$$\bigoplus_{j\in \ZZ} \left(\mathrm{TR}^n_q(A;p)\oplus \mathrm{TR}^n_{q-1}(A;p) \right)
\oplus \bigoplus_{s=0}^{n-1}\bigoplus_{j\in \ZZ\setminus p\ZZ}
\left( \mathrm{TR}^{s}_q(A;p)\oplus \mathrm{TR}^{s}_{q-1}(A;p)\right)
\rightarrow \mathrm{TR}^n_q(A[C_\infty];p)$$
which sends
$$(a_j, b_j , a_{j'}^{(s)} ,b_{j'}^{(s)}),$$
to
\begin{equation}\label{eq:maintheoremsummands}
f(a_j)[x]^j_n + f(b_j)d([x]^j_n)
+V^{n-s}(f(a_{j'}^{(s)})[x]^{j'}_{s}) + dV^{n-s}(f(b_{j'}^{(s)})[x]^{j'}_{s})
\end{equation}
 is an isomorphism.
\end{thm}
\begin{rem}
Using the relations from lemma \ref{lem:wittcomplexrelations}, lemma
\ref{lem:restrictionrelations} and the relation $Fd([x]_n) = [x]_{n-1}^p d([x]_{n-1})$,
which is proven in \cite[Lemma~1.5.6, \pno~19]{hesselholtacta}, one can 
explicitly compute formula's for $F,V,R$ and $d$ and  define
a multiplication for 
\[
\bigoplus_{j\in \ZZ} \left(\mathrm{TR}^n_q(A;p)\oplus \mathrm{TR}^n_{q-1}(A;p) \right)
\oplus \bigoplus_{s=0}^{n-1}\bigoplus_{j\in \ZZ\setminus p\ZZ}
\left( \mathrm{TR}^{s}_q(A;p)\oplus \mathrm{TR}^{s}_{q-1}(A;p)\right),
\]
such that it becomes a pro-differential graded ring and the
isomorphism from theorem \ref{thm:maintheorem} preserves all the structure.
We refer to \cite[Section~4.2, \pno~24-27]{hmmixed} for the specific formula's.
\end{rem}




\begin{proof}
By theorem \ref{thm:polynomialextension} there is an $\mathcal{F}$-equivalence
of ring $\TT$-spectra
\[
\varphi: T(\A)\wedge |CN(C_\infty)_\bullet|_+\to T(\A[C_\infty]),
\]
where $\mathcal{F}$ denotes the family of finite subgroups of $\TT$.
This yields for each $n\in \NN_0$ an isomorphism of differential graded rings
\begin{equation}\label{eq:original}
%\rho_{p^{n-1}}^\ast \left(R\left(T(A)\wedge \left|N^{\text{cy}}(\Pi)_\bullet\right|\right)\right)^{C_{p^{n-1}}}\rightarrow \rho_{p^{n-1}}^\ast \left( R (T ( A [x,x^{-1}] ) )\right)^{C_{p^{n-1}}}.
\pi_\ast^{C_{p^{n}}}( T(A)\wedge \left|CN(C_\infty)_\bullet\right|_+ )
\rightarrow \pi_\ast^{C_{p^{n}}}( T(A[C_\infty]) )
\end{equation}
We describe the left hand side. There are canonical isomorphisms
and an $\mathcal{F}$-equivalence of $\TT$-spectra
\begin{equation}
\begin{tikzcd}[column sep = scriptsize]
\underset{i\in \ZZ}{\bigvee} T(A)\wedge \TT_{i+}
\arrow{r}{\cong} 
&T(A)\wedge\underset{i\in \ZZ}{\bigvee} \TT_{i+}
\arrow{r}{\cong}
&T(A)\wedge\left(\underset{i\in \ZZ}{\coprod} \TT_{i}\right)_+
\arrow{r}{\sim}
&T(A)\wedge\lvert CN(C_\infty)_\bullet \rvert_+,
\end{tikzcd}
\end{equation}
where the $\mathcal{F}$-equivalence is obtained from theorem \ref{thm:cycretract} and we recall that
the action on $\TT_i$ is given by \eqref{eq:circlemodifiedaction}. 



%\begin{align*}
%\bigoplus_{i \in \ZZ} \pi_\ast^{ C_{p^{n}}}
%\left( T(A)\wedge \TT_{i+}\right) 
%&\cong  \pi_\ast^{ C_{p^{n}}}
%\left( \bigvee_{i \in \ZZ} T(A)\wedge \TT_{i+}\right) \\
%&\cong  \pi_\ast^{ C_{p^{n}}}
%\left(  T(A)\wedge \bigvee_{i \in \ZZ}\TT_{i+}\right) \\
%&\cong  \pi_\ast^{ C_{p^{n}}}
%\left(  T(A)\wedge \left(\coprod_{i \in \ZZ}\TT_i\right)_+\right) \\
%&\cong  \pi_\ast^{ C_{p^{n}}}
%\left(  T(A)\wedge \left| CN_\bullet(C_\infty)\right|_+ \right).
%\end{align*}
%The final isomomorphism is obtained from theorem \ref{thm:cycretract} and we recall that
%the action on $\TT_i$ is given by \eqref{eq:circlemodifiedaction}.
%Recall from (?) the decomposition
%$$N^{\text{cy}}(\Pi)_\bullet\cong \bigvee_{i\in \ZZ} N^{\text{cy}}(\Pi,i)_\bullet
%\cong \bigvee_{i\in \NN_0} N^{\text{cy}}(\Pi,i)_\bullet\vee 
%\bigvee_{i\in \NN} N^{\text{cy}}(\Pi,-i)_\bullet,$$
%where $N^{\text{cy}}(\Pi,i)_\bullet$ is the cyclic submonoid with $k$-simplices
%$$N^{\text{cy}}(\Pi,i)_k = \left\{ x^{ i_0 }\wedge \cdots \wedge x^{ i_k }: 
%\text{ sgn}( i_j ) = \text{sgn}( i )  \text{ for all } j \text{ and } \sum_{ j = 0}^k i_j = i \right\}.$$

%Combining this decomposition with lemma~\ref{lem:wedgefixed} and the fact that restriction
%commutes with wedges, we obtain a $\TT$-equivalence
%\begin{align}\label{eq:decomp}
%\begin{split}
%\bigvee_{i \in \ZZ}\rho_{p^{n-1}}^\ast \left(R\left(T(A)\wedge 
%\left|N^{\text{cy}}(\Pi,i)_\bullet\right|\right)\right)^{C_{p^{n-1}}}
%\xrightarrow{\sim} 
%&\rho_{p^{n-1}}^\ast 
%\left( R \left( \bigvee_{i \in \ZZ} (T(A) \wedge 
%\left| N^{\text{cy}}(\Pi,i)_\bullet \right|) \right) \right)^{C_{p^{n-1}}}\\
%\xrightarrow{\sim} 
%&\rho_{p^{n-1}}^\ast \left(R\left(T(A)\wedge 
%\left|N^{\text{cy}}(\Pi)_\bullet\right|\right)\right)^{C_{p^{n-1}}}.
%\end{split}
%\end{align}
%Plugging in \eqref{eq:decomp} into \eqref{eq:original} and a
We can write the index set of the left hand term
as the disjoint union:
\[
\ZZ = \{p^nj:j\in \ZZ\}\cup\bigcup_{s = 0}^{n-1}\{p^sj:j\in \ZZ\setminus p\ZZ\}.
\]
Thus, we have an $\mathcal{F}$-equivalence
of $\TT$-spectra
\begin{equation}\label{eq:padic}
\begin{tikzcd}[column sep = scriptsize]
T(A)\wedge \underset{j\in \ZZ}{\bigvee}
\TT_{p^{n}j+}\vee
\overset{n-1}{\underset{s= 0}{\bigvee}}\underset{j\in \ZZ\setminus p\ZZ }{\bigvee}
T(A)\wedge \TT_{p^sj+} 
\arrow{d}{\cong}\\
\underset{j\in \ZZ}{\bigvee}T(A)\wedge \TT_{p^{n}j+}\vee
\overset{n-1}{\underset{s= 0}{\bigvee}}\underset{j\in \ZZ\setminus p\ZZ }{\bigvee}
T(A)\wedge \TT_{p^sj+} 
\arrow{r}{\sim} 
&T(A)\wedge \left| CN(C_\infty)_\bullet\right|_+ 
\arrow{r}[swap]{\varphi}{\sim}
&T(A[C_\infty])
\end{tikzcd}
\end{equation}
The $C_{p^n}$-action on $\bigvee_{j\in \ZZ} \TT_{p^nj+}$ is trivial
and we have the equality $\rho_{p^n}^\ast \bigvee_{j\in \ZZ} \TT_{p^nj+} = 
\bigvee_{j\in \ZZ} \TT_{j+}$, so by
lemma~\ref{lem:smashfixed} and lemma
\ref{lem:homotopygroupsrestriction} we obtain an equivalence
of non-equivariant spectra
\begin{equation}\label{eq:pdivisible1}
\rho_{p^{n}}^\ast T(A)  ^{ C_{ p^{ n  } } }
\wedge 
\underset{j \in \ZZ}{\bigvee}\TT_{j+} =
\rho_{p^{n}}^\ast \left(T(A)  ^{ C_{ p^{ n  } } }
\wedge 
\underset{j \in \ZZ}{\bigvee}\TT_{p^nj+}\right)
\xrightarrow{\sim}
\rho_{p^n}^\ast Q\left(T(A)\wedge \underset{j \in \ZZ}{\bigvee}\TT_{p^nj+}\right)^{C_{p^n}}.
\end{equation}

%the equivalence of $\TT$-spectra
%\begin{align}\label{eq:padic}%Take a look at the indices here! Fix the alignment later.
%\begin{split}
%\bigvee_{j \in \ZZ}\rho_{p^{n-1}}^\ast \left(R\left(T(A)\wedge 
%\left|N^{\text{cy}}(\Pi,p^{n-1}j)_\bullet\right|\right)\right)^{C_{p^{n-1}}}
%\vee \bigvee_{s=0}^{n-1}
%\bigvee_{j\in I_p} \rho_{p^{n-1}}^\ast \left(R\left(T(A)\wedge 
%\left|N^{\text{cy}}(\Pi,p^{s}j)_\bullet\right|\right)\right)^{C_{p^{n-1}}}\\
%\xrightarrow{\sim} \rho_{p^{n-1}}^\ast R(T(A[x,x^{-1}]))^{C_{p^{n-1}}}.
%\end{split}
%\end{align}

By lemma~\ref{lem:separatefixed}, if $n,s \in \NN_0$ are such that 
$s < n$, there is an isomorphism of $\TT$-spectra
\begin{equation}
\rho^\ast_{ p^{ n } } \left( Q \left( T(A) \wedge 
\TT_{p^sj+} \right) \right)^{ C_{ p^{n}} }
\cong 
\rho^\ast_{ p^{ n - s } } \left(  \rho_{ p^s }^\ast \left( Q 
\left( T(A) \wedge \TT_{p^sj+} \right) 
\right)^{ C_{ p^s } } \right)^{ C_{ p^{ n - s} } }.
\end{equation}
Furthermore,  the $C_{ p^s }$-action on 
$\TT_{p^sj+}$ is trivial and we have
equalities $\TT_{j+} = \rho^\ast_{p^s}\TT_{p^sj+}$
and 
\[
\rho_{ p^s }^\ast T(A) ^{ C_{ p^s } } \wedge 
\rho_{ p^s }^\ast \TT_{p^sj+}  = 
\rho_{ p^s }^\ast \left(T(A) ^{ C_{ p^s } } \wedge 
 \TT_{p^sj+}\right) ,
\]
thus  combining lemma~\ref{lem:smashfixed} and lemma
\ref{lem:homotopygroupsrestriction}  
yields a $C_{p^n}/C_{p^{n-s}}$-equivalence
\begin{equation}\label{eq:pdivisible2}
\begin{tikzcd}
  Q \left( \rho_{ p^s }^\ast  
  T(A) ^{ C_{ p^s } } \wedge 
 \TT_{j+}\right) ^{ C_{ p^{ n - s} } }
\arrow[equal]{r}
& Q\left(\rho_{ p^s }^\ast 
\left( 
  T(A)  ^{ C_{ p^s } } \wedge 
  \TT_{p^sj+} \right)  \right)^{ C_{ p^{ n  - s} } }
\arrow{d}{\simeq}\\
&Q\left( \rho_{ p^s }^\ast \left( Q 
 \left( T(A) \wedge \TT_{p^sj+} \right)^{ C_{ p^s } } \right) \right)^{ C_{ p^{ n  - s} } }.
\end{tikzcd}
\end{equation}
%Finally, we have the equality
%$\TT_{j+} = \rho^\ast_{p^s}\TT_{p^sj+}$. 
If we apply $C_{p^n}$-fixed points to \eqref{eq:padic},  use lemma \ref{lem:wedgefixed} and
plug in \eqref{eq:pdivisible1} and \eqref{eq:pdivisible2},
we obtain a zig-zag of equivalences of non-equivariant
spectra
\begin{equation}\label{eq:mainmapfixedpoints}
\begin{tikzcd}
\mathrm{res_1}\, \rho_{p^{n}}^\ast T(A)  ^{ C_{ p^{ n  } } }
\wedge 
\underset{j \in \ZZ}{\bigvee}\TT_{j+}
\vee
\overset{n-1}{\underset{s=0}{\bigvee}} \underset{j\in \ZZ\setminus p\ZZ}{\bigvee}\mathrm{res_1} \,  Q \left( \rho_{ p^s }^\ast  
  T(A) ^{ C_{ p^s } } \wedge 
 \TT_{j+}\right) ^{ C_{ p^{ n - s} } }
 \arrow{d}{\simeq}\\
\mathrm{res}_1\, Q(T(A)[C_\infty])^{C_{p^n}}\\
\mathrm{res}_1\,T(A[C_\infty])^{C_{p^n}}.
\arrow{u}[swap]{\simeq}
 \end{tikzcd}
\end{equation}
We have that $\mathrm{res_1}\, \rho_{p^{n}}^\ast T(A)  ^{ C_{ p^{ n  } } }
\wedge 
\bigvee_{j \in \ZZ}\TT_{j+}$ is a ring spectrum and intuitively the zig-zag
\eqref{eq:mainmapfixedpoints} identifies it with a subring of $\mathrm{res}_1\,T(A[C_\infty])^{C_{p^n}}$.
More precisely, if we take homotopy groups of \eqref{eq:mainmapfixedpoints} and apply lemma 
\ref{lem:omegaspectrafixedpoints} and lemma \ref{lem:wedgefixed},
\eqref{eq:original} becomes
\[
\begin{tikzcd}
 \pi_\ast^{\{1\}}\left(
\rho_{p^{n}}^\ast T(A)  ^{ C_{ p^{ n  } } }
\wedge 
\underset{j \in \ZZ}{\bigvee}\TT_{j+}\right)
\oplus \overset{ n - 1 }{\underset{ s = 0}{\bigoplus}}
\underset{ j\in \ZZ\setminus p\ZZ }{\bigoplus}
\pi_\ast^{C_{p^{n-s}}}  \left(
 \rho^\ast_{ p^s } 
T(A) ^{ C_{ p^s } }\wedge 
\TT_{j+}  \right)
\arrow{d}{\cong}\\ 
\pi_\ast^{C_{p^{n}}} ( T (A[ C_\infty] ) )
\end{tikzcd}
\]
and the restriction of this map to the summand $ \pi_\ast^{\{1\}}\left(
\rho_{p^{n}}^\ast T(A)  ^{ C_{ p^{ n  } } }
\wedge 
\bigvee_{j \in \ZZ}\TT_{j+}\right)$ is a map of differential graded rings
by remark \ref{rem:smashfixed}, 
hence identifies it with a sub-differential graded ring of
$\pi_\ast^{C_{p^{n}}} ( T (A[ C_\infty] ) )$.
%The claim will follow from the next two lemma's. 
The next lemma then identifies
the first two summands in \eqref{eq:maintheoremsummands} with the 
elements of $ \pi_\ast^{\{1\}}\left(
\rho_{p^{n}}^\ast T(A)  ^{ C_{ p^{ n  } } }
\wedge 
\bigvee_{j \in \ZZ}\TT_{j+}\right)$. The second lemma then
shows the final two
summands of \eqref{eq:maintheoremsummands}
correspond to the 
the elements of $ \bigoplus_{ s = 0}^{ n - 1 }
\bigoplus_{ j\in \ZZ\setminus p\ZZ }
\pi_\ast^{C_{p^{n-s}}}  \left(
 \rho^\ast_{ p^s } 
T(A) ^{ C_{ p^s } }\wedge 
\TT_{j+}  \right)$.
Proposition \ref{prop:thhplocalization} says that $ \rho^\ast_{ p^s } 
T(A) ^{ C_{ p^s } }$ satisfies the assumptions of the second lemma.
\end{proof}
%There are $\TT$-equivalences
%
%Since restriction commutes
%with wedges, we obtain a $\TT$-equivalence
%$$\rho_{p^{n-1}}^\ast\bigvee_{i \in \NN_0}Q^{\mathcal{U}}(T(A)\wedge N^{cy}(\Pi,i))^{C_{p^{n-1}}}
%\rightarrow \rho_{p^{n-1}}^\ast Q^{\mathcal{U}}(T(A)\wedge N^{cy}(\Pi))^{C_{p^{n-1}}}.$$ 
%
\begin{lem}
The map of differential graded rings
\[
\mathrm{TR}^{n+1}_\ast(A;p)\otimes \Omega^\ast_{\ZZ[C_\infty]}\rightarrow 
\mathrm{TR}^{n+1}_\ast(A[C_{\infty}];p),
\]
which maps $a\otimes 1$ to $f(a)$ and $1\otimes x$ to  $[x]_{n+1}$ is an isomorphism onto the
sub-differential graded ring $ \pi_\ast^{\{1\}}\left(
\rho_{p^n}^\ast T(A) ^{ C_{ p^n } }
\wedge 
\bigvee_{j \in \ZZ}\TT_{j+}\right)$.
\end{lem}
\begin{proof}
We first show that the image of the map is contained in 
\[
 \pi_\ast^{\{1\}}\left(
\rho_{p^n}^\ast  T(A) ^{ C_{ p^n } }
\wedge 
\bigvee_{j \in \ZZ}\TT_{j+}\right).
\]
 Since $a \otimes 1$ and $1 \otimes x $
are generators, it suffices to show 
\[
f(a),  [x]_{n+1}  \in 
 \pi_\ast^{\{1\}}\left(
\rho_{p^n}^\ast  T(A)^{ C_{ p^n } }
\wedge 
\bigvee_{j \in \ZZ}\TT_{j+}\right).
\]
For $f(a)$ this is clear. Consider the following diagram
\[
\begin{tikzcd}
C_\infty\arrow{r}\arrow{d}
&A[C_\infty]\arrow{d}\\
\lvert CN(C_\infty)_\bullet\rvert_+
\arrow{r}\arrow{d}{1\wedge\mathrm{id}}
&\lvert CN(A[C_\infty])_\bullet\rvert
\arrow{dd}{\Delta}\\
\lvert CN(A)_\bullet \rvert\wedge\lvert CN(C_\infty)_\bullet \rvert_+
\arrow{d}{\Delta\wedge \Delta}
\\
\rho_{p^n}^\ast \left( \lvert CN(A)_\bullet \rvert\right)^{C_{p^n}}\wedge
\rho_{p^n}^\ast \left(\lvert CN(C_\infty)_\bullet \rvert_+\right)^{C_{p^n}}
\arrow{r}{\varphi}\arrow{d}
& \rho_{p^n}^\ast \left( \lvert CN(A[C_\infty])_\bullet\rvert\right)^{C_{p^n}}
\arrow{d}\\
\rho_{p^n}^\ast (T(A)(0))^{C_{p^n}} \wedge
\rho_{p^n}^\ast \left(\lvert CN(C_\infty)_\bullet \rvert_+\right)^{C_{p^n}}
\arrow{r}{\varphi}
&\rho_{p^n}^\ast (T(A[C_\infty])(0))^{C_{p^n}},
\end{tikzcd}
\]
where the top horizontal arrow is given by the canonical inclusion, the top vertical
arrows are the inclusion of the vertices, the bottom vertical arrows are given
by \eqref{eq:teichmueller} and $1:S^0\to |CN(A)_\bullet|$ denotes
the unit map. Here we regard $A$ as a monoid with basepoint $0$.
If we apply $\pi_0$ the right hand column becomes $[-]_{n+1}$. The claim
follows chasing $x$ through the diagram above and by combining propositions \ref{prop:cyclicnervefixedpoints}
and \ref{prop:cyclicnervegrading}, which imply that the composition
\[
C_\infty\to |CN(C_\infty)_\bullet| \xrightarrow{\Delta}
\rho_{p^n}^\ast \left( \lvert CN(C_\infty)_\bullet \rvert\right)^{C_{p^n}}
\xrightarrow{\sim} \rho_{p^n}^\ast \left(\coprod_{j\in \ZZ}\TT_j\right)^{C_{p^n}}
\]
maps $x$
into $\rho_{p^n}^\ast \TT_{p^n}^{C_{p^n}} = \TT_1$.

%
%Recall that 
%$\lambda ( [x]_n ) = \omega_n ( x )$. Note that $A[ x, x^{ -1 } ] = \pi_0
%((HA[ x, x^{ -1 } ])_0)$ and $\Pi = \pi_0 ( \Pi )$, since $\Pi$ is discrete. 
%Then $\omega_n$ is given by the composite
%\begin{align}
%\Pi = \pi_0(\Pi)&\xrightarrow{ g_\ast } 
%\pi_0 ( (HA[ x, x^{ -1 } ])_0 ) \\
%&\rightarrow \pi_0 ( N^{ \text{cy} }( ( HA[ x, x^{ -1 } ])_0 ) )\\
%&\xrightarrow{ \Delta } \pi_0 \left( \rho_{ p^{ n - 1 } }^\ast N^{ \text{cy} }
%( ( HA[ x, x^{ -1 } ])_0 )^{ C_{ p^{ n - 1 } } } \right) \label{eq:Eilenbergcyclicbar} \\
%&\to \pi_0 \left( \rho_{ p^{ n - 1 } }^\ast T( A[ x, x^{ -1 } ] )^{ C_{ p^{ n - 1 } } } \right).
%\end{align}
%Here $g$ is the map $$\Pi\rightarrow A[ x, x^{ -1 }]\rightarrow A[ x, x^{ -1 }]
%\left[ S^0 \right]=( HA[ x, x^{ -1 } ])_0$$ and \eqref{eq:Eilenbergcyclicbar} is the inclusion of the vertices. Now let
%$h$ be the composite
%\begin{align}
%\Pi &\rightarrow N^{ \text{cy} } (\Pi)\cong N^{ \text{cy} } (\Pi) \wedge S^0 \\
%&\rightarrow N^{ \text{cy} } (\Pi) \wedge A\left[ S^0 \right] = 
%N^{ \text{cy} } (\Pi) \wedge ( HA )_0\\
%&\rightarrow N^{ \text{cy} } (\Pi) \wedge N^{ \text{cy} } ( ( HA )_0 ) 
%\end{align}
%Then the following diagram commutes:
%$$
%\begin{tikzcd}
%\Pi \arrow{r}{g} \arrow{d}{h}&H( A[ x, x^{ -1 } ] )_0 \arrow{d}\\
%N^{ \text{cy} } (\Pi) \wedge N^{ \text{cy} } ( ( HA )_0 )  \arrow{d}{ \Delta \wedge \Delta }
%&N^{ \text{cy} }( H( A[ x, x^{ -1 } ] )_0 ) \arrow{d}{ \Delta } \\
%\rho_{ p^{ n - 1 } }^\ast N^{ \text{cy} } (\Pi)^{ C_{ p^{ n - 1 } } } 
%\wedge \rho_{ p^{ n - 1 } }^\ast N^{ \text{cy} } ( ( HA )_0 ) ^{ C_{ p^{ n - 1 } } }  \arrow{r}
%&\rho_{ p^{ n - 1} }^\ast N^{ \text{cy} }( H( A[ x, x^{ -1 } ] )_0 ) ^{ C_{ p^{ n - 1 } } } 
%\end{tikzcd}
%$$
%The claim then follows after applying $\pi_0$ and the following commutative diagram:
%$$
%\begin{tikzcd}
%\pi_0 \left( \rho_{ p^{ n - 1 } }^\ast N^{ \text{cy} } (\Pi)^{ C_{ p^{ n - 1 } } } 
%\wedge \rho_{ p^{ n - 1 } }^\ast N^{ \text{cy} } ( ( HA )_0 ) ^{ C_{ p^{ n - 1 } } } \right)
%\arrow{d}  \arrow{r} 
%&\pi_0 \left( \rho_{ p^{ n - 1} }^\ast N^{ \text{cy} }( H( A[ x, x^{ -1 } ] )_0 ) ^{ C_{ p^{ n - 1 } } }  \right)
%\arrow{d} \\
%\pi_0 \left( \rho_{ p^{ n - 1 } }^\ast 
%\left( T(A) \wedge N^{ \text{cy} } (\Pi) \right) ^{ C_{ p^{ n - 1 } } }  \right)  \arrow{r}
%&\pi_0 \left( \rho_{ p^{ n - 1 } }^\ast \left( T(A[ x, x^{ -1 }] ) \right) ^{ C_{ p^{ n - 1 } } } \right) 
%\end{tikzcd}
%$$
%\todo{This does not fit with the above proof, it needs to be adjusted}

We define a differential  
\[
\delta:\tilde H_\ast\left(\left( \coprod_{j\in \ZZ} \TT_j \right)_+\right)\to \tilde H_{\ast+1}\left(\left( \coprod_{j\in \ZZ} \TT_j \right)_+\right)
\]
 as the composition
\begin{align*}
\tilde  H_0\left(\left( \coprod_{j\in \ZZ} \TT_j \right)_+\right)
&\to \tilde H_1(\TT_+)\otimes  \tilde H_0\left(\left( \coprod_{j\in \ZZ} \TT_j \right)_+\right)\\
&\xrightarrow{\times}
\tilde  H_1\left(\TT_+\wedge\left( \coprod_{j\in \ZZ} \TT_j \right)_+\right)\\
&\xrightarrow{\mathrm{action}}
\tilde H_1\left(\left( \coprod_{j\in \ZZ} \TT_j \right)_+\right),
\end{align*}
where the first map is tensoring with the fundamental class
corresponding to the counterclockwise orientation
of the circle. This turns $\tilde H_\ast\left(\left( \coprod_{j\in \ZZ} \TT_j \right)_+\right)$
into a differential graded ring.
To see this, denote by $x_i$ the class corresponding to the point
\[
(1,i)\in \coprod_{j\in \ZZ} \TT_j
\]
and let 
\[
\iota_i:\TT_{i+}\to \left( \coprod_{j\in \ZZ} \TT_j \right)_+
\]
be the inclusion into the coproduct. Since $(1,i)\cdot (1,j) = (1,i+j)$, $x_i\cdot x_j = x_{i+j}$.
Furthermore, if $[\TT_i]$ is the fundamental class of $\TT_i$ corresponding to the counterclockwise
orientation of the circle, then $\delta x_i = i\cdot (\iota_i)_\ast([\TT_i])$. It is straightforward to check
that $x_i\cdot (\iota_j)_\ast([\TT_j]) = (\iota_{i+j})_\ast([\TT_{i+j}])$ and we compute
\begin{align*}
\delta(x_i\cdot x_j) &= \delta x_{i+j} =    (i+j)\cdot (\iota_{i+j})_\ast([\TT_{i+j}]) \\
&= j\cdot  x_i\cdot (\iota_j)_\ast([\TT_j])+ i\cdot(\iota_i)_\ast([\TT_i])\cdot x_j\\
&= x_i\cdot \delta x_j + (\delta x_j)\cdot x_i,
\end{align*}
showing the Leibniz rule holds.
%We let $x_i\in \tilde H_0\left(\left( \coprod_{i\in \ZZ} \TT/C_{|i|} \right)_+\right)$ be the image of 
%$[ C_{|i|} ]\in \tilde H_0( \TT/C_{|i|} )$ under the map induced by the inclusion
%$$\TT/C_{|i|} \rightarrow \left( \coprod_{i\in \ZZ} \TT/C_{|i|} \right)_+$$
From the remarks above we see that the map of differential graded rings 
\[
\Omega^\ast_{\ZZ[C_\infty]}\to
\tilde H_\ast\left(\left( \coprod_{j\in \ZZ} \TT_j \right)_+\right),
\]
 which maps $x$ to $x_1$, is an isomorphism. 

%The map
%in homology induced by the product
%$$\mu:\TT/C_{|i|}\times \TT/C_{|j|} \rightarrow \TT/C_{|i+j|}$$
%maps the cycle $C_{|i|} \otimes C_{|j|}$ to the cycle $C_{|i+j|}$, hence 
%$x_i x_j = x_{i+j}$. It follows that the map is an isomorphism in degree $0$. To prove
%it is an isomorphism in degree $1$, it is enough to show
%\todo{Complete this part. This uses products in homology.}


We now show the map from the statement is an isomorphism.
From the skeletal filtration of 
$X=\left( \coprod_{j\in \ZZ} \TT_j \right)_+$ we obtain a spectral sequence
\[
E^1_{s,t}=\pi^{\{1\}}_{s+t}(\rho_{p^n}^\ast T(A)^{C_{p^n}}
\wedge X_s/X_{s-1})
\Rightarrow 
\pi^{\{1\}}_\ast\left(\rho_{p^n}^\ast T(A)^{C_{p^n}}
\wedge \left( \coprod_{j\in \ZZ} \TT_j \right)_+\right).
\]
There are natural isomorphisms of abelian groups
\[
\mathrm{TR}^{n+1}_t(A;p)\otimes\pi_s(X_s/X_{s-1}) \xrightarrow{\cong}
\pi^{\{1\}}_{s+t}(\rho_{p^n}^\ast T(A)^{C_{p^n}}
\wedge X_s/X_{s-1})\]
and
\[
\mathrm{TR}^{n+1}_t(A;p)\otimes\pi_s(X_s/X_{s-1}) \xrightarrow{\cong} 
\mathrm{TR}^{n+1}_t(A;p)\otimes \tilde H_s(X_s/X_{s-1})
\]
induced by the natural pairing from lemma \ref{lem:pairing}
and the Hurewicz homomomorphism respectively.
This gives an isomorphism of chain complexes
\[
E^1_{\ast, t}\cong \mathrm{TR}^{n+1}_t(A;p)\otimes \tilde C_\ast(X),
\]
where $\tilde C_\ast(X)$ denotes the reduced cellular chain complex of $X$
and since the homology of $\left( \coprod_{i\in \ZZ} \TT_j \right)_+$ 
is torsion free, the universal coefficient theorem yields a natural isomorphism
of abelian groups
\[
E^2_{s,t}\cong \mathrm{TR}^{n+1}_t(A;p)\otimes
 \tilde H_s\left(\left( \coprod_{j\in \ZZ} \TT_j \right)_+\right).
 \]
We see that $E^2_{s,t}=0$ if $s\neq0,1$. Thus, we obtain a diagram
\[
\begin{tikzcd}
0\arrow{d}
&0\arrow{d} \\
\mathrm{TR}^{n+1}_t(A;p)\otimes \ZZ[C_\infty]
\arrow{d}\arrow{r}{\cong}
&E^2_{0,t}
\arrow{d} 
\\
\left(\mathrm{TR}^{n+1}_\ast\otimes \Omega^\ast_{\ZZ[C_\infty]}\right)_t
\arrow{r}\arrow{d}
&\pi^{\{1\}}_t\left(\rho_{p^n}^\ast T(A)^{C_{p^n}}\wedge \left( \coprod_{j\in \ZZ} \TT_j \right)_+\right)\arrow{d}\\
\mathrm{TR}^{n+1}_{t-1}\otimes \Omega^1_{\ZZ[C_\infty]} 
\arrow{d}\arrow{r}{\cong} 
&E^2_{1,t-1}
\arrow{d} \\
0
&0
\end{tikzcd}
\]
where the right vertical exact sequence arises from 
the spectral sequence and the outer two horizontal maps 
are isomorphisms induced from 
$\Omega^\ast_{\ZZ[C_\infty]}\xrightarrow{\cong} 
\tilde H_\ast\left(\left( \coprod_{j\in \ZZ} \TT_j \right)_+\right)$ 
after tensoring with $\mathrm{TR}^{n+1}_t(A;p)$ and 
$\mathrm{TR}^{n+1}_{t-1}(A;p)$ respectively. A diagram 
chase shows that the middle horizontal map is the map from 
the statement, hence the claim follows from the five lemma.
\end{proof}

\begin{lem}
Let $T$ be a ring $\TT$-spectrum such that $\pi_q^{\{1\}}(T)$ is a $\ZZ_{(p)}$-module for all $q$. Let $j\in \ZZ$ such that $(j,p) = 1$ and 
$$\iota:S^0\rightarrow \TT_{j+}$$
the map which sends the non-basepoint to $1$.
For all $q$ and $\nu\ge 0$ the map $$V^\nu \iota_\ast +d V^\nu \iota_\ast :\pi^{\{1\}}_q(T)\oplus \pi^{\{1\}}_{q-1}(T)\rightarrow \pi^{C_{p^\nu}}_q(T\wedge \TT_{j+})$$
is an isomorphism.
\end{lem}


\begin{proof}
Let $X$ be a pointed  $C_{p^\nu}$-CW-complex. The skeletal filtration of $X$ gives rise to a spectral sequence
\[
E^1_{s,t}=\pi^{C_{p^\nu}}_{s+t}(T\wedge (X_s/X_{s-1}))
\Rightarrow \pi_{s+t}^{C_{p^\nu}}(T\wedge X).
\]
If the action of $C_{p^\nu}$ on $X$ is free away from the basepoint, 
lemma \ref{lem:switchfixedpoints} gives a natural isomorphism
of abelian groups
\begin{equation}\label{eq:changefixed}
%\pi^{C_{p^\nu}}_{s+t}(T \wedge X)\xrightarrow{\cong} 
%\left(\pi^{\{1\}}_{s+t}(T \wedge X)\right)^{C_{p^\nu}},
\mathrm{res}^{C_{p^\nu}}_1:\pi^{C_{p^\nu}}_{s+t}(T \wedge X_s/X_{s-1})\xrightarrow{\cong} 
\left(\pi^{\{1\}}_{s+t}(T \wedge X_s/X_{s-1})\right)^{C_{p^\nu}}
\end{equation}
Similar to the previous proof, there are natural isomorphisms
of abelian groups
\[
\pi^{ \{ 1\} }_t(T)\otimes\pi_s(X_s/X_{s-1}) \xrightarrow{\cong}
\pi^{\{1\}}_{s+t}(T\wedge X_s/X_{s-1})
\]
and
\[
\pi^{ \{ 1\} }_t(T)\otimes\pi_s(X_s/X_{s-1}) \xrightarrow{\cong} 
\pi^{ \{ 1\} }_t(T)\otimes \tilde H_s(X_s/X_{s-1})
\]
induced by the natural pairing
and the Hurewicz homomomorphism respectively and these maps
are $C_{p^\nu}$-equivariant. If $\tilde C_\ast(X)$ denotes
the reduced cellular chain complex, these isomorphisms
together with \eqref{eq:changefixed}
 yield an isomorphism of chain complexes
\begin{equation}\label{eq:ffirstpage}
E^1_{t,\ast}\cong \left( \pi^{\{1\}}_t(T)\otimes \tilde C_\ast(X) \right)^{C_{p^\nu}}.
\end{equation}

We specialize to the case $X=\TT_{j+}$, which has a free
$C_{p^\nu}$-action by the assumption $(p,j) = 1$. 
We give it the $C_{p^\nu}$-CW-structure induced from the obvious $C_{p^\nu}$-CW-structure
on a circle with a free $C_{p^\nu}$-action. More precisely, let $g=e^{2\pi i/p^\nu}\in C_{p^\nu}$ be the canonical generator
and define the characteristic maps
\[
\alpha_s: D^s\times C_{p^\nu}\to \TT_j,
\]
as by $\alpha_0(g^n) = g^{jn} $ and $\alpha_1(x, g^n) = g^{jn} e^{\pi i(x+1)j/p^\nu}$. 
Then the $0$-skeleton of $\TT_j$ is $C_{p^\nu}$ and we note for future reference that
the (unpointed) inclusion $\TT_j\to \TT_{j+}$ induces a (pointed) $C_{p^\nu}$-homeomorphism
$\TT_j/C_{p^\nu}\cong X_1/X_0$. 

Let $W(j)$ be the complex of $\ZZ[C_{p^\nu}]$-modules, which
in degrees $s=0,1$ is a free $\ZZ[C_{p^\nu}]$-module with generator $y_s$ and is trivial in all other degrees. We equip $W(j)$ with a differential given by $d(y_1) = 
(g-1)y_0$. Then there is an isomorphism of complexes
\[W(j)\cong\tilde C_\ast(\TT_{j+}).
\]
Since $T$ is a $\TT$-spectrum, the left translation map
$\ell_g:T\to T$ is homotopic to the identity, hence $C_{p^\nu}$ acts trivially on $\pi_t^{\{1\}}(T)$ and
\eqref{eq:ffirstpage} becomes
\begin{equation}
E^1_{s,t}\cong \begin{cases}
\pi^{\{1\}}_t(T)\cdot Ny_s &\text{ for } s=0,1,\\
0 &\text{ for } s\neq 0,1.
\end{cases}
\end{equation}
Here $N=\sum_{c\in C_{p^\nu}} c\in \ZZ[C_{p^\nu}]$ denotes the norm element,
which is a generator of the fixed  module $\ZZ[C_{p^\nu}]^{C_{p^\nu}}$.
We have
$$d(Ny_1) = Nd(y_1) = N(g-1)y_0 = (N-N)y_0 = 0,$$
showing that the $d^1$-differential vanishes and the higher
differentials vanish for degree reasons. Thus, the spectral sequence 
stabilizes at the first page.

The skeleton filtration also yields a spectral sequence
\begin{equation}\label{eq:ssnonequivariant}
E'^1_{s,t} = \pi^{\{1\}}_{s+t}(T\wedge X_s/X_{s+1})\Rightarrow 
\pi^{\{1\}}_{s+t}(T\wedge\TT_{j+})
\end{equation}
and a similar argument as above shows that there is a natural isomorphism of complexes
\begin{equation}\label{eq:firstpage}
E'^1_{\ast,t}\cong \pi^{\{1\}}_t(T)\otimes W(j).
\end{equation}
Since $W(j)$ is a complex consisting of free $\ZZ[C_{p^\nu}]$-, hence by extension, free $\ZZ$-modules, we obtain $E'^2_{0,t}\cong \pi^{\{1\}}_t(T)\cdot y_0$ and 
$E'^2_{1,t}\cong \pi^{\{1\}}_t(T)\cdot Ny_1$ from the universal coefficient theorem.



The map $$V^\nu: \pi^{\{1\}}_\ast(T\wedge \TT_{j+})\rightarrow 
\pi^{C_{p^\nu}}_\ast\left( T\wedge \TT_{j+}\right)$$ induces a map of spectral sequences.
Under the above identifications of the $E^1$-terms this corresponds to multiplication with the norm element
$$N:\pi^{\{1\}}_t(T)\otimes W(j)\rightarrow \left(\pi^{\{ 1 \} }_t(T)\otimes W(j)\right)^{C_{p^\nu}}.$$
To see this, note that by definition of $V$ and by proposition \ref{prop:transfertransitivity}
$V^\nu$ is equal to the transfer map
\[
\mathrm{tr}^{C_{p^\nu}}_1:E^1_{s,t} = \pi_{t+s}^{\{1\}}(T\wedge X_s/X_{s-1})\to 
\pi_{t+s}^{C_{p^\nu}}(T\wedge X_s/X_{s-1}) = E'^1_{s,t}
\]
and by proposition \ref{prop:doublecoset} the following diagram commutes:
\[
\begin{tikzcd}
\pi_{t+s}^{\{1\}}(T\wedge X_s/X_{s-1})
\arrow{r}{\mathrm{tr}^{C_{p^\nu}}_1} \arrow{dr}[swap]{\underset{c\in C_{p^\nu}}{\sum}\ell_{c\ast}}
&\pi_{t+s}^{C_{p^\nu}}(T\wedge X_s/X_{s-1})
\arrow{d}[swap]{\cong}{\mathrm{res}^{C_{p^\nu}}_1}\\
&\left(\pi_{t+s}^{\{1\}}(T\wedge X_s/X_{s-1})\right)^{C_{p^\nu}}.
\end{tikzcd}
\]
We obtain a diagram
$$
\begin{tikzcd}
0\arrow{r} &\pi^{\{1\}}_q(T)\cdot y_0 \arrow{r} \arrow{d} 
&\pi^{\{1\}}_q(T)\oplus \pi^{\{1\}}_{q-1}(T)\arrow{r} \arrow{d} 
&\pi^{\{1\}}_{q-1}(T)\cdot Ny_0 \arrow{r} \arrow{d} & 0\\
0\arrow{r} &\pi^{\{1\}}_q(T)\cdot Ny_0 \arrow{r} &\pi^{C_{p^\nu}}_q(T\wedge \TT_{j+})\arrow{r} &\pi^{\{1\}}_{q-1}(T)\cdot Ny_1\arrow{r} & 0
\end{tikzcd}
$$
with exact rows, where the lower exact sequence results from
\eqref{eq:ffirstpage} and the vertical maps are as follows. The middle map is the map from the statement.
The left vertical map 
sends $x\cdot y_0$ to $x\cdot Ny_0$, hence is an isomorphism and the right vertical map is given by
\begin{equation}\label{eq:mainmap}
\pi^{\{1\}}_q(T)\cdot Ny_0\cong E^2_{0,q}\hookrightarrow \pi^{C_{p^\nu}}_q(T\wedge\TT_{j+})\xrightarrow{d} 
\pi^{C_{p^\nu}}_{q+1}(T\wedge\TT_{j+})\rightarrow E^2_{1,q}\cong \pi_q^{\{1\}}(T)\cdot Ny_1.
\end{equation}
We will show that it takes $x\cdot Ny_0$ to $jx\cdot Ny_1$,
and our assumptions then imply it is an isomorphism, yielding the result.

In fact, we also prove this for the additional case $T = \Sigma^\infty S^q$. We can assume that $T$ is
a $\TT$-$\Omega$-spectrum, since if $r_T:T\xrightarrow{\sim} QT$ is a fibrant replacement,
then 
\[
T\wedge \TT_{j+}\xrightarrow{r_T\wedge \mathrm{id}} QT\wedge\TT_{j+}
\]
is an equivalence of $\TT$-spectra by \cite[Theorem 3.11, \pno~47]{mandellmay}
and we have the following diagram:
\[
\begin{tikzcd}
\pi^{\{1\}}_q(T)\cdot Ny_0
\arrow{r}\arrow{d}{\cong}
&\pi^{C_{p^\nu}}_q(T\wedge\TT_{j+})
\arrow{r}{d}\arrow{d}{\cong}
&\pi^{C_{p^\nu}}_{q+1}(T\wedge\TT_{j+})
\arrow{r}\arrow{d}{\cong}
&\pi_q^{\{1\}}(T)\cdot Ny_1
\arrow{d}{\cong}\\
\pi^{\{1\}}_q(QT)\cdot Ny_0
\arrow{r}
&\pi^{C_{p^\nu}}_q(QT\wedge\TT_{j+})
\arrow{r}{d}
&\pi^{C_{p^\nu}}_{q+1}(QT\wedge\TT_{j+})
\arrow{r}
&\pi_q^{\{1\}}(QT)\cdot Ny_1,
\end{tikzcd}
\]
where the rows are given by \eqref{eq:mainmap}.

If $\pi^{\{1\}}_q(T)$ has no $p$-torsion, it is clearly enough to show that
\begin{equation}\label{eq:mainmaptorsion}
\pi^{\{1\}}_q(T)\cdot Ny_0 \to\pi^{C_{p^\nu}}_q(T\wedge\TT_{j+})\xrightarrow{p^\nu d} 
\pi^{C_{p^\nu}}_{q+1}(T\wedge\TT_{j+})\rightarrow  \pi_q^{\{1\}}(T)\cdot Ny_1
\end{equation}
maps $x\cdot Ny_0$ to $p^\nu jx\cdot Ny_1$. Granting this fact,
we obtain the claim from the following observation. If $x\in \pi^{\{1\}}_q(T)$, then it
is represented by a map $f:S^q\to T(0)$, since $T$ is an $\Omega$-spectrum.
We can extend this to a map of $\TT$-spectra $\Sigma^\infty S^q\to T$,
which we also denote by $f$ and this
gives a diagram
\begin{equation}\label{eq:spherespectrumtorsion}
\begin{tikzcd}
\pi_q^{\{1\}}(\Sigma^\infty S^q)\cdot Ny_0
\arrow{d}\arrow{r}{f_\ast}
&\pi^{\{1\}}_q(T)\cdot Ny_0
\arrow{d}\\
\pi_q^{C_{p^\nu}}(\Sigma^\infty S^q\wedge \TT_{j+})
\arrow{r}{(f\wedge \mathrm{id})_\ast}\arrow{d}{d}
&\pi^{C_{p^\nu}}_q(T\wedge\TT_{j+})
\arrow{d}{d}\\
\pi_{q+1}^{C_{p^\nu}}(\Sigma^\infty S^q\wedge \TT_{j+})
\arrow{r}{(f\wedge \mathrm{id})_\ast}\arrow{d}
&\pi^{C_{p^\nu}}_{q+1}(T\wedge\TT_{j+})
\arrow{d}\\
\pi_q^{\{1\}}(\Sigma^\infty S^q)\cdot Ny_1
\arrow{r}{f_\ast}
&\pi_q^{\{1\}}(T)\cdot Ny_1,
\end{tikzcd}
\end{equation}
%
%\begin{equation}\label{eq:spherespectrumtorsion}
%\begin{tikzcd}
%\pi_q^{\{1\}}(\Sigma^\infty S^q)\cdot Ny_0
%\arrow{r}\arrow{d}{f_\ast}
%&\pi_q^{\{1\}}(\Sigma^\infty S^q\wedge \TT_{j+})
%\arrow{d}{(f\wedge \mathrm{id})_\ast}\arrow{r}{d}
%&\pi_{q+1}^{\{1\}}(\Sigma^\infty S^q\wedge \TT_{j+})
%\arrow{d}{(f\wedge \mathrm{id})_\ast}\arrow{r}
%&\pi_q^{\{1\}}(\Sigma^\infty S^q)\cdot Ny_1
%\arrow{d}{f_\ast}\\
%\pi^{\{1\}}_q(T)\cdot Ny_0
%\arrow{r}
%&\pi^{C_{p^\nu}}_q(T\wedge\TT_{j+})
%\arrow{r}{d}
%&\pi^{C_{p^\nu}}_{q+1}(T\wedge\TT_{j+})
%\arrow{r}
%&\pi_q^{\{1\}}(T)\cdot Ny_1,
%\end{tikzcd}
%\end{equation}
where the columns are given by \eqref{eq:mainmap}. We have 
$\pi_q^{\{1\}}(\Sigma^\infty S^q)\cong \ZZ$ and 
$f_\ast$ maps a generator to $x$. Thus, we can assume $\pi^{\{1\}}_q(T)$ has no $p$-torsion.

We reduce to the non-equivariant homotopy groups. By 
lemma \ref{lem:wittcomplexrelations} (v), (vi) the following diagram commutes:
\[
\begin{tikzcd}
\pi_q^{\{1\}}(T)\cdot y_0
\arrow{r}\arrow{d}{\cdot N}
&\pi_q^{\{1\}}(T\wedge \TT_{j+})
\arrow{d}{V^\nu}\arrow{r}{d}
&\pi_{q+1}^{\{1\}}(T\wedge \TT_{j+})
\arrow{d}{V^\nu}\arrow{r}
&\pi_q^{\{1\}}(T)\cdot Ny_1
\arrow{d}{\cdot N}\\
\pi_q^{C_{p^\nu}}(T)\cdot Ny_0
\arrow{r}
&\pi_q^{C_{p^\nu}}(T\wedge \TT_{j+})
\arrow{r}{p^\nu d}
&\pi_{q+1}^{C_{p^\nu}}(T\wedge \TT_{j+})
\arrow{r}
&\pi_q^{C_{p^\nu}}(T)\cdot Ny_1.
\end{tikzcd}
\]
Here the upper row is the non-equivariant version of \eqref{eq:mainmap},
where the left and right horizontal maps are obtained from the spectral sequence \eqref{eq:ssnonequivariant}.
We note that the right vertical arrow sends $jx\cdot Ny_1$
to $jx\cdot NNy_1 = p^\nu jx \cdot Ny_1$. Thus, it suffices to
show that the upper row sends $x\cdot y_0$ to $jx \cdot Ny_0$.

We fix $x\in \pi_q^{\{1\}}(T)$. Similar to \eqref{eq:spherespectrumtorsion}
there is a diagram 
\[
\begin{tikzcd}
\pi_q^{\{1\}}(\Sigma^\infty S^q)\cdot y_0
\arrow{r}\arrow{d}{f_\ast}
&\pi_q^{\{1\}}(\Sigma^\infty S^q\wedge \TT_{j+})
\arrow{d}{(f\wedge \mathrm{id})_\ast}\arrow{r}{d}
&\pi_{q+1}^{\{1\}}(\Sigma^\infty S^q\wedge \TT_{j+})
\arrow{d}{(f\wedge \mathrm{id})_\ast}\arrow{r}
&\pi_q^{\{1\}}(\Sigma^\infty S^q)\cdot Ny_1
\arrow{d}{f_\ast}\\
\pi^{\{1\}}_q(T)\cdot y_0
\arrow{r}
&\pi^{\{1\}}_q(T\wedge\TT_{j+})
\arrow{r}{d}
&\pi^{\{1\}}_{q+1}(T\wedge\TT_{j+})
\arrow{r}
&\pi_q^{\{1\}}(T)\cdot Ny_1,
\end{tikzcd}
\]
such that $f_\ast$ maps $\mathrm{id}_{S^q}\cdot y_0$ to $x\cdot y_0$.
Hence, it suffices to show that the upper row maps 
$\mathrm{id}_{S^q}\cdot y_0$ to $ \mathrm{id}_{S^q}\cdot jNy_1$.
Consider the maps $p:\TT_j\to \TT_j/C_{p^\nu}\cong X^1/X_0$
and $s_j:\TT\to \TT_j, z\mapsto z^j$. Then \eqref{eq:firstpage} maps
the composite
\[
\varphi: S^{q+1}\cong S^q\wedge \TT
\xrightarrow{\mathrm{id}_{S^q}\wedge s_j} S^q\wedge \TT
\xrightarrow{\mathrm{id}_{S^q}\wedge p} S^q\wedge(X_1/X_0)
\]
to $ \mathrm{id}_{S^q}\cdot jNy_1$. We show that
\[
\pi_q^{\{1\}}(\Sigma^\infty S^q)\cdot y_0
\to
\pi_q^{\{1\}}(\Sigma^\infty S^q\wedge \TT_{j+})
\xrightarrow{d}
\pi_{q+1}^{\{1\}}(\Sigma^\infty S^q\wedge \TT_{j+})
\to
\pi_q^{\{1\}}(\Sigma^\infty S^q\wedge (X^1/X_0))
\]
sends  $\mathrm{id}_{S^q}\cdot y_0$ to the class of $\varphi$.
We choose a representative $\psi:S^{n+1}\to S^n\wedge \TT_+$
of $\sigma\in \pi^{\{1\}}_1(\Sigma^\infty \TT_+)$. Then  $\mathrm{id}_{S^q}\cdot y_0$ 
is mapped to the class of
\begin{align*}
%\begin{tikzcd}
S^q\wedge S^0\wedge S^{n+1}
&\xrightarrow{\mathrm{id}\wedge \iota\wedge \psi}
S^q\wedge \TT_{j+}\wedge S^n\wedge \TT_+\\
&\xrightarrow{\mathrm{action}}
S^q\wedge \TT_{j+}\wedge S^n\\
&\xrightarrow{\mathrm{collapse}} 
S^q\wedge (X_1/X_0)\wedge S^n.
%\end{tikzcd}
\end{align*}
But this is the same as the map
\begin{align*}
%\begin{tikzcd}
S^q\wedge S^{n+1}
&\xrightarrow{\mathrm{id}\wedge \psi}
S^q\wedge S^n\wedge \TT_+
\xrightarrow{\mathrm{collapse}}
S^q\wedge S^n\wedge \TT\\
&\xrightarrow{\mathrm{id}\wedge s_j}
S^q\wedge S^n\wedge \TT
\xrightarrow{\mathrm{id}\wedge p}
S^q\wedge S^n\wedge (X_1/X_0)
%\end{tikzcd}
%\]
\end{align*}
and since 
\[
S^{n+1}\xrightarrow{\psi} S^n\wedge \TT_+\xrightarrow{\mathrm{collapse}}
S^n\wedge \TT
\]
is (stably) homotopic to the identity, the claim follows.
\end{proof}
